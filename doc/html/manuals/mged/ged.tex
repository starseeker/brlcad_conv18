\title{\bf{\huge{
DRAFT *** DRAFT *** DRAFT \\
Users Manual for \\
BRL-CAD Graphics Editor \\
MGED
}}}
\author{
Keith A. Applin \\
Michael J. Muuss \\
Robert J. Reschly \\
{\em US Army Ballistic Research Laboratory} \\
{\em Aberdeen Proving Ground, MD} \\
\and
Alan Collier \\
{\em US Army Foreign Science and Technology Center} \\
{\em Charlottesville, VA} \\
\and
Mike Gigante \\
Ian Overend \\
{\em The Royal Melbourne Institute of Technology} \\
{\em Australia}
}

\date{6-October-1988}

\maketitle

\tableofcontents
\listoffigures

% ---------------------------------------------------------------------------

\chapter{INTRODUCTION}

Computer graphics is one of the fastest growing fields
in the computer industry.
Computer graphics has applications in many diverse areas, from electronic
games to medicine; from cartoons to the space industry.  Just
what is interactive computer graphics and why is it so versatile?
Human visual perception is quite keen and communicating with a
computer is generally faster and easier
with images, rather than with
numbers. Furthermore, by
having the computer continuously updating a display,
the display itself becomes the communications medium.
The user converses with the computer through the display using
devices such as light pens,
mice, data tablets, buttons, and
knobs.  The response of the computer is immediately reflected
on the display,
providing a fast communication channel between person and machine.
This technology is called interactive computer graphics.

As the Army's lead laboratory for vulnerability technology, the
Ballistic Research Laboratory (BRL) constantly performs
analyses for a wide variety of military systems.
Three dimensional computer models of the
physical characteristics of these systems
are vital to these studies.
Since the mid-1960's, BRL has used a solid modeling technique
called Combinatorial Solid Geometry (CSG or COMGEOM)
for representing these models.
The COMGEOM technique uses
Boolean logic operations to combine basic geometric
shapes or primitives to produce complex three-dimensional objects.
The COMGEOM geometric models are processed by
the Geometric Information
For Targets (GIFT)
\cite{gift1,gift2}
and LIBRT
\cite{solid-models}
for use in follow-on engineering analysis.

Geometric models are large collections
of numerical data which have traditionally
been created and edited manually, and analyzed in a batch environment.
The production and modification of geometric models has been a slow,
labor-intensive
process.
In 1980, BRL initiated an effort to improve the response
time of the geometric modeling process by applying interactive
computer graphics techniques.
As a result of this work, BRL
developed the Multi-device Graphics EDitor (MGED),
an interactive editor for solid models
based on the COMGEOM technique.
Using MGED, a designer can build, view, and modify model descriptions
interactively by manipulating the graphical representation,
receiving immediate visual feedback on a graphics display.
MGED replaces the manual method for the production
and modification of geometric models.

Before MGED was built,
existing packages were evaluated with respect to
their utility for the geometric modeling process.
Quite an exhaustive search of commercially available systems
was conducted and none were found which met
the BRL requirements.
A study was initiated to examine the feasibility of producing
the required capability in-house;
a preliminary version of MGED which
quite convincingly demonstrated the
feasibility of such an undertaking \cite{interactive-construction}.
It was then decided to develop MGED into a full production code.
Production MGED code has been used since January 1982 to
build models interactively at BRL.

This report is intended to serve as  a user manual
for the MGED program.
The process of viewing and editing a description using MGED
is covered in detail.  The internal data structure is also covered, as
it is an important part in the overall design.
All the commands will be discussed and a command summary table presented.
Also, a section will be devoted to the hardware interfaces for each
major class of workstations which MGED supports.

\section{Philosophy}

The role of CAD models at BRL differs somewhat from
that of CAD models being built in the automobile and aerospace industries,
resulting in some different design choices
being made in the BRL-CAD software.
Because BRL's main use for these models is to conduct detailed
performance and survivability analyses of large complex vehicles,
it is required that the model of an entire vehicle be completely contained
in a single database suitable for interrogation by the application codes.
This places especially heavy demands on the database software.
At the same time, these analysis codes require less detail
than would be required if NC machining were the primary goal.

At BRL, there are only a small number of primary designers responsible
for the design of a vehicle, and for the construction of the corresponding
solid model.  Together they decide upon and construct the
overall structure of the model,
then they perform the work of building substructures in parallel,
constantly combining intermediate results into the full model database.
Because of the need to produce rapid prototypes (often creating a full design
within a few weeks), there is no time for a separate integration stage;
subsystem integration must be an ongoing part of the design process.

Once an initial vehicle design is completed, there is usually the
need for exploring many alternatives.  Typically, between three and twelve
variations of each design need to be produced, analyzed, and optimized
before recommendations for the final design can be made.
Also, there is a constantly changing definition of performance;
new developments may necessitate rapidly re-evaluating
all the designs of the past several years for trouble spots.

The user interface is designed to be powerful and ``expert friendly'' rather
than foolproof for a novice to use.
However, it only takes about two days for new users to start doing useful
design work with MGED.
True proficiency comes with a few months practice.

Finally, it is vitally important that the software offer the same capabilities
and user interface across a wide variety of display and processor hardware.
Government procurement regulations make single-vendor solutions difficult.
The best way to combat this is with highly portable software.

\section{Displays Supported}

It is important for a CAD system to have a certain degree of independence
from any single display device in order to provide longevity of the
software and freedom from a single equipment supplier.
The MGED editor supports serial use of multiple displays by way of
an object-oriented programmatic
interface between the editor proper and the display-specific code.
All display-specific code for each type of hardware is isolated
in a separate {\em display manager} module.
High performance of the display manager was an important design goal.
Existing graphics libraries
were considered, but no well established standard existed with the necessary
performance and 3-dimensional constructs.
By having the display manager modules incorporated as a direct part of
the MGED editor, the high rates of display update necessary to deliver
true interactive response are possible, even when using CPUs of modest power.

An arbitrary number of
display managers may be included in a copy of MGED, allowing the user
to rapidly and conveniently move his editing session from display to display.
This is useful for switching between several displays, each of
which may have unique benefits:  one might have color capability,
and another might have depth cueing.
The {\bf release} command closes out MGED's use of the current
display, and does an implicit attach to the ``null'' display manager.
This can be useful to allow another user to briefly examine an image
on the same display hardware without having to lose the state of
the MGED editing session.  The {\bf attach} command is used to
attach to a new display via it's appropriate display manager routines.
If another display is already attached, it is released first.
The null display manager also allows the MGED editor to be run from a normal
alphanumeric terminal with no graphic display at all.  This can be useful
when the only tasks at hand involve viewing or changing
database structures, or entering or adjusting geometry parameters
in numerical form.

Creation of a new display manager module in the ``{\bf C}'' language
\cite{c-prog-lang}
generally takes an experienced
programmer from one to three days.
The uniform interface to the display manager provides two levels
of interactive support.
The first level of display support includes
the Tektronix 4014, 4016, and compatible displays,
including the Teletype 5620 bit-mapped displays.
However, while storage-tube style display devices allow MGED to
deliver the correct functionality, they lack the
rate of screen refresh needed for productive interaction.
The second level of support, including real-time interaction,
is provided by
the Vector General 3300 displays,
the Megatek 7250 and 7255 displays,
the Raster Technologies Model One/180 display,
the Evans and Sutherland PS300 displays
with either serial, parallel, or Ethernet attachment,
the Sun workstations,
and the Silicon Graphics IRIS workstation family.

\section{Portability}

Today, the half-life of computer technology is
approximately two to three years.
To realize proper longevity of the modeling software, it needs to be written
in a portable language to allow the software to be moved readily from
processor to processor without requiring the modeling software or users
to change.
Then, when it is desirable to
take advantage of the constantly increasing
processor capabilities and similarly increasing memory capacity by replacing
the installed hardware base, there are a minimum of ancillary costs.
Also, it may be desirable to connect together processors from a variety
of vendors, with the workload judiciously allocated to
the types of hardware that best support the requirements of each particular
application program.
This distribution of processing when coupled with the fact that
users are spread out over multiple locations makes networking a vital
ingredient as well.

BRL's strategy for achieving this high level of portability was to target
all the software for the UNIX operating system,
\cite{unix-ts-sys},
with all the software written in the ``{\bf C}''
programming language \cite{c-prog-lang}.
The entire BRL-CAD Package, including the MGED editor
is currently running on all UNIX machines at BRL,
under several versions of the UNIX operating system, including
Berkeley 4.3 BSD UNIX, Berkeley 4.2 BSD UNIX, and AT\&T System V UNIX.

The list of manufacturers and models of CPUs that support the UNIX
operating system \cite{modern-tools-hi-res}
is much too lengthy to include here.  However, BRL
has experience using this software on
DEC VAX 11/750, 11/780, 11/785 processors,
Gould PN6000 and PN9000 processors,
Alliant FX/8 and FX/80 processors (including systems with eight CPUs),
Silicon Graphics IRIS 2400, 2400 Turbo, 3030, 4-D, and 4-D/GT workstations,
the Cray X-MP, the Cray-2,
and the ill-fated Denelcor HEP H-1000 parallel supercomputer.

\section{Object-Oriented Design}

The central editor code has four sets of object-oriented interfaces
to various subsystems, including database access, geometry processing,
display management, and command parser/human interface.
In each case, a common interface has been defined for the set of
functions that implement the subsystem;
multiple instances of these function sets can exist.
The routines in each instance of a subsystem are completely independent
of all the routines in other functions sets, making it easy to add new
instances of the subsystem.  A new type of primitive geometry,
a new display manager, a new database interface, or a new command
processor can each be added simply by writing all the routines
to implement a new subsystem.
This approach greatly simplifies software maintenance, and allows
different groups to have responsibility for the
creation and enhancement of features within each of the subsystems.

\chapter{THE COMBINATORIAL GEOMETRY METHODOLOGY}

\section{Background}

Since the MGED system is presently based on the COMGEOM solid modeling
technique, a brief overview of the COMGEOM technique is required
to effectively use MGED.
For more detailed information on the COMGEOM technique see
\cite{gift1,gift2}.

\begin{figure}[tb]
\begin{tabular}{l l}
Symbol & Name \\
\\
ARS	& Arbitrary Triangular Surfaced Polyhedron \\
ARB	& Arbitrary Convex Polyhedron \\
ELLG	& General Ellipsoid \\
POLY	& Polygonal Faceted Solid \\
SPL	& Non-Uniform Rational B-Spline (NURB) \\
TGC	& Truncated General Cone \\
TOR	& Torus \\
HALF	& Half Space (Plane)
\end{tabular}
\caption{Basic Solid Types \label{list-of-basic-solids} }
\end{figure}
\begin{figure}[tb]
\begin{tabular}{l l}
Symbol & Name \\
\\
RPP	& Rectangular Parallelpiped \\
BOX	& Box \\
RAW	& Right Angle Wedge \\
SPH	& Sphere \\
RCC	& Right Circular Cylinder \\
REC	& Right Elliptical Cylinder \\
TRC	& Truncated Right Cylinder \\
TEC	& Truncated Elliptical Cylinder \\
\end{tabular}
\caption{Special-Case Solid Types \label{list-of-special-case-solids} }
\end{figure}
The COMGEOM technique utilizes two basic entities - a solid and a region.
A solid is defined as one of fifteen basic geometric shapes or
primitives.  Figure \ref{list-of-basic-solids} lists the
basic solid types, and Figure \ref{list-of-special-case-solids}
lists special cases of the basic solid types for which support exists.
The individual parameters of each solid define the solid's
location, size, and orientation.  A region is a combination of
one or more solids and is defined as the volume occupied
by the resulting combination of solids.
Solids are combined into regions using any of three logic
operations: union (OR), intersection (+), or difference (-).
The union of two solids is defined as the volume in either
of the solids.
The difference of two solids is defined as the volume of the first
solid minus the volume of the second solid.
The intersection of two solids is defined as the volume
common to both solids.
%%% XXX Figure 1 presents a graphical representation of these operations.

Any number of solids may be combined to produce a region.
As far as the COMGEOM technique is concerned, only a region can
represent an actual component of the model.
Regions are homogeneous;  they are composed of a single material.
Each region represents a single object in the model;
the solids are only building blocks which are combined to
define the {\em shape} of the regions.
Since regions represent the components of the model, they
are further identified by code numbers.
These code numbers either identify the region as
a model component (nonzero item code)
or as air (nonzero air code).
Any volume not defined as a region is assumed to be ``universal air'' and
is given an air code of ``01''.
If it is necessary to distinguish between universal ``01'' air and any
other kind of air, then that volume must be defined as a region
and given an air code other than ``01''.
Normally, regions cannot occupy the same volume (overlap),
but regions identified with
air codes can overlap with any region identified as a component
(i.e. one that has a nonzero item code).
Regions identified with different air codes, however, can not overlap.

\section{Directed Acyclic Graph and Database Details}

One of the critical aspects of a graphics software package
is its internal data structure.
Since geometric models often result
in very large volumes of data being generated,
the importance of the data structure here is emphasized.
Thus it is felt that a brief introduction to the
organization of the MGED database is
important for all users.

The database is stored as a single,
binary, direct-access
UNIX file for efficiency and cohesion,
with fixed length records called database {\em granules}.
Each object occupies one or more granules of storage.
The user sees and manipulates the directed acyclic graphs
like UNIX paths (e.g., car/chassis/door),
but in a global namespace.
There can be many independent or semi-independent
directed acyclic graphs within the same database,
each defining different models.
The figure also makes heavy use of the {\em instancing} capability.
As mentioned earlier, the
{\em leaves} of the graph are the primitive solids.

Commands exist to import sub-trees from other databases and libraries,
and to export sub-trees to other databases.
Also, converters exist to dump databases in printable form for
non-binary interchange.

\section{Model Building Philosophy}

The power of a full directed acyclic graph structure for representing
the organization of the database gives a designer a great deal of
flexibility in structuring a model.
In order to prevent chaos, most designers at BRL choose to
design the overall structure of their model in a top-down manner,
selecting meaningful names for the major structures and sub-structures
within the model.
Actual construction of the details of the
model generally proceeds in a bottom-up
manner, where each sub-system is fabricated from component primitives.

The first sub-systems to be constructed are the chassis and skin of the
vehicle, after which a set of analyses are run to validate the geometry,
checking for unintentional gaps in the skin or for solids which overlap.
The second stage of model construction is to build the features of the
main compartments of the vehicle.  If necessary for the analysis
codes that will be used, the different types of air compartments within
the model also need to be described.
The final stage of model construction is to build the internal
objects to the desired level of detail.
This might include modeling engines, transmissions, radios,
people, seats, etc.
In this stage of modeling, the experienced designer will draw heavily on the
parts-bin of model components and on pieces extracted from earlier
models, modifying those existing structures to meet his particular
requirements.

Throughout the model building process it is important for the model builder
to choose part names carefully, as the MGED database currently has a
global name space, with individual node names limited to 16 characters.
In addition, BRL has defined conventions for naming the elements in the
top three levels of database structure,
allowing people to
easily navigate within models prepared at
different times by different designers.
This naming convention
facilitates the integration of design changes into existing models.

\chapter{THE BASIC EDITING PROCESS}

\section{Interaction Forms}

Textual and numeric interaction with the MGED editor is the most
precise editing paradigm because it allows exact
manipulation of known configurations.
This works well when the user is designing the model
from an existing drawing, or when all dimensions are known (or are computable)
in advance.

The use of a
tablet or mouse, knob-box or dial-box, buttons, and a joystick
are all simultaneously supported by MGED for analog inputs.
Direct graphic interaction via a ``point-push-pull'' editing paradigm
tends to be better for
prototyping, developing arbitrary geometry, and fitting
together poorly specified configurations.
Having both types of interaction capability available at all times
allows the user to select the style of interaction that best
meets his immediate requirements.

\section{The Faceplate}

\mfig faceplate, The MGED Editor Faceplate.
\mfig buttonmenu, The Pop-Up Button Menu.
When the MGED program has a display device attached, it
displays a border around the region of the screen being used
along with some ancillary status information.  Together, this
information is termed the editor ``faceplate''.
See Figure \ref{faceplate}.
In the upper left corner of the display is a small enclosed area
which is used to display the current editor state;
this is discussed further in the Editor States section, below.

Underneath the state display is a zone in which three ``pop-up'' menus
may appear.
The top menu is termed the ``button menu,'' as it
contains menu items which duplicate many of the functions assigned to
the button box.
Having these frequently used
functions available on a pop-up menu
can greatly decrease the number of times that the user needs to remove
his hand from the pointing device (either mouse or tablet puck)
to reach for the buttons.
An example of the faceplate and first level menu is shown in
Figure \ref{buttonmenu}.
The second menu is used primarily for the various editing states,
at which time it contains all the editing operations which are generic
across all objects (scaling, rotation, and translation).
The third menu contains selections for object-specific editing operations.
The choices on these menus are detailed below.

It is important to note that while some display hardware that MGED runs on
has inherent support for pop-up menus included, MGED does not
presently take advantage of that support, preferring to depend
on the portable menu system within MGED instead.
It is not clear whether the slight increase in functionality that might
accrue from using display-specific menu capabilities would offset the
slight nuisance of a non-uniform user interface.

Running across the entire bottom of the faceplate is a thin rectangular
display area which holds two lines of text.
The first line always contains a numeric display of the model-space
coordinates of the center of the view and the current size of
the viewing cube, both in the currently selected editing units.
The first line also contains the current rotation rates.
The second line has several uses, depending on editor mode.
Normally it displays the formal name of the database that is being
edited, but in various editing states this second line will instead
contain certain path selection information.
When the angle/distance cursor function is activated, the second
line will be used to display the current settings of the cursor.

It is important to mention that while the database records all
data in terms of the fixed base unit of millimeters, all numeric interaction between
the user and the editor are in terms of user-selected display [or local] units.
The user may select from millimeters, centimeters, meters, inches, and
feet, and the currently active display units are noted in the first
display line.

The concept of the ``viewing cube'' is an important one.
Objects drawn on the screen are clipped in X, Y, and Z, to the size
indicated on the first status line.
This feature allows extraneous wireframes which are positioned within view
in X and Y, but quite far away in the Z direction to not be seen,
keeping the display free from irrelevant objects when zooming in.
Some display managers can selectively enable and disable Z axis clipping
as a viewing aid.

\section{The Screen Coordinate System}

\mfig coord-axes, The Screen Coordinate System.
The MGED editor uses the standard right-handed
screen coordinate system,
as shown in Figure \ref{coord-axes}.
The Z axis is perpendicular to the screen and the positive Z direction is
out of the screen.  The directions of positive (+) and negative (-) axis
rotations are also indicated.  For these rotations, the ``Right
Hand Rule'' applies:  Point the thumb of the right hand along the direction
of +X axis and the other fingers will describe the sense of positive
rotation.

\section{Changing the View}

At any time in an editing session, the user may add one or more
subtrees to the active model space.  If the viewing cube is
suitably positioned, the newly added subtrees are drawn on the display.
(The ``reset'' function can always be activated to get the entire active
model space into view).
The normal mode of operation is for users to work with wireframe
displays of the unevaluated primitive solids.  These wireframes can be
created from the database very rapidly.

\mfig crod, An Engine Connecting Rod.
\mfig crod-close, {Close-Up Connecting Rod, Showing Z-clipping}.
On demand, the user can request the calculation of
approximate boundary wireframes that account for
all of the boolean operations specified along the arcs of the
directed acyclic graph in the database.
This is a somewhat time consuming process, so it is not used
by default, but it is quite reasonable to use whenever the
design has reached a plateau.
Note that these boundary wireframes are not stored in the database,
and are generally used as a visualization aid for the designer.
Figure \ref{crod} shows an engine connecting rod.
On the left side is the wireframe of the unevaluated primitives
that the part is modeled with, and on the right side is the approximate
boundary wireframe that results from evaluating the boolean expressions.

Also, at any time the user can cause any part of the active model space
to be dropped from view.
This is most useful when joining two complicated subsystems
together;  the first would be called up into the active model space,
manipulated until ready, and then the second subsystem would also be
called up as well.  When any necessary adjustments had been made,
perhaps to eliminate overlaps or to change positioning tolerances,
one of the subassemblies could be dropped from view,
and editing could proceed.

The position, size, and orientation of the viewing cube can be
arbitrarily changed during an editing session.
The simplest way to change the view is by selecting one of nine
built in preset views, which can be accomplished by a simple keyboard
command, or by way of a button press or first level menu selection.
The view can be rotated and translated to any arbitrary position.
The user is given the ability to execute a {\bf save view} button/menu
function that attaches the current view to a {\bf restore view} button/menu
function.

The rate of rotation around each of the X, Y, and Z axes
can be selected by knob, joystick, or keyboard command.
Because the rotation is specified as a rate, the view
will continue to rotate about the view center until the rotation
rate is returned to zero.
(A future version of MGED will permit selection of rate or value
operation of the knobs).
Similarly, the zoom rate (in or out) can be set by keyboard
command or by rotating a control dial.
Also, displays with three or more mouse buttons have binary (2x) zoom
functions assigned to two of the buttons.
Finally, it is possible to set a slew rate to translate the view
center along any axis in the current viewing space, selectable
either by keyboard command or control dial.
In VIEW state, the main mouse button translates the
view center;  the button is defined to cause the indicated point to become
the center of the view.

The assignment of zoom and slew functions to the mouse buttons tends to
make wandering around in a large model very straightforward.
The user uses the binary zoom-out button to get an overall view, then
moves the new area for inspection to the center of the view and uses
the binary zoom-in button to obtain a ``close up'' view.
Figure \ref{crod-close}
shows such a close up view of the engine connecting rod.
Notice how the wireframe is clipped in the Z viewing direction
to fit within the viewing cube.

\section{Model Navigation}

In order to assist the user in creating and manipulating a complicated
hierarchical model structure, there is a whole family of editor commands
for examining and searching the database.
In addition, on all keyboard commands, UNIX-style regular-expression
pattern matching, such as ``*axle*'' or ``wheel[abcd]'', can be used.
The simplest editor command ({\bf t}) prints a table of contents, or directory,
of the node names used in the model.  If no parameters are specified,
all names in the model are printed,
otherwise only those specified are printed.
The names of solids are printed unadorned, while the names of combination
(non-terminal) nodes are printed with a slash (``/'') appended to them.

If the user is interested in obtaining detailed information about the
contents of a node, the list ({\bf l}) command will provide it.
For combination (non-terminal) nodes, the information about all departing
arcs is printed, including the names of the nodes referenced, the boolean
expressions being used, and an indication of any translations and rotations
being applied.
For leaf nodes, the primitive solid-specific ``describe yourself''
function is invoked, which provides a formatted display of the parameters
of that solid.

The {\bf tops} command is used to find the names of all nodes which are
not referenced by any non-terminal nodes;  such nodes are either
unattached leaf nodes, or tree tops.
To help visualize the tree structure of the database,
the {\bf tree} command exists to
print an approximate representation of the database subtree below the
named nodes.
The {\bf find} command can be used to find the names of all non-terminal
nodes which reference the indicated node name(s).  This can be very helpful
when trying to decide how to modify an existing model.
A related command ({\bf paths}) finds the full tree path specifications
which contain a specified graph fragment, such as ``car/axle/wheel''.
In addition to these commands, several more commands exist
to support specialized types of searching through the model database.

\section{Editor States}

The MGED editor operates in one of six states.
Either of the two PICK states can be entered by button press,
menu selection, or keyboard command.  The selection of the desired
object can be made either by using {\em illuminate mode}, or by
keyboard entry of the name of the object.

Illuminate mode is arranged such that if there are {\bf n} objects visible on
the screen, then the screen is divided into {\bf n} horizontal bands.
By moving the cursor (via mouse or tablet) up and down through these bands,
the user will cause each solid in turn to be highlighted on the screen,
with the solid's name displayed in the faceplate.
The center mouse button is pressed when the desired solid is located, causing
a transition to the next state (Object Path, or Solid Edit).

Illuminate mode offers significant advantages over more conventional pointing
methods when the desired object lies in a densely populated region of the
screen.  In such cases, pointing methods have a high chance of making an
incorrect selection.
However, in sparsely populated regions of the screen, a pointing paradigm
would be more convenient, and future versions of MGED will support this.

\section{Model Units}

All databases start with an ``ident'' record which contains
a title string that identifies the model, the
current local units (eg, mm, cm or inches) of the model,
and a database version identification number.
As noted, all numerical information
in the database is stored in the fixed base
unit of millimeters,
and all work (input and output) is done in a user-selected local unit.
The user can change his local unit at any time
by using the {\bf units} command.
This way of handling units was selected to free the user from worrying
about units conversion when components are drawn from the ``parts bin''.
\chapter{PERIPHERAL DEVICES}

Before we discuss the features of MGED, we will introduce
the hardware devices used to implement them.
These devices are the ``tools of the trade'' for the MGED user.
We will discuss only basic operational characteristics here.
Specific use of these devices will be covered in the later sections
on the viewing and editing features of MGED.

\section{Joystick}

The joystick is a mechanical device used to do rotations in MGED.
Any movement left or right rotates the display about the
X-axis.  Any movement up or down rotates the display
about the Y-axis.  When the joystick top is twisted in a clockwise or
counterclockwise direction, the display rotates about the Z-axis.
Any combination motion of the joystick will produce a ``combined''
rotation about the appropriate axes.
As implemented on the Vector General hardware,
all of these motions have a spring return to a null center position.

\section{Button Box}

The button box contains a collection of buttons.
On each button is a light that can be lit under program control.
Pressing a button sends a ``press'' event to MGED,
and results in an action occurring, or a condition being set.
The exact functions assigned to these buttons will be discussed
in the sections on viewing the display and on editing.

\subsection{Vector General Buttons}

\PostScriptPicture 6in by 5.6in, fig-vg-buttons.ps, Vector General Button Assignments, vg-buttons.
\PostScriptPicture 4.5in by 3.25in, fig-sgi-buttons.ps, Silicon Graphics Button Assignments, sgi-buttons.
% \gluein 4.5in by 4.5in, Vector General Button Assignments, vg-buttons.
% XXX \gluein 4.5in by 4in, Megatek Dial and Button Box, mg-buttons.
The Vector General has thirty-two buttons.
Figure \ref{vg-buttons} depicts the functions programmed
for each button.
The buttons in the shaded area are used for editing while the
rest are used for viewing the display.

\subsection{Megatek Buttons}

\begin{figure}[tbp]
\begin{tabular}{rl}
Button  & Function\\
 \\
1       & View Mode:  Restores View \\
        & Edit Mode:  Translation in the Object-Edit mode \\
2       & View Mode:  Saves View \\
        & Edit Mode:  Translation in the Object-Edit mode \\
3 \\
        & Edit Mode:  Saves the model being displayed on the screen \\
4       & Off:        Viewing mode \\
        & On:         Edit mode \\
5       & View Mode:  Resets View \\
        & Edit Mode:  Scaling in the Object-Edit mode \\
6 \\
        & Edit Mode:  Rotation in the Object-Edit mode \\
7       & View Mode:  Angle/Distance Cursor \\
        & Edit Mode:  Translation in the Object-Edit mode \\
8 \\
        & Edit Mode:  Rejects display and returns to Viewing display \\
9       & View Mode:  Bottom View \\
        & Edit Mode:  Scaling in the Solid-Edit mode \\
10      & View Mode:  Left View \\
        & Edit Mode:  Rotation in the Solid-Edit mode \\
11      & View Mode:  Rear View \\
        & Edit Mode:  Translation in the Solid-Edit mode \\
12      & View Mode:  90, 90 View \\
        & Edit Mode:  Restores Edit mode menu \\
13      & View Mode:  Top View \\
        & Edit Mode:  Transfers from Viewing to Solid Pick \\
14      & View Mode:  Right View \\
        & Edit Mode:  Transfers from Viewing to Object Pick \\
15      & View Mode:  Front View \\
 \\
16      & View Mode:  35/45 View
\end{tabular}
\caption{Megatek Buttons \label{mg-button-table} }
\end{figure}

The Megatek button box
is a general purpose input/output device that communicates with
MGED through an intelligent control unit.  The device has eight
rotatable knobs and 16 buttons with lights.
% XXX See Figure \ref{mg-buttons}.
The ``buttons'' and ``knobs'' of the Megateks are located in the same box.
There are not enough buttons to have just one assigned meaning, hence
most buttons have dual functions.
To toggle the functions of the buttons,
use the upper right button (toggle button).
When the light on this button is ON, the functions listed on the RIGHT above
each button is the current function.
When the light on the ``toggle'' button is OFF, the functions labeled on the
LEFT are then in effect.
The left/right meaning of these buttons is grouped generally according to
viewing functions on the left and editing functions on the right.

Figure \ref{mg-button-table}
summarizes the uses of the buttons.
Depressing the button switches the light on and off.  Many of these serve a
dual role depending upon the selected mode - viewing or editing.  The mode is
selected by depressing button 4.  If light 4 is off, the system is performing
in the viewing mode, and the commands shown in the top half of the table are
executed.  If light 4 is on, the system is performing in the edit mode, and
the commands shown in the bottom half are executed.

\subsection{Silicon Graphics Buttons}

The button box layout for the SGI Iris is given
in Figure \ref{sgi-buttons}.
Note that the ``right'' button shows you the right side of the
model, as if you were looking in from the left.
To achieve the customary draftsman views, this function
goes on the left.

The upper left button is the {\bf help} key.
If this button is held down, and any other button (or knob)
is activated, a descriptive string is displayed in the eight character
LED display on the button box.
The upper right button is used to reset all the knobs to zero.
This is useful to halt a runaway rotation or zoom operation.

%\begin{figure}[tbp]
%{\tt \begin{verbatim}
%          |---------|---------|---------|---------|
%          |         |         |         |  Zero   |
%          |  Help   |   ADC   | Reset   |  Knobs  |
%|---------|---------|---------|---------|---------|---------|
%|   Obj   |   Obj   |   Obj   |   Obj   |         |  Save   |
%|  Scale  | ScaleX  | ScaleY  | ScaleZ  | empty   |  View   |
%|---------|---------|---------|---------|---------|---------|
%|   Obj   |   Obj   |   Obj   |   Obj   |         | Restore |
%| TransX  | TransY  | TransXY | Rotate  | empty   |  View   |
%|---------|---------|---------|---------|---------|---------|
%|  Solid  |  Solid  |  Solid  |  Solid  |   Obj   | Solid   |
%|  Trans  |   Rot   |  Scale  |  Menu   |  Pick   | Pick    |
%|---------|---------|---------|---------|---------|---------|
%| REJECT  | Bottom  |  Top    |  Rear   |  90,90  | ACCEPT  |
%|---------|---------|---------|---------|---------|---------|
%          |  Right  |  Front  |  Left   |  35,25  |
%          |---------|---------|---------|---------|
%\end{verbatim} }
%\caption{Silicon Graphics Button Layout \label{XXsgi-buttons} }
%\end{figure}

\section{Knobs (Dials)}

The knobs (or control dials) are used to send digital information
to the computer.
As a knob is turned, a succession of numbers are available for
use by the computer.
The knobs can be used to rotate a displayed object about the x, y, or z
axis, translate the object along the x or y axis, and change the size of the
view.  Action performed by these knobs is continuous and is initiated by
turning the knob in the proper direction and terminated by turning the knob
in the opposite direction.

\subsection{Vector General Knobs}

\PostScriptPicture 4.5in by 2.8in, fig-vg-knobs.ps, Vector General Knob Assignments, vg-knobs.
\PostScriptPicture 4.5in by 3.25in, fig-sgi-knobs.ps, Silicon Graphics Knob Assignments, sgi-knobs.
% \gluein 4.5in by 3.5in, Vector General Knobs, vg-knobs.
% \gluein 4.5in by 3.5in, Silicon Graphics Knobs, sgi-knobs.
Figure \ref{vg-knobs} depicts the functions assigned to each of the ten knobs.
The exact functions of each of these knobs will be discussed in
the angle distance cursor section and in the viewing features section.

\subsection{Megatek Knobs}

The ``buttons'' and ``knobs'' of the Megateks are located in the same box.
% XXX as shown in Figure \ref{mg-buttons}.
There are not enough knobs to have ONE assigned meaning, hence
three knobs have dual functions.
The second function of the first three knobs is only in effect when
the angle-distance cursor (ADC) is on the screen.

\subsection{Silicon Graphics Knobs}

Figure \ref{sgi-knobs} depicts the functions assigned to the
eight knobs on the Silicon Graphics knob box.
In normal operation, the left knobs provide rotations,
and the right knobs provide translations and zooming.
When the angle/distance cursor is activated, some of the
knobs are redefined.

\section{Mouse or Data Tablet}

Moving the mouse or the data tablet ``pen'' causes a cursor
on the screen to move.
The screen X-Y coordinates of the cursor can be sensed by MGED
at any time.
Clicking one of the mouse buttons,
or depressing the tip of the pen, results in MGED receiving
a special event notification.
The meaning of this mouse event depends on the current editing mode
and which portion of the display faceplate that the cursor is located
in.

Below is a list of some of the functions the mouse is used for in MGED;
\begin{itemize}
\item
Selecting editing menus, edit functions (move faces, move edges
etc.) and viewing functions (selected from main edit menu); move
pointer to appropriate edit function and press center mouse button.
\item
Pointing functions; interactively positioning solid primitive
relative to other solids with positioning or size update being
displayed at the same time, position pointer where required and click
center mouse button.
\item
Scaling of view size; enlarge or reduce for a more detailed view of
object, left button shrinks view size, right button enlarges view
size.
\item
During the solid or object illuminate phase of editing,
the screen is divided into
invisible horizontal sections.
The available selections are scanned by moving the mouse up and down.
\item
When MGED is is the viewing state,
and a mouse event is received which is not in the menu area of the faceplate,
the point at which the cursor is pointing at will be translated to become
the center of the current view.
By pointing and clicking the center mouse button,
the center of the viewing cube
can be moved to allow close-up viewing of different areas in your
model.
\end{itemize}

\subsection{Vector General Data Tablet}

Position information is entered using a pen-like stylus.
The distance this pen is from the tablet is important.
If the pen tip is within one half inch of the tablet surface, the cursor
location on the screen corresponds to the X,Y location of the
pen on the tablet.  This condition is called the ``near'' position.
If the pen is more than one half inch from the tablet surface, the
cursor remains located in the center of the screen.
When the pen is pressed against the tablet surface,
the pressure switch is activated and a ``mouse'' event
is sent to MGED.

\subsection{Megatek Data Tablet}

Some Megatek systems enter position data on the data tablet
using a pen-like stylus.
If the tip of the stylus is
within one-half inch of the surface of the tablet, a ``star'' corresponding
to this location is displayed on the display screen.  If the tip is moved
more than one-half inch from the surface, the position of the star remains
fixed.  When the stylus is pressed against the tablet surface, the pressure
switch is activated and a ``mouse'' event is sent MGED.

Other Megatek data tablets have a mouse instead of a pen.
This mouse has four buttons on it.
The yellow (top) button is used during illumination and editing just as the
pen on the Vector General terminals.
However, in the viewing mode, when pushed, the point which it was pointing
at will be drawn at the center of the screen.
The blue (bottom) button has this same function at ALL times and is used
to ``slew'' the display during editing.
The white (left) and the green (right) buttons on the mouse are used for
zooming the display at a fixed rate.
The white button will zoom out and the green button will zoom in.

\subsection{Silicon Graphics Mouse}

The left and right mouse buttons are used for binary (2x) zooming,
and the center mouse button is used for all other MGED mouse functions.

On the Silicon Graphics 3-D workstations, MGED can be run directly,
or it can be run under the window manager MEX.  In both cases,
MGED opens two windows, one outlined in white for all text interaction,
and one outlined in yellow for all graphics display.
When running MGED directly (without MEX), all mouse events are
sent the MGED, regardless of where the mouse is pointing.
In order to shift emphasis between the graphics and text windows,
the smaller one can be enlarged by pointing the cursor within the
boundaries of the smaller window, and pressing the center button.
This enlarges that window, and reduces the size of the other window.

When MEX is running, it is necessary to follow the MEX convention of
moving the cursor into the desired window, and clicking the right mouse
button, to ``attach'' all input to that window.
This has the unfortunate consequence of requiring a lot of extra
mouse clicking, because the graphics window needs to be attached
when using the buttons, knobs, and mouse, while the text window
needs to be attached in order to enter keyboard commands.

On the Silicon Graphics 4-D workstations running 4Sight,
mouse events are sent to MGED only when the cursor is within the
boundaries of the MGED graphics window.

\subsection{Sun Workstation Mouse}

On the Sun workstation, MGED must be run in a {\bf suntools} window.
The main consequence of this is that mouse events are sent to MGED
only when the cursor is within the boundaries of the MGED graphics window
on the screen.
The left and right mouse buttons are used for binary (2x) zooming,
and the center mouse button is used for all other MGED mouse functions.

\section{Keyboard}

The keyboard is used to issue commands and supply parameters to MGED.
It is also used to login and logout of the UNIX system, and to
run other UNIX programs.
All characters typed on the keyboard,
with the exception of the user's password, are displayed (echoed) on the
monitor.
In this text, all input typed by the user is
shown in {\em italics}, while all literal MGED output
is shown in {\tt typewriter font}.
All entries are
terminated by depressing the RETURN key.
This action immediately precedes the execution of the directive.
In most cases, lower case letters
must be used.  A space must be used between the command and its
arguments.
Embedded blanks are not allowed.
Entering Control/H causes cursor to backspace and erase entered
information.
An MGED command is interrupted by entering Control/C.
End-of-File is sent to MGED by entering Control/D.
The graphics editor displays the prompt
\begin{verbatim}
	mged>
\end{verbatim}
on the display whenever it is ready to accept a command from the keyboard.

\chapter{OPERATING INSTRUCTIONS}

\section{Entering the Graphics Editor}

Type {\em mged filename}, e.g.:
\begin{verbatim}
     mged s_axle.g
     mged shaft.g
     mged fred.g
\end{verbatim}
where the filename is the name of the UNIX file in which
your object description data is stored.
It is conventional that the
extension ``.g'' on the filename
signifies a graphics file, and is a good practice, but is not required.
If the named database does not already exist,
MGED will ask if it should create a new database.
MGED will ask:

{\tt \begin{verse}
\% {\em mged new.g} \\
BRL-CAD Release 3.0 Graphics Editor (MGED) \\
\ \ \ \ Tue Sep 6 02:52:55 EDT 1988 \\
\ \ \ \ mike@video:/cad/mged.4d \\
new.g: No such file or directory \\
Create new database (y|n)[n]? {\em y} \\
attach (nu|tek|tek4109|ps|plot|sgi)[nu]? {\em sgi} \\
ATTACHING sgi (SGI 4d) \\
Untitled MGED Database (units=mm) \\
mged>
\end{verse} }
Here, the {\em italic} type indicates the user's response:
{\em y} instructs MGED to create the new database, and
{\em sgi} instructs MGED to attach to a window
on the Silicon Graphics (SGI) workstation.

Directives to the graphics editor are made by
\begin{enumerate}
\item entering information from the keyboard,
shown here in the text by the use of {\em italics},
\item using the stylus to select items from
the menu (select), and
\item pressing buttons and twisting knobs on the
function control box (press, twist).
\end{enumerate}

The prompt for a command is {\tt mged>}.

\subsection{Running MGED on a Silicon Graphics}

When running MGED from the console of a Silicon Graphics workstation,
MGED retains the text window from which it was started,
and opens a second window for the graphics display.
By default, the graphics window is quite large, and the text window
is rather small.

On the SGI 3-d workstations,
should you wish to have a large text window to scan printed output, move
the mouse pointer into the text window and click the center mouse button.
Use the reverse procedure to regain a large graphics window, i.e.,
move the mouse pointer into the graphics window
and click the center mouse button.

\subsection{Running MGED on a Tektronix}

To run MGED on the tek4014 class of terminals one needs to have TWO
terminals - the graphics terminal (4014 or one which emulates a 4014) and
another terminal to enter commands on.

The procedure is as follows:
\begin{enumerate}
\item login on the graphics terminal
\item enter {\em tty} to find out which terminal number has been assigned
\item  enter {\em sleep 32000} to put the graphics terminal in sleep mode
\item  login on the other terminal
\item  enter {\em mged file} to execute MGED
\item  enter {\em tek} to select the tek4014 device processor
\item  enter the tty value found in step 2.
\item  perform editing
\item  enter {\em q} to quit MGED
\item  enter control-c on the graphics terminal to end the sleep mode
\item  logout on both terminals
\end{enumerate}

Since there are no knobs or buttons on the tek4014 class of terminals, one
is forced to use the {\em press} and {\em knob} commands to emulate these
peripherals.
Other commands which can/should be used are:

\begin{tabular}{rl}
ill     & put up a desired path \\
center  & slew the display \\
size    & zoom the display \\
sed     & solid edit immediately
\end{tabular}

The main force behind the writing of a driver for the tek4014 terminals
was to allow the use of the Teletype 5620 terminals.
These graphic terminals have an internal processor and different windows
can be set up which represent different terminals.
Hence two terminals are NOT necessary.
The use of the Teletype 5620 terminals is then the same as the procedure
outlined above, except each window represents a terminal.

\section{The Pop-Up Button Menu}

The default MGED faceplate is shown in Figure \ref{faceplate}.
If the BUTTON MENU area on the screen is selected with the mouse,
then the pop-up button menu appears, as shown in Figure \ref{buttonmenu}.
This menu can be very useful in reducing the amount of hand motion
between the mouse and the button box.

\section{Starting Your Model}

Modeling practices using MGED can be quite individual.  The following is a
suggested modeling method to start with; you may end up developing your own
style as you become more familiar with MGED.

First of all, decide how you want to represent your model, including the
amount of detail, types of solids and regions necessary.  Have an accurate
sketch or engineering drawing available, so that you can easily tranfer its
information into the types of primitive solids necessary to create your model.
Where possible it is recommended to start with a large block solid and
``subtract'' pieces from it.  In this way you avoid errors with abutting
faces of a collection of solids ``unioned'' together.

Next the solids are created using the
{\em make}, {\em cp}, {\em mirror} or {\em in}
commands.  Depending on the complexity of the model, the solids may be
created in the desired location or created at the origin and later
translated to the desired location.  Creation at the origin provides
an opportunity to take advantage of possible symmetries in the geometry.
Once all the solids are finished it is time to create the region[s],
which will describe (to MGED) how to combine the solids to represent
the model.

The region[s] are then given the desired item/air code (if this is
necessary, otherwise leave it as the system default value), and material
codes.  The regions are then put onto a group, usually for functionality only.
A group has no operations as such (like union [u], intersection [+] or
difference [-]) and is just a collection of objects for convenient naming
of a whole screen or collection of objects.
\chapter{CREATING NEW OBJECTS}

\section{Creating New Leaves (Solids/Primitives)}

A family of commands exists to allow the user to
add more actual solids (leaf nodes) to the model database.
To obtain a precise duplicate of an existing solid (presumably to be
changed by a subsequent editing command), the copy ({\em cp}) command
can be used.  It is important to note that the copy operation is
different from creating an {\em instance} of an existing solid;
there are occasions to use both operations.
If the precise configuration of the solid desired is not important, the
{\em make} command can be used to create a stock prototype solid of the
desired type with the given name, which can then be edited to suit.
The {\em mirror} command makes a
duplicate of an existing solid reflected about
one of the coordinate axes.

If the actual numeric parameters of a solid are known, then the {\em in}
command can be used.  In addition to prompting for the descriptions of
the full generic primitive solids, this command also accepts
abbreviated input formats.  For example, a wedge or an RPP can be entered
with a minimum of parameters, even though a database ARB8 is created.
Similarly, the parameters for a right circular cylinder can be given,
resulting in a truncated general cone (TGC) being stored.
This is not a very sophisticated way to build solids, but it receives
a surprising amount of use.

A number of commands also exist to create new solids with some
higher level description.  For example, the {\em inside} command
creates a new solid inside an existing solid, separated from the
existing solid by specified tolerances.  This is quite useful for
creating hollow objects such as fuel tanks.
It is possible to create a plate with a specified
azimuthal orientation and fallback angle, or to create an ARB8 (plate)
by specifying three points and a thickness, or to create an ARB8
given one point, an azimuthal orientation, and a fallback angle.

\section{Specific Cases}

After having started MGED and created a new database, the next
step is to use the {\em units} command to tell the system that you will
be entering values in mm, cm, m, in or ft.  For our example we will
work in mm:

\begin{verbatim}
     units mm
\end{verbatim}

Now you may give your database a title using the {\em title} command as in:

\begin{verbatim}
     title Mechanical Bracket
\end{verbatim}

This title (``Mechanical Bracket'') now appears at bottom left hand
corner of graphics window.  Further examples:
\begin{verbatim}
     title six wheeled tank
     title stub axle
\end{verbatim}

At this point you can start creating your solid objects using the ``arbs'',
``sph'', ``tor'', {\em etc},
arguments to the {\em make} or {\em in} command.
The {\em make} command gives you a solid to a default size,
you then have to use the
solid edit mode to interactively edit the solid to the desired size.
{\em make} command is entered as below:
Examples:
\begin{verbatim}
make box arb8
make cyl rcc
make ball sph
\end{verbatim}
The first argument is the solid name, and the second argument is
the primitive type.

The {\em in} command expects you to key in a set of parameters to describe your
solid; the parameters can be the x y and z of a vertex (such as the
corner of an ARP8), or the x y and z of a vector (such as the height
or H vector of a BOX) or the radius (such as for a torus).
Below is a list of primitives
with their {\em in} commands as requested by MGED and sample input.
Reading an {\em in} file into a MGED data file will be discussed later.
Note how providing incomplete input to the {\em in} command will result
in MGED repeating the prompt for the missing information.

\mfig ex.rpp, Example RPP.
\subsection{RPP (rectangular parallelepiped)}

{\tt
mged> {\em in name rpp}\\
Enter XMIN, XMAX, YMIN, YMAX, ZMIN, ZMAX: {\em 0 25} \\
Enter YMIN, YMAX, ZMIN, ZMAX: {\em 0 50} \\
Enter ZMIN, ZMAX: {\em 0 100} \\
}

This sequence produces the RPP shown in Figure \ref{ex.rpp}.

\mfig ex.box, Example BOX.
\subsection{BOX (BOX)}

{\tt
mged> {\em in my box} \\
Enter X, Y, Z of vertex: {\em 0 0 0} \\
Enter X, Y, Z of vector H: {\em 25 0 0} \\
Enter X, Y, Z of vector W: {\em 0 50 0} \\
Enter X, Y, Z of vector D: {\em 0 0 100} \\
}
This sequence produces the BOX shown in Figure \ref{ex.box}.

\mfig ex.arb8, Example ARB8.
\subsection{ARB8: Arbitrary Convex Polyhedron, 8 Vertices}

{\tt
mged> {\em in poly arb8} \\
Enter X, Y, Z for point 1: {\em 0 0 0} \\
Enter X, Y, Z for point 2: {\em 0 150 0} \\
Enter X, Y, Z for point 3: {\em 0 150 200} \\
Enter X, Y, Z for point 4: {\em 0 0 200} \\
Enter X, Y, Z for point 5: {\em 75 0 0} \\
Enter X, Y, Z for point 6: {\em 75 150 0} \\
Enter X, Y, Z for point 7: {\em 75 150 200} \\
Enter X, Y, Z for point 8: {\em 75 0 200} \\
}
This sequence produces the ARB8 shown in Figure \ref{ex.arb8}.

\mfig ex.arb4, Example ARB4.
\subsection{ARB4: Arbitrary Convex Polyhedron, 4 vertices}

{\tt
mged> {\em in a4 arb4} \\
Enter X, Y, Z for point 1: {\em 0 0 0} \\
Enter X, Y, Z for point 2: {\em 10 60 0} \\
Enter X, Y, Z for point 3: {\em 40 20 0} \\
Enter X, Y, Z for point 4: {\em 20 15 70} \\
}
This sequence produces the ARB4 shown in Figure \ref{ex.arb4}.

\mfig ex.rcc, Example Right Circular Cylinder.
\subsection{RCC (Right Circular Cylinder)}

{\tt
mged> {\em in rcyl rcc} \\
Enter X, Y, Z of vertex: {\em 0 0 0} \\
Enter X, Y, Z of height (H) vector: {\em 0 0 60} \\
Enter radius: {\em 15} \\
}
This sequence produces the RCC shown in Figure \ref{ex.rcc}.
Note that in this case, the A,B,C, and D vectors have magnitude
which equal the radius, 15.

\mfig ex.trc, Example Truncated Right Cylinder.
\subsection{TRC (Truncated Right Cylinder)}

{\tt
mged> {\em in trcyl trc} \\
Enter X, Y, Z of vertex: {\em 0 0 0} \\
Enter X, Y, Z of height (H) vector: {\em 40 0 0} \\
Enter radius of base: {\em 20} \\
Enter radius of top: {\em 10} \\
}
This sequence produces the TRC shown in Figure \ref{ex.trc}.
Note that the magnitude of A and B equal the base radius, 20,

\mfig ex.raw, Example Right Angle Wedge.
\subsection{RAW (Right Angle Wedge)}

{\tt
mged> {\em in weg raw} \\
Enter X, Y, Z of vertex: {\em 0 0 0} \\
Enter X, Y, Z of vector H: {\em 40 0 0} \\
Enter X, Y, Z of vector W: {\em 0 70 0} \\
Enter X, Y, Z of vector D: {\em 0 0 100} \\
}
This sequence produces the RAW shown in Figure \ref{ex.raw}.

\mfig ex.sph, Example Sphere.
\subsection{SPH (Sphere)}

{\tt
mged> {\em in ball sph} \\
Enter X, Y, Z of vertex: {\em 0 0 0} \\
Enter radius: {\em 50} \\
}
This sequence produces the sphere shown in Figure \ref{ex.sph}.
Note that the A, B, and C vectors all have magnitude equal to
the radius, 50.

\mfig ex.ellg, Example General Ellipsoid.
\subsection{ELLG (General Ellipsoid)}

{\tt
mged> {\em in egg ellg} \\
Enter X, Y, Z of vertex: {\em 0 0 0} \\
Enter X, Y, Z of vector A: {\em 20 0 0} \\
Enter X, Y, Z of vector B: {\em 0 60 0} \\
Enter X, Y, Z of vector C: {\em 0 0 40} \\
}
This sequence produces the ellipsoid shown in Figure \ref{ex.ellg}.

\mfig ex.tor, Example Torus.
\subsection{TOR (Torus)}

{\tt
mged> {\em in tube tor} \\
Enter X, Y, Z of vertex: {\em 0 0 0} \\
Enter X, Y, Z of normal vector: {\em 0 0 50} \\
Enter radius 1: {\em 20} \\
Enter radius 2: {\em 10} \\
}
This sequence produces the torus shown in Figure \ref{ex.tor}.

\section{Creating New Combinations}

Non-terminal nodes in the directed acyclic graph stored in the database
are also called {\em combinations}.
It is possible to extend the definition of a non-terminal node by
adding an instance of an existing node to the non-terminal node
with an associated boolean
operation of union;  this is done by the {\em i}
(instance) command.  To start with, such an instance has an identity
matrix stored in the arc;  the user needs to separately edit the
arc to move the instance to some other location.
If the non-terminal node being extended does not exist, it is created first.

The instance command provides the simplest way to create a reference to
another node.  Instances of a whole list of nodes can be added to a
non-terminal node by way of the group {\em g} command.
If instances of a list of nodes with non-union boolean operations
is to be added to a non-terminal node, the region {\em r} command
accepts a list of (operation, name) pairs, where the single lower case
character ``u'' indicates union, ``--'' indicates subtraction, and
``+'' indicates intersection.  The first operation specified
is not significant.  An example of this command might be:
\begin{center}
{\em r non-terminal u node1 -- node2 + node3}
\end{center}
For historical reasons,
there is no explicit grouping possible, occasionally forcing
the user to create intermediate non-terminal nodes to allow the
realization of the desired boolean formula.
It is also important to note that for the same reasons
there is an {\em implicit} grouping between union terms, i.e.
\begin{center}
u n1 -- n2 + n3 u n4 -- n5
\end{center}
is evaluated as
\begin{center}
(n1 -- n2 + n3) union (n4 -- n5)
\end{center}
rather than
\begin{center}
((((n1 -- n2) + n3) union n4) -- n5)
\end{center}
Therefore, you can think of the solids
on either side of the union operators as surrounded by parentheses.
The order of the region members is
critical, and must be scrutinized when members are added or deleted from a
region.
The order can be printed out using the {\em l} or {\em cat} commands.
\chapter{VIEWING FUNCTIONS}

The MGED viewing features are designed to allow one to examine an object
in close detail.  Any of the viewing features can be invoked at any time.
It should be noted, that these functions do not change the actual data,
only the way the data is displayed.

\section{Preset Views}

Six standard views (front, rear, top, bottom, left, and right) and one
oblique view (azimuth 35, elevation 25 isometric) are each assigned to the
function buttons, views 35 25 (isometric), top, right, front,
45 45 are available from the screen editor menu.  Hence, any of these views
is immediately available at the press of the appropriate function button or
mouse selection.  The views available are not limited to these standard views
however, as the display can be rotated to any view by using the dial box.  By
pressing the function button labeled ``save view'' or entering
the keyboard {\em saveview} command, the present view as displayed display is
saved (used for raytracing,
producing colored pictures, which will be discussed later).
At any time, the saved view can be immediately returned to the
screen by pressing the ``restore view'' function button.  The
``restore view'' button will be lit whenever a view has been saved.
The function button labeled ``reset'',
restores the display to the default view (front) when pressed.

\section{View Translation}

The displays can be panned or slewed on the screen in two ways -- using the
mouse pointer or by using the dial box knobs.  When one is editing, the mouse
functions are not available for slewing, hence one must use the dial box knobs
to slew the display.

To slew the display using the control knobs, one uses the knobs labeled
``slew x'' or ``slew y''.  The null positions on these knobs is in the
center or straight up.  If the ``slew x'' knob is turned clockwise of center,
the display will move to the right.  If it turned counterclockwise, the
display will move to the left.  For the ``slew y'' central knob, clockwise
of the center moves the display up and counterclockwise moves the display
down.  The further these knobs are turned from center, the faster the display
moves.

\section{View Zooming}

One can zoom the display by using the dial box knob labeled ``zoom''.
Again the null position of this knob is center or straight
up.  Turning this knob clockwise of center causes the display to increase in
size producing a zoom-in effect.  Turning this knob counterclockwise of center
causes the display to decrease in size or zoom-out.  Again, the further the
``zoom'' knob is turned from center, the faster the zooming will occur.

\mfig adc, The Angle Distance Cursor.
\section{The Angle Distance Cursor (ADC)}

The angle distance cursor is a construction aid used to measure
angles and distances. It should be noted that all measurements are
made in the projected space of the screen, so one should measure only
in a view normal to the surface where the measurement is to take place.
The ADC is placed on (or removed from) the display by pushing the ``ADC''
button.
The ADC consists of three cursors which cover the entire screen.
Figure \ref{adc} depicts the ADC as it appears on the screen.
All the cursors are centered at the same point and can be moved to any
location on the screen.  Two of these cursors rotate for angle measuring
purposes. Angle cursor 1 is solid while angle cursor 2 is dashed.  Angle
cursor 1 has movable tic marks for measuring distances on the screen.
The two angle cursors move with the horizontal and vertical
lines of the main cursor.
The resulting effect is the moving of the center point
horizontally or vertically.
The ADC is controlled by the bottom row
of the (Megatek) knobs:

\begin{tabular}{rl}
Knob  & Function \\
 \\
6     & moves the center in the horizontal direction \\
7     & moves the center in the vertical direction \\
8     & rotates angle cursor 1  (alpha) \\
9     & rotates angle cursor 2  (beta) \\
10    & moves the tic marks
\end{tabular}

Whenever the ADC is on the screen, there is a readout at the bottom of
the screen listing pertinent information about the ADC.
This information includes the angles that angle cursors 1 and 2
have been rotated (alpha and beta), the distance the tic marks are
from the center of the ADC, and the location of the center of the ADC.
This information is continually updated on the screen.

\chapter{MGED EDITING FEATURES}

The heart of the MGED system is its editing features.
The editing features are divided into two classes: object editing
and solid editing.
Object editing is designed to allow one to change the location,
size, and orientation of an object.
Recall that an object is defined as the basic data unit of the
MGED system and includes both combinations and solids.
In the case of a solid, one needs to change not only its location,
size, and orientation, but also its ``shape''.
Changing the shape of a solid means changing any of its individual parameters.
Hence, solid editing is handled separately.

\section{Combination Editing (OBJECT EDIT)}

Before being able to enter the OBJECT EDIT state
(i.e. edit non-terminal),
it is necessary to pass through two intermediate states
in which the full path of an object to be edited is specified,
and the location of one arc along that path is designated for editing.
It is possible to create a transformation matrix to be applied
above the root of the tree, affecting everything in the path,
or to apply the matrix between any pair of nodes.
For example, if the full path /car/chassis/door is specified,
the matrix could be applied above the node ``car'', between
``car/chassis'', or between ``chassis/door''.

The transformation matrix to be applied at the
designated location can be created by the concatenation of
operations, each specified through several types of user direction.
Trees can be rotated around the center of the viewing cube;
this rotation can be specified in degrees via keyboard command, or can
be controlled by the rotation of a set of control dials or motions on
a three-axis joystick.
Translation of trees can be specified in terms of a precise new location
via keyboard command, or by adjusting a set of control dials.
Tree translation can also be accomplished by pointing and
clicking with the mouse or tablet.
Uniform and single-axis (affine) scaling of a tree can be controlled
by a numeric scale factor via keyboard command, or by way of repeated
analog scaling by pointing and clicking with the mouse or tablet.
Before we discuss the editing features of MGED, we will discuss
how one selects objects for editing.

\section{Selecting Objects For Editing}

To select a displayed object for editing, press the object illuminate button
or select ``Object illum'' from the ***BUTTON MENU***.
The object selection is a two step process.

Whenever an object is displayed (using the {\em e} command), all paths in the
object's hierarchy are traversed recursively, accumulating the transformation
matrices.  When the bottom of the path (a solid) is encountered, the
accumulated
transformations are applied to the solid's parameters and the solid is drawn.
Thus every solid displayed is really a path ending with that solid.
If the object has been displayed using the {\em E} command, the same procedure
is followed, but only until a region is encountered.
Then all members of the region have the accumulated transformations
applied and the region is then ``evaluated'' and drawn.

In the first step of the object selection process, the path is selected.
Again, the data tablet is divided in as many horizontal sections as there are
paths drawn.
The path (solid or evaluated region) corresponding to the horizontal
section the pen/mouse is located in will be illuminated (brighter on B/W
displays and white on color displays).
This complete path is also listed on the display.
When the pen/mouse is pressed the illuminated path is selected.

In the second step, a member of the selected path is chosen.
All editing will then be applied to this member.
The tablet is divided into as many horizontal sections as there are
path members.
The word ``[MATRIX]'' is used to illuminate path members and will appear
above the member corresponding to the location of the pen/mouse.
Pressing the pen/mouse when the desired path member is ``illuminated''
will put MGED in the object edit state.
The editing will be performed on the path member selected.

If a solid is located at the bottom of this path, it becomes the key
solid and its vertex becomes the key point.
If an evaluated region is at the bottom of the path, the center of
this region becomes the key point.
All object editing is done with respect to this key point.

The object editing features can be invoked in any order and at any time
once an object has been selected for editing.  During object editing, any of
the viewing features, such as changing views, zooming, and slewing, can be
used, and in fact, are usually quite useful.  Again, the only way to exit the
object editing mode is to ``accept'' or ``reject'' the editing.
If the ``reject'' button is pressed (or selected from the edit menu), the
object will return to its pre-edit state.  If the ``accept'' button is pressed
(or selected from the edit menu), the data base will be changed to reflect the
object editing performed.

\section{Object Edit State}

When MGED enters the object edit state, the following occurs:

\begin{enumerate}
\item all the solids/evaluated regions of the edited object
become illuminated
\item the key solid's parameters are labeled  OR \\
the center of the key evaluated region is marked
\item the key solid's parameters are listed and
continually updated  OR \\
the key evaluated region's center is listed and
continually updated
\item the ***OBJ EDIT*** menu is displayed
\end{enumerate}

\section{Translate An Object}

There are three ways to translate an object:  translate in the screen
X direction only (X move), translate in the screen Y direction only
(Y move) or just straight translation (XY move).
In all cases, the complete object is translated so that the ``key point''
is positioned at the desired location.
The {\em translate} command is used to enter a precise location (x,y,z) for
the key point.
Entering {\em translate x y z} will move the complete object so that the key
point will be at coordinates (x,y,z).

\section{Rotate An Object}

Rotation of the object may be accomplished by selecting the ``Rotate''
menu item, or pressing the ``Rotate'' button.
Turning the knobs results in the object being rotated.
The {\em rotobj x y z} command can be used here, to specify
a precise rotation in degrees.
While in this edit state, the only way to rotate view is to use
the {\em vrot x y z} command.

\section{Scale An Object}
\subsection{Global Scale}

To select global object scale, press the object scale button or select
``Scale'' from the ***OBJ EDIT*** menu.
When the pen/mouse is pressed, the edited object is scaled about the key
point by an amount
proportional to the distance the pen/mouse is from the center of the screen.
If the pen/mouse is above the center, the edited object will become larger.
If it is below the center, the object will become smaller.
The {\em scale} command can be used to enter precise scale factors.
The value entered is applied to the object as it existed when object
scale was entered.
Hence entering {\em scale 1} will return the object to its size when the
object scale session first started.

\subsection{Local Scale}

Local object scaling is allowed about any of the coordinate axes.
To select local scaling, press one of the buttons (OBJ Scale X, OBJ Scale Y,
or OBJ Scale Z) or select ``Scale X'', ``Scale Y'', or ``Scale Z'' from
the ***OBJ EDIT*** menu.
When the pen/mouse is pressed, the edited object is scaled in the selected
coordinate axis only, about the key point.
The amount of scaling is proportional to the distance the pen/mouse is
from the center of the screen.
If the pen/mouse is above the center, the edited object will become larger
in the selected axis direction.
If it is below the center, the object will become smaller in the selected
axis direction.
The {\em scale} command can be used to enter precise scale factors.
The value entered is applied to the object as it existed when local object
scale was entered.
Hence entering {\em scale 1} will return the object (in the selected axis
direction) to its size when the object scale session first started.

\section{Solid Editing}

There are two classes of editing operations that can be performed on
leaf nodes, the primitive solids.
The first class of operations are generic operations which can be applied to
any type of solid, and the second class of operations are those operations
which are specific to a particular type of solid.
Generic operations which can be applied to all primitive solids are
rotation, translation and scaling.
Recall that primitives can be treated as any other object and ``object
edited'' as detailed above.
Each primitive solid also has a variety of editing operations available that
are specific to the definition of that solid.  These operations are
detailed below.

The solid editing mode is necessary to
perform to basic shapes of solids.
Precise modifications
of the shape are possible (using the {\em p} keyboard command) in the solid
editing mode.

The solid editing feature allows the user to interactively translate,
rotate, scale, and modify individual parameters of a solid.  Whenever one is
in the solid edit mode, the parameters of the solid being edited are listed
and continually updated at the top of the screen.  Certain parameters are
also labeled on the solid being edited.  Solid editing  is generally used to
``build'' objects by producing solids of the desired shape and size in the
correct orientation and position.  Once the object is built, object editing
is used to scale, orient, and position the object in the description.  The
general philosophy of solid editing is to first create a solid with the
desired name and then to edit this solid.  As an example, suppose one were
to build an object called ``BRACKET''; to produce the base of the object the
primitive solid type ARB8 (see Figure 1) would be used along with either the
{\em in} command or {\em make} command, so one would type:
\begin{verse}
     in btm box 0 0 0  0 -90 0  40 0 0  0 0 6 \\
     make block arb8
\end{verse}
A new solid called ``btm'' or ``block'' would be created and displayed on the
screen.  These solids would then be edited using solid editing to produce the
solid parameters for the shape desired.

\section{Selecting Solids For Editing}

The procedure for solid editing is quite similar to that for object editing.
First, solid edit state must be entered, by pressing the
``solid illuminate'' button, or selecting the ``solid illum'' menu item.
Second,
A solid is selected for
editing using the illuminate mode, just as in object editing,
by moving the cursor up and down, and choosing the desired solid.
The solid data is listed at the top of the screen and a
header depending on the solid type is written above the solid editing data.
Third, select the appropriate function button or edit menu operations,
and perform the editing desired.  Finally, the solid
editing mode is exited
by either accepting or rejecting the editing performed.

A solid must be displayed before it can be picked for editing.
To pick a displayed solid for editing, press the ``solid illum'' button or
select ``Solid Illum'' from the ***BUTTON MENU***.
The data tablet and pen/mouse are then used to pick the solid.
The surface of the data tablet is divided into as many horizontal sections
as there are solids displayed.
The displayed solid corresponding to the horizontal section the pen/mouse
is located in will be ``illuminated'' (it will become brighter on black and
white devices and white on color devices).
The complete hierarchical path to reach the solid is also listed on the
display.  When the pen/mouse is pressed, MGED enters the solid edit state
with the illuminated solid as the solid to be edited.
If the solid is not multiply referenced, entering {\em sed solidname} on
the keyboard will immediately put MGED in the solid edit state with
{\em solidname} as the edited solid.

\section{Solid Edit State}

When MGED enters the solid edit state, the following occurs:

\begin{enumerate}
\item the edited solid remains illuminated
\item the edited solid's parameters are labeled
\item the edited solid's parameters are listed
\item  (and continually updated)
\item the ***SOLID EDIT*** menu is displayed
\item the parameter edit menu is initially displayed (default)
\end{enumerate}

\section{Rotate A Solid}

Solid rotation allows the user to rotate the solid being edited to any
desired orientation.  The rotation is performed about the vertex of the
solid.  To select this option, one presses the function button labeled
``solid rotate'' or selects from the edit menu on screen.
The rotation can be done using the dial box or one can input exact angles
of rotation of the solid by using the {\em p} keyboard command.
For example, typing:
{\em \center
p alpha beta gamma
}
will rotate the solid {\em alpha} degrees about the x-axis, {\em beta} degrees
about the y-axis and {\em gamma} degrees about the z-axis.  Alpha, beta, and
gamma are measured from the original ``zero'' orientation of the solid,
defined when the ``solid edit'' function button was
pressed.  Hence, typing
{\em \center
p 0 0 0
}
will always return the solid to its original position (its position when the
current solid editing session began) before accepting edit.

To select solid rotation, press the solid rotate button or select ``Rotate''
from the ***SOLID EDIT*** menu.
The joy stick or appropriate rotation knobs then will rotate the edited solid
about the coordinate axes.
The solid is rotated about its vertex.
The parameter (p) command can be used to make precise rotation changes.
The values entered after the p are absolute -- the rotations are applied
to the solid as it existed when solid rotation was first selected.
Thus entering {\em p 0 0 0} will ``undo'' any rotations performed since
solid rotation was selected.
The rotation about the z-axis is done first, then the y, then the x.

\section{Translate A Solid}

Solid translation allows the user to place the solid being edited anywhere
in the description.  To invoke this option, one presses the function
button labeled ``solid trans'' or selects from the screen edit
menu.  To move the solid, use the mouse pointer to position the solid and
click the center mouse button.  Whenever the mouse button is pressed, the
VERTEX of the solid moves to that location on the screen.

One can read the actual coordinates of the vertex on the top of the
screen, along with other data.  If the actual desired coordinates of the
vertex are known, one can place the solid exactly using the {\em p} keyboard
command.  For example, to place a solid's vertex at the coordinates (x,y,z)
one would type:
{\em \center
p40 20 10
}
The solid would then jump to this location.

To select solid translation, press the solid translate button or
select ``Translate'' from the ***SOLID EDIT*** menu.
When the pen/mouse is pressed, the vertex of the edited solid will
move to that location.
The parameter (p) command can be used to translate the solid to
a precise location.
Entering {\em p x y z} will place the vertex of the edited solid at (x, y, z).

\section{Scale A Solid}

The solid SCALE feature allows the user to scale the solid being to any
desirable size.  The scaling is done about the vertex of the solid, hence NO
translation of the solid occurs.  The scaling is performed using the mouse
pointer and clicking the center mouse button, just as in object scaling.  One
can input an exact scale factor using the {\em p} keyboard command, in the form
of.  For example, typing
{\em \center
p factor
}
will scale the solid by an amount equal to {\em factor}.   The value of
{\em factor} is absolute -- the original solid is scaled.  By setting {\em factor}
equal to one (1), the original size solid will be displayed on the screen
before accepting your edit.

To select solid scale, press the solid scale button or select ``Scale''
from the ***SOLID EDIT*** menu.
When the pen/mouse is pressed, the edited solid is scaled by an amount
proportional to the distance the pen/mouse is from the center of the screen.
If the pen/mouse is above the center, the edited solid will become larger.
If it is below the center, the solid will become smaller.
The parameter (p) command can be used to enter precise scale factors.
The value entered is applied to the solid as it existed when solid
scale was entered.
Hence entering {\em p 1} will return the solid to its size when solid scale
session first started.

\section{Solid Parameter Editing}

To modify individual solid parameters, press the menu button or select
``edit menu'' from the ***SOLID EDIT*** menu.
A menu listing what parameter editing is available for that particular
solid type will be displayed.
Using the pen/mouse select the desired item(s) from this menu.
For most of the parameter editing, the {\em p} command can be used to
make precise changes.
Parameter editing is the default edit mode entered when MGED first
enters the solid edit state.
In the following paragraphs, we will discuss parameter editing
for each of the MGED general types of solids.

\mfig menu-arb-ctl, ARB Control Menu.

\subsection{ARB Parameter Editing}

The GENERAL ARB class of solids represents all the convex polyhedrons
(RPP, BOX, RAW, and ARBs).
The ARBs comprise five classes of polyhedrons each with a characteristic
number of vertices.
These are the ARB8, ARB7, ARB6, ARB5, and ARB4, where the ARB8 has
eight vertices, etc.
During editing, all the vertices are labeled on the screen.

An ARB is defined by a fixed number of vertices where all faces must
be planar.  This fact means that during parameter editing, movement
of individual vertices in faces containing four vertices is not allowed.
There are three classes of parameter editing that can be done to ARBs:
move edges,
move faces, and rotate faces.  There is an ``ARB control menu''
(see Figure \ref{menu-arb-ctl}) from
which one selects the type of parameter editing to be done.
A specific ARB edit menu will appear dependent on which parameter editing
option was selected.  The ``return'' entry on each of these specific menus
will return the ``ARB control'' menu to the screen.

Note that there are several keyboard commands that apply only to ARB solids
which are being edited in SOLID EDIT state.
Once such command is {\em mirface}, which replaces a designated
face of the ARB with a copy of an original face mirrored about
the indicated axis.
Another such command is {\em extrude}, which projects a designated face
a given amount in the indicated direction.

\mfig menu-arb8-edge, Move Edge Menu for ARB8.
\mfig menu-arb4-edge, Move Edge Menu for ARB4.
\mfig menu-arb8-face, Move Face Menu for ARB8.
\mfig menu-arb4-face, Move Face Menu for ARB4.

\subsection{Move ARB Edges}

To move an ARB edge, select the desired edge from the ``move edge'' menu.
For example, Figure \ref{menu-arb8-edge} shows the menu for
moving an edge of an ARB8, and Figure \ref{menu-arb4-edge}
shows the menu for moving an edge of an ARB4.
A point is then ``input'' either through a pen press or through the {\em p}
command.
The line containing the selected edge is moved so that it goes through
coordinate of the input point.
Any affected faces are automatically adjusted to remain planar.

\subsection{Move ARB Faces}

To move an ARB face, select the desired face from the ``move face'' menu.
A point is then ``input'' either through a pen press or through the {\em p}
command.  The plane containing the edited face is then moved so that it
contains the input point.  The new face is then calculated and the ARB
is displayed.
The move face menus for an ARB8 are shown
in Figure \ref{menu-arb8-face}, and the move face menus for an ARB4
are shown in Figure \ref{menu-arb4-face}.

\mfig menu-arb8-rot, Rotate Face Menu for ARB8.
\mfig menu-arb4-rot, Rotate Face Menu for ARB4.

\subsection{Rotate ARB Faces}

ARB faces may be rotated around any of the vertices comprising that face.
First, select the desired face from the ``rotate face'' menu.  You will then
be asked to select the vertex number around which to rotate the face.
The face can be rotated about the three coordinate axes.  The knobs (Rotate X,
Rotate Y, and Rotate Z) are used for this purpose.  For precise rotations,
use the {\em p} command.  If three values are entered after the {\em p}, then
they are interpreted as angles (absolute) of rotation about the X, Y, Z axes
respectively.  If only two values are entered, then they are considered as
rotation and fallback angles for the normal to that face.  The {\em eqn}
command can also be used here to define the plane equation coefficients of
the face being rotated.
The rotate face menus for an ARB8 are shown
in Figure \ref{menu-arb8-rot}, and the rotate face menus for an ARB4
are shown in Figure \ref{menu-arb4-rot}.

\mfig ped-tgc, Typical TGC During Parameter Editing.
\subsection{Truncated General Cone (TGC) Parameter Editing}

The TGC general class of solids includes all the cylindrical COMGEOM solids.
The defining parameters of the TGC are two base vectors (A and B), a height
vector (H), two top vectors (C and D), and the vertex (V).
The top vectors C and D are directed the same as the base vectors A and
B respectively,
hence the top vectors are defined only by their lengths (c and d).
During solid editing, only vectors A and B are
labeled on the display.
Figure \ref{ped-tgc} depicts a typical TGC during parameter editing.

It is possible to change the length of the H, A, B, C, or D
vectors, resulting in a change in height or eccentricity of the
end plates.  The overall size of the A,B or C,D end plates can
be adjusted, or the size of both can be changed together, leaving
only the H vector constant.
The H vector or the base plate (AXB) can be rotated.
Recall that vectors A \& C and vectors B \& D have like directions, hence
rotating the base (AXC) will automatically rotate the top (BXD).
Finally, one can move the end of the height vector H
with the TGC becoming or remaining
a right cylinder (move end H (rt)),
or with the orientation of the base (and top)
unchanged (move end H).
Either the mouse/tablet or the {\em p} command can be used.
These functions are selected from the menu which can be seen
in Figure \ref{ped-tgc}.

\mfig ped-ell, Ellipsoid Parameter Editing Menu.
\subsection{Ellipsoid Parameter Editing}

The ELLG general class represents all the ellipsoidal solids, including
spheres and ellipsoids of revolution.
The defining parameters of the ELLG are three mutually perpendicular
vectors (A, B, and C) and the vertex (V).
When an ELLG is being edited, only vectors A and B are labeled on the display.
Figure \ref{ped-ell} depicts a typical ELLG during parameter editing.

The parameter editing of the ELLG consists of scaling the lengths of the
individual vectors A, B, C.
One may also scale all theses vectors together of equal length.

The scaling of these vectors is done using the data tablet/mouse in
exactly the same manner as in object scaling.
The {\em p} keyboard command again can be used to produce a vector of
desired length.

\mfig ped-tor, Torus Parameter Editing Menu.
\subsection{Torus Parameter Editing}

The TOR general class of solids contains only one type of torus, one with
circular cross-sections.
The defining parameters of the TOR are two radii (r1 and r2), a normal
vector (N), and the vertex (V).
The scalar r1 is the distance from the vertex to the midpoint of the
circular cross section.
The scalar r2 is the radius of the circular cross-section.
The vector N is used to orient the torus.
During solid editing, none of these parameters are labeled on the screen.
Figure \ref{ped-tor} depicts a typical torus during parameter editing.

The parameter editing of the TOR consists of scaling the radii, hence the
menu contains only two members.
\chapter{KEYBOARD COMMANDS}

The MGED keyboard commands are used to maintain overall control of the
system and to perform general housekeeping functions.
They are summarized in Figure \ref{cmd-summary}.

\begin{figure}[tb]
\begin{tabular}{l l l}
Key	& Argument[s]	& Description \\
\\
  e	& obj1* obj2* ... objn*	& display objects on the screen \\
  E	& obj1* obj2* ... objn*	& display objects evaluating regions \\
  B	& obj1* obj2* ... objn*	& Zap screen, display objects \\
  d	& obj1* obj2* ... objn*	& delete objects from screen \\
  cp	& oldobj newobj	& copy 'oldobj' to 'newobj' \\
  cpi	& oldtgc newtgc	& copy 'oldtgc' to 'newtgc' inverted \\
  Z	& -none-	& Zap (clear) the screen \\
  g	& groupname obj1* obj2*....objn*	& group objects \\
  r	& region op1 sol1....opn soln	& create/modify a region \\
  i	& object instname	& create instance of an object \\
  mv	& oldname newname	& rename object \\
  mvall	& oldname newname	& rename all occurences of an object \\
  l	& object*	& list object information \\
  kill	& obj1* obj2* ... objn*	& remove objects from the file \\
  killall	& obj1* obj2* ... objn*	& remove object[s] + references from file \\
  killtree	& obj1* objn* ... objn*	& remove complete paths  **CAREFUL** \\
  t	& object*	& table of contents \\
  mirror	& oldobj newobj axis	& mirror image of an object \\
  mirface	& \#\#\#\# axis	& mirror face \#\#\#\# about an axis \\
  extrude	& \#\#\#\# distance	& extrude an arb face \\
  item	& region item air	& change region item/air codes \\
  mater	& region material los	& change region mat/los codes \\
  rm	& comb mem1* mem2*....memn*	& delete members from combination \\
  units	& mm|cm|m|in|ft	& change the units of the objectfile \\
  title	& new-title	& change the title of the description \\
  p	& dx [dy dz]	& precise commands for solid editing \\
  rotobj	& xdeg ydeg zdeg	& rotate(absolute) an edited object \\
  scale	& factor	& scale(absolute) an edited object \\
  translate	& x y z	& translate an edited object \\
  arb	& name rot fb	& make arb8 with rot and fb \\
  analyze	& solids	& print much info about a solid \\
  summary	& s|r|g	& solid/region/group summary \\
  tops	& -none-	& list all top level objects \\
  find	& obj1* obj2* ... objn*	& find all references to an object \\
  area	& [endpoint-tolerance]	& find presented area of E'd objects \\
\end{tabular}
\caption{MGED Command Summary 1 \label{cmd-summary} }
\end{figure}

\begin{figure}
\begin{tabular}{l l l}
Key	& Argument[s]	& Description \\
\\
  plot	& [-zclip] [-2d] [out-file] [| filter]	& make UNIX-plot of view \\
  color	& low high r g b str	& assign color(r g b) to item range \\
  edcolor	& -none-	& text edit the color/item assignments \\
  prcolor	& -none-	& print the current color/item assignments \\
  make	& name type	& create and display a primitive \\
  fix	& -none-	& restart the display after hangup \\
  rt	& [options]	& raytrace view onto framebuffer \\
  release	& -none-	& release current display processor \\
  attach	& nu|tek|tek4109|plot|mg|vg|rat	& attach new display processor \\
  ae	& az elev	& rotate view w/azim and elev angles \\
  regdef	& item [air los mat]	& set default codes for next region created \\
  ted	& -none-	& text edit a solids parameters \\
  vrot	& xdeg ydeg zdeg	& rotate view \\
  ill	& name	& illuminate object \\
  sed	& solidname	& solid edit the named solid \\
  center	& x y z	& set view center \\
  press	& button-label	& emulate button press \\
  knob	& id value	& emulate knob twist \\
  size	& value	& set view size \\
  x	& -none-	& debug list of objects displayed \\
  status	& -none-	& print view status \\
  refresh	& -none-	& send new control list \\
  edcomb	& comb flag item air mat los	& edit comb record info \\
  edgedir	& delta\_x delta\_y delta\_z	& define direction of an ARB edge being moved \\
  in	& name type {parameters}	& type-in a new solid directly \\
  prefix	& string obj1* obj2* ... objn*	& prefix objects with 'string' \\
  keep	& file.g obj1* obj2* ... objn*	& keep objects in 'file.g' \\
  tree	& obj1* obj2* ... objn*	& list tree for objects \\
  inside	& --prompted for input--	& find inside solid \\
\end{tabular}
\caption{MGED Command Summary 2 }
\end{figure}

\begin{figure}
\begin{tabular}{l l l}
Key	& Argument[s]	& Description \\
\\
  solids	& file obj1* obj2* ... objn*	& make ascii solid parameter summary in 'file' \\
  regions	& file obj1* obj2* ... objn*	& make ascii region summary in 'file' \\
  idents	& file obj1* obj2* ... objn*	& make ascii region ident summary in 'file' \\
  edcodes	& obj1* obj2* ... objn*	& edit region ident codes \\
  dup	& file {prefix}	& checks for dup names in 'file' \& current file \\
  cat	& file {prefix}	& cat's 'file' onto end of current file \\
  track	& --prompted for input--	& builds track given appropriate 'wheel' data \\
  3ptarb	& --prompted for input--	& makes arb8 given 3 pts, etc. \\
  rfarb	& --prompted for input--	& makes arb8 given point, rot, fallback, etc \\
  whichid	& ident1 ident2 ... identn	& list all regions with given ident \\
  paths	& --prompted for input--	& lists all paths matching input path \\
  listeval	& -prompted for input--	& gives 'evaluated' path summary \\
  copyeval	& --prompted for input--	& copy an 'evaluated' path-solid \\
  tab	& obj1* obj2* ... objn*	& list objects as stored in data file \\
  push	& obj1* obj2* ... objn*	& push object transformations to solids \\
  facedef	& \#\#\#\# {data}	& define plane of an edited ARB face \\
  eqn	& A B C	& define plane coefficients of rotating ARB face \\
  q	& -none-	& quit \\
  %	& -none-	& escape to shell \\
  ?	& -none-	& help message \\
\end{tabular}
\caption{MGED Command Summary 3 }
\end{figure}

\section{Copy Object}

{\em \center cp oldobj newobj}

This command is used to produce a copy of an object (solid or comb).
In this case, the
object "oldobj" will be copied into an object called "newobj".

Examples:
{\em
              cp arb8 hullbot.s \\
              cp tgc wheelrim.s \\
              cp torso.r driver\_torso \\
              cp proto.man driver \\
}

\section{Zap Screen}

{\em \center Z}

This is the Zap command.  It clears all objects from the screen.

\section{Drop objects from display screen}

{\em \center d obj1 obj2 ... objn}

This command allows one to remove objects from the display screen.  In
this case "obj1" thru "objn" will be removed from the display.

\section{Move (rename) object}

{\em \center mv old new}

This command is used to rename objects in the data file.  In this
case, the object "old" will be renamed "new".
A note of caution:  the name is changed only in the object record itself, not
in any member records.  Thus if the object "old" appears as a member
of any other object, the name will not be changed there.
To rename all occurrences of an object, use the "mvall" command.

Examples:
{\em
              mv test hull \\
              mv g00 air \\
              mv g1 turret \\
}

\section{Set Local Working Units}

{\em \center units ab}

This command allows one to change the local or working units at ANY time.
The only allowable values for "ab" are "mm", "cm", "m", "in", or "ft".

Examples:
{\em
           units mm \\
           units in \\
}

\section{Group objects}

{\em \center g group obj1 obj2 ..... objn}

This command creates or appends to a combination record and
is used to group objects together either for editing or displaying
purposes.  In this case, "obj1" through "objn" are added as members
to the combination "group".  If "group" does not exist, it is
created and "obj1" through "objn" are added as members.
NOTE: no checking to see if "obji" is already a member of "group".

Examples:
{\em
            g shell hull turret \\
            g tank wheels engine crew shell \\
            g tank track \\
}

\section{Create Region}

{\em \center
r region op1 sol1 op2 sol2 .... opn soln
}

This command is used to create regions or append to regions.
If "region" exists, then solids "sol1" through "soln" are
added as members with "op1" through "opn" as the defining operations.
If "region" does not exist, then it is created and solids "sol1" through
"soln" are added as members with "op1" through "opn" as the
defining operations.  A region is merely a combination
record with a flag set and is distinguished from other combinations (groups)
since it has meaning to the COMGEOM solid modeling system.
Note that "+" or "u" must be the first operations in a region.

When a region is created, the item and air codes are set equal to default values.
If the "regdef" command has been used, then those values will be used,
otherwise the values "1000 0 100 1" will be used respectively.
To change
the item and air codes use the "item" command.
The "edcodes" command is probably the easiest and fastest way to change these
identifing codes.
Note:  In the past, all members of a region had to be solids, but
recently combinations have been allowed as members of regions.  Hence,
the names "soli" can also be combinations (groups) now.
Also, as in grouping, no checking for members already in a region.

Examples:
{\em
             r hulltop.r + hulltop.s -- hullleft.s -- hullright.s \\
             r gun + gun.s -- gunin.s \\
             r gunair + gunin.s \\
}

\section{Instance an object}

{\em \center
i object combname
}

This command is used to make an instance of an object.
An instance of an object is produced by creating a combination
and making the object a member.
In this case, an instance of "object" is made by
creating the combination record "combname" (if "combname" does not
already exist) and adding "object" as a member.
If "combname" already exists, then "object" is added as the next member.

An instance is used to refer to an object, without making actual copies
of the object.  Instances are useful when one is adding a certain
component to a target description many times.
Furthermore, any modifications to an object which has been instanced need only be
done in the original (prototype) object.
These modifications will then be automatically reflected in all the
instances of the object.

Examples:
{\em
                i heround he1 .he1. \\
                i heround he2 .he2. \\
                i heat heat1 .heat1. \\
                i heat heat2 .heat2. \\
}

\section{Change Title of Database}

{\em \center
title newtitle
}

This command allows one to change the title of the model database at any time.
The string "newtitle" will become the new title, and may contain blanks.
The title is limited to 72 characters including blanks.

Examples:
{\em
          title XM89A -- New version of tank \\
          title M345 (groups are m345 and m345a) \\
}

\section{Extrude}

{\em \center
extrude \#\#\#\# distance
}

This command allows the user to project
a face(\#\#\#\#) of an arb being edited a normal distance to create a new arb.
The value of "face" is 4 digits such as 1256. If the face is projected
in the wrong direction use a negative "distance".
One common use for this command is
for producing armor plates of a desired thickness.

Examples:
{\em
              extrude 1234 20 \\
              extrude 2367 34.75 \\
              extrude 2367 -34.75 \\
}

\section{Remove members from Combination}

{\em \center
rm comb mem1 mem2 .... memN
}

This command allows one to delete members from a combination (group or region) record.
In this case, members "mem1" through "memn" will be deleted from
the combination "comb".

Examples:
{\em
              rm tank hull wheels \\
              rm region1 solid8 solid112 \\
              rm turtop.r tursidel.s tursider.s  \\
}

\section{List Object Information}

{\em \center
l object
}

This command is used to list information about objects in the data file.
The information listed depends on what type of record "object" is.
If "object" is a combination record, then the members are listed.
If "object" is a solid record, then the MGED general solid type and
the parameters as presently in the data file are listed.
Note:  only the solid parameters as they exist in the solid record are
listed, no transformation matrix is applied.
Hence, if the solid was edited as a member of a combination, the "l"
command will not reflect the editing in the listed parameters.
To produce this type of listing, see the "listeval" command.

Examples:
{\em
               l hull \\
               l turret \\
               l turtop.s \\
               l arb8 \\
}

\section{Analyze Solid}

{\em \center
analyze solid
}

This command produces information about a solid (all except ARS).
The information includes surface area(s) and volume.
Also, in the case of ARBs, edge lengths and rot and fallback angles
and plane equations are given for each face.
If "solid" is present that solid name will be used and analyzed.
If "solid" is absent, the solid at the bottom of the present path
being edited will be analyzed.

\section{Mirror Object}

{\em \center
mirror oldobj newobj axis
}

This command is used to create a new object which is
the mirror image of an existing object about an axis.
The object may be either a solid or a combination.
In this case, a mirror image of the object "oldobj" will created
about the axis indicated by "axis" and the new object record will
be called "newobj".
The only acceptable values for the parameter "axis" are "x", "y", and "z".

Examples:
{\em
              mirror tur.left.s tur.right.s y
              mirror tur.top.s tur.bot.s z
              mirror tur.front.s tur.back.s x
              mirror lt\_gun rt\_gun y
}

\section{Create ARB8}

{\em \center
arb name rot fb
}

This command allows one to create an arb8
with the desired rotation and fallback angles.
In this case, an arb8 with the name of "name" will be created with the desired
rotation angle of "rot" degrees and the fallback angle of "fb" degrees.

Examples:
{\em
           arb top1.s 0 90 \\
           arb sidelt.s 90 0 \\
           arb upglacis.s 0 60 \\
}

\section{Change Item (Ident) and Air codes of Region}

{\em \center
item region ident air
}

This command allows one to change the item or
air code numbers of a region.  If the air code ("air") is not included,
a zero is assumed.
To change the air code, a zero item code must be used (see second
example below).

Examples:
{\em
            item region1 105 \\
            item region7 0 2 \\
            item region11 129 0 \\
}

\section{Specify Material Properties}

{\em \center
mater comb [material]
}

This command is used to change the material properties specification for
a combination.

\section{Edit Combination Record Info}

{\em \center
edcomb comb regionflag regionid air los GIFTmater
}

This command is used to change the material and the los percent for
a region.

\section{Edit (Display) an object on the screen}

{\em \center
e obj1 obj2 ... objn
}

This command allows one to display objects on the screen.
In this case, "obj1" thru "objn" will be displayed on the screen.

\section{Evaluated Display of Object on the screen}

{\em \center
E obj1 obj2 ... objn
}

This command is the same as the "e" command, except the regions will
be evaluated before being displayed.

\section{Zap screen and Display Object}

{\em \center
B obj1 obj2 ... objn
}

This command is the same as the "e" command, except that the screen
is cleared (Zap) before the objects are displayed.

\section{Kill (delete) object from database}

{\em \center
kill obj1 obj2 ... objn
}

This command allows one to remove objects from the file itself.
Only the object records themselves are removed, any references made
to these objects still will exist.
To remove the references also, see the "killall" command.

\section{List Table of Contents}

{\em \center
t obj1 obj2 ... objn
}

This is the table of contents command.  If arguments are present, a list
of all objects in the file matching these names will be printed.
If there are no arguments, then a listing of all objects will be printed.

\section{List Tree Tops}

{\em \center
tops
}

This command will search the target file hierarchy, and list all "top level"
objects (objects which are not members of any other object).
This command is useful to make sure objects have been grouped properly.

\section{Make prototypical solid}

{\em \center
make name type
}

This command will create a solid of a specified type.
This solid will be named "name" and the solid type will be "type".
The acceptable types are: arb8, arb7, arb6, arb5, arb4, tor, tgc, tec,
rec, trc, rcc, ellg, ell, sph.
This new solid will be drawn at the center of the screen.
This command should be used to create
solids for editing.

\section{Mirror ARB Face}

{\em \center
mirface \#\#\#\# axis
}

This command allows one to mirror a face of an edited arb about an axis.
This command is quite useful for adding air to a "symmetric" target.

\section{Print Summary of Objects}

{\em \center
summary s|r|g
}

This command will produce a summary of objects in the target file.
If the options s, r, or g are entered a listing of the solids, regions,
or groups will also be presented.

\section{Specify Numeric Parameter(s)}

{\em \center
p dx [dy dz]
}

This is the parameter modification command and is used during solid
editing to make precise changes.
The actual meaning of the values typed after the "p", depend on what
editing option is being performed.
If one were translating a solid, then the values would be the x,y,z
coordinates of the vertex of the solid.

\section{Release Current Display}

{\em \center
release
}
This command releases the current display device,
and attaches the null device.

\section{Attach to Display Device}

{\em \center
attach device
}

This command allows one to attach a display device (the present display device
is released first).
The present acceptable values for "device" are vg, mg, tek, rat, plot, tek4109, ir, sgi, and nu.
The "plot" device will produce a UNIX-plot of the present view (including the
faceplate) on a display device using a specific filter.
You will be asked which filter to use.
Sample filters include "tplot" and "plot-fb".

\section{Numeric Object Rotation Edit}

{\em \center
rotobj xdeg ydeg zdeg
}

This command allows one to make precise rotations of an object during
object editing.
MGED must be in the "object edit" state for this command to have effect.
If "object rotation" is not in effect, MGED will select this option for you
and perform the rotation.
The object will be rotated "xdeg" about the x-axis, "ydeg" about the
y-axis, and "zdeg" about the z-axis.
The rotation is "absolute"....the total rotation since the beginning
of object editing will be equal to the input values.
The rotation is done about the "KEY" point for the object being edited.

\section{Scale Edited Object}

{\em \center
scale xxx.xx
}

This command allows one to make precise scaling changes to an object
during object editing.
MGED must be in the "object edit" state for this command to have effect.
If one of the object scale options is not in effect, the "global scale"
options will be selected.
The object will be scaled by a TOTAL amount equal to the input value.
If one of the local scale options is in effect, the object will be
scaled in the selected axis direction by an amount equal to the input value.
The scaling is done about the "KEY" point of the object being edited.

\section{Translate Edited Object}

{\em \center
translate xxx.xx yyy.yy zzz.zz
}

This command allows one to make precise translation changes to an object
during object editing.
MGED must be in the "object edit" state for this command to have effect.
If the object translation option is not in effect, this option will be
selected and the translation performed.
The "KEY" point of the object being edited will move to the input coordinates.

\section{Fix Broken Hardware (sometimes)}

{\em \center
fix
}

This command will "fix" (restart) the display device after a hardware error.

\section{Ray-Trace Current View}

{\em \center
rt [-s\#]
}

This command will run the {\bf rt}(1) program
to produce a color shaded iamge of objects on the currently
selected framebuffer.
The resolution of the iamge (number of rays) is equal to "\#" from the "-s"
(square view resolution) option.
If the -s option is absent, 50x50 ray resolution will be used.

\section{Emulate Knob Twist}

{\em \center
knob id value
}

This command is used to emulate a "knob twist".
Generally this command is used for display devices which have no actual
knob peripherals (eg. tek).
Any non-zero number entered for "value" is converted to 1 (if "value" is
greater than zero) or is converted to -1 (if "value is less than zero).
The user must enter the same command with "value" equal to zero to
stop the action envoked by the knob twist.

The "id" defines which knob is to be "twisted":

\begin{tabular}{rl}
	x	& rotates about x-axis \\
	y	& rotates about y-axis \\
	z	& rotates about z-axis \\
	X	& slew view in x direction \\
	Y	& slew view in y direction \\
	Z	& zoom the view \\
\end{tabular}

Examples:
{\em
       knob x 1 \\
       knob x 0 \\
       knob Z -1 \\
       knob Z 0 \\
}

\section{Solid\_Edit Named Solid}

{\em\center
sed name
}

This command allows one to immediately enter the solid edit mode
with the solid "name" as the edited solid.
Note that the solid must be displayed but not multiply referenced.

\section{Illuminate Named Object}

{\em\center
ill name
}

This command is used to illuminate an object ... a path containing this
object ("name") will be illuminated.
This command is primarily used with display devices which do not have
a tablet to pick objects for editing.

\section{Rotate the View}

{\em\center
vrot xdeg ydeg zdeg
}

This command rotates the VIEW "xdeg" degrees about the screen x-axis,
"ydeg" degrees about the screen y-axis, and "zdeg" degrees about the
screen z-axis.

This command is useful when the precise rotation desired is known.
It is also useful when in a rotation edit mode, and the viewing
rotation needs to be changed, without affecting the current edit.

\section{Move Screen Center}

{\em\center
center xx.xx yy.yy zz.zz
}

This command moves the screen center to (xx.xx, yy.yy, zz.zz).
Using this command is one way of slewing the view.

\section{Set View Size}

{\em\center
size xx.xx
}

This command sets the view size to xx.xx and is one way of zooming
the display.
Making the view size smaller has the effect of zooming in on the view.

\section{Extended List of all Objects in Displaylist}

{\em\center
x
}

This command produces a list of all objects displayed, listing the center
of the object, its size, and if it is in the present view.
It is intended primarily for software debugging.

\section{Refresh Display}

{\em\center
refresh
}

This command will send a new display list to the display device.

\section{Print View Status}

{\em\center
status
}

This is a debug command which prints the status of the current view,
including all viewing and editing matrices.

\section{Simulate Button Press}

{\em\center
press button-label
}

This command allows one to emulate a button press and is generally used
on display devices which do not have actual button peripherals.
The following are the strings allowed for "button-label" and all produce
the indicated view on the device screen:
        top, bottom, right, left, front, rear, 90,90, 35,25

The following is a listing of the remaining acceptable strings for
"button-label" and the resulting action:

\begin{tabular}{rl}
       reset         &  reset the view \\
       save          &  save the present view \\
       restore       &  restor the saved view \\
       adc           & display the angle-distance cursor \\
       oill          &  begin object illumination (pick) \\
       sill          & begin solid illumination (pick) \\
       oscale        & object scale \\
       ox            & object translat ion in x direction only \\
       oy            & object translation in y direction only \\
       oxy           & object translation \\
       orot          & object rotation \\
       sedit         & put up solid parameter menu \\
       srot          & solid rotation \\
       sxy           & solid translation \\
       sscale        & solid scale \\
       accept        & accept editing done \\
       reject        & reject editing done \\
\end{tabular}

Examples:
{\em
         press 90,90 \\
         press front \\
         press oill \\
         press orot \\
         press reject \\
}

\section{Escape to the Shell}

{\em\center
%
}

This command allows one to escape to the shell to perform multiple commands
without having to terminate the current MGED session.
To return to mged, enter a control-d to the Shell.
Note that the "!" escape at the beginning of a line can be used
to send a single command to the shell.

\section{Get Short Help Listing}

{\em\center
?
}

This is the short form of the help command,
that lists the names of all MGED commands.

\section{Get Long Help Listing}

{\em\center
help
}

This is the long form of the help command,
that produces a listing of all the available commands and their
arguments, and a one sentence summary of the commands purpose.

\section{Exit (Quit) MGED}

{\em\center
q
}

Running the "q" command, or entering an End-Of-File (EOF) (typ. Control/D)
is the normal way of exiting MGED.

\section{Copy and Translate TGC}

{\em \center
cpi oldtgc newtgc
}

This command is a specialized copy command and is designed to be used when one
is "running wires" in a description.
The object being copied must a cylindrical solid (TGC).
The following occurs when cpi is used:  first the cylinder ("oldtgc") is
copied to "newtgc"; then "newtgc" is translated to the end of "oldtgc";
then "newtgc" is displayed; and finally, MGED is put in the SOLID EDIT state
with "newtgc" as the edited solid.

\section{Remove Object and All References}

{\em \center
killall obj1 obj2 ... objn
}

This command will accomplish two things:  first, the object[s] will be
removed from the data file just as in the "kill" command;  second, all
references to the object[s] will also be removed.

\section{Remove Complete Tree}

{\em \center
killtree obj1 obj2 ... objn
}

This command will remove from the file complete trees originating with
obj1, obj2, ..., objn.
Every object in the designated paths will be removed from the file, hence
CAUTION is urged.
Make sure that "killtree" is what you want to do.
Using the "paths" or "tree" command on an object before "killtree" will
show what objects will be killed.

\section{Add Color To Display}

{\em \center
color low high r g b string
}

This command allows one to make color assignments to a range of item codes.
The arguments "low" and "high" are the item ranges.
The arguments "r", "g", and "b" are the red, green, and blue values
respectively (the range of these numbers is generally 0-255).
The argument "string" is a string describing this class of items.
A blank is considered a terminator, so there can be no blanks in
this string.

\section{Edit Display Colors}

{\em \center
edcolor
}

This command allows one to edit the existing color assignments (table).
The changes are made using the user's "selected" text editor
found in environment variable EDITOR.
Note that this method of specifying object colors is obsolete,
and has been replaced by the {\em mater} command.

\section{Print Display Colors}

{\em \center
prcolor
}

This command prints the color assignments as they presently exist.

\section{Find Objects}

{\em \center
find obj1 obj2 ... objn
}

This command will find ALL references of obj1 obj2 ... objn in the file.

\section{Estimate Presented Area}

{\em \center
area [tolerance]
}

This command finds an estimate of the presented area of all E'd objects
in the present view from that aspect.
The argument "tolerance" is the tolerance for the endpoints of
line segments being "equal" and is optional.

\section{Produce UNIX Plot}

{\em \center
plot [-zclip] [-2d] [out-file] [| filter]
}

This command is used to produce a UNIX-plot hardcopy of the present view of
the 'geometry' on the display device.
The MGED faceplate will not be drawn.
Some useful examples are:

\begin{tabular}{rl}
plot-fb & LIBFB framebuffer (low res) \\
plot-fb -h & LIBFB framebuffer (high res) \\
tplot -Ti10 & Imagen laser printer \\
tplot -Tmeg & Megatek 7250 \\
tplot -T4014 & Tek4014 \\
tplot -Thpgl & HP7550A plotter
\end{tabular}

\section{Text Edit Solid Parameters}

{\em \center
ted
}

This command allows one to edit a solid's parameters using the text editor
defined by the users's path.
The solid being edited (solid edit mode) will be the one that will be
"text edited".

\section{Keyboard Input of Solid Parameters}

{\em \center
in name type {parameters}
}

This command allows one to enter a new solid directly via the keyboard.
The user will be prompted for any missing input and the solid will be
displayed on the screen.  WARNING: only minimal checking of parameters
is done.

\section{Change View Azimuth, Elevation}

{\em \center
ae az elev
}

This command sets the display viewing angles using the input azimuth (az)
and elevation (elev) angles.

\section{Define Region Identifiers}

{\em \center
regdef item [air los mat]
}

This command allows one to change the default codes which will be given
to the next region created.
If the "air" code is non-zero, then the "item" code will be set to zero.

\section{Change Edge Direction}

{\em \center
edgedir delta\_x delta\_y delta\_z
}

This command allows one to define the direction of an ARB edge being moved.
If only two arguments are input, the code assumes these to be the
rotation and fallback angles for the edge.
If three arguments are present, the code will use them as "deltas" to
define the direction of the edge.
Note that this command can be useful to find the intersection of
a line (edge) with planes (faces).

\section{Prefix Object Names}

{\em \center
prefix string obj1 obj2 ... objn
}

This command will prefix obj1, obj2, .... objn with "string".  All occurences
of these names will be prefixed.  String matching is allowed for the objects
to prefix.

\section{Keep Objects in Another File}

{\em \center
keep file.g obj1 obj2 ... objn
}

This command allows one to keep the listed objects in the file "file.g".
This command is useful for pulling out parts of a description.

\section{List Object Hierarchy}

{\em \center
tree obj1 obj2 ... objn
}

This command will print the tree structure for the objects listed.

\section{Create Inside Solid}

{\em \center
inside
}

This command is used to define a solid (inside solid) such that when it is subtracted
from another certain solid (outside solid), the resulting region will have desired
thicknesses.
To invoke this option, enter "inside".
You will then be prompted for the required data.
If you are in the solid edit mode, the "outside solid" will default to the
solid presently being edited.
If you are in the object edit mode, the "outside solid" will default to the
"key solid" of the object path being edited.
If you are not in an edit mode, you will then be asked for the name of
the "outside solid".
Next, you will be asked what name you wish to call the new "inside solid"
to be calculated.
Finally, you will be asked to enter the thickness[es], depending on the
solid type.
The "inside solid" will then be displayed on the screen.
If the thickness values input are negative, then the thickness will be
directed to the "outside" of the solid.

\section{Produce ASCII Summary of Solids}

{\em \center
solids file obj1 obj2 ... objn
}

This command will produce an ascii summary of all solids involved with
objects obj1, obj2, ..., objn.  This summary will be written in 'file'.
In this file, all regions and solids are numbered and will match the
numbers of a COMGEOM deck produced by VDECK or GIFT if the objects are
entered in the same order.
The file 'file' will be overwritten if it already exists.
String matching is allowed for the objects.

\section{Produce ASCII Summary of Regions}

{\em \center
regions file obj1 obj2 ... objn
}

This command will produce an ascii summary table of all regions involved with
objects obj1, obj2,...,objn.
This summary table will be written in 'file'.
In this file, all regions and solids will be numbered and will match the numbers
of a COMGEOM description produced by VDECK or GIFT if the objects are entered
in the same order.
This file will be identical to the "solids" command except the actual
parameters are not listed.
The file 'file' will be overwritten if it already exists.
String matching is allowed for the object names.

\section{Produce ASCII Summary of Idents}

{\em \center
idents file obj1 obj2 ... objn
}

This command will produce an ascii summary table of all region idents involved with
objects obj1, obj2,...,objn.
This summary table will be written in 'file'.
In this file, all regions will be numbered and will match the region numbers
of a COMGEOM description produced by VDECK or GIFT if the objects are entered
in the same order.
At the end of this file will be the same region information, but ordered
by ident number.
The file 'file' will be overwritten if it already exists.
String matching is allowed for the object names.

\section{List Objects Paths}

{\em \center
paths
}

This command will print ALL paths matching an "input" path.
You will be asked to enter the path to match.
The path members are entered on ONE line, seperated by spaces.
The input path need not be a complete path, but must contain at least one object.
All paths with the same first members as the input path will be listed.
This command is useful for finding complete paths which begin with certain objects.

\section{List Evaluated Path Solid}

{\em \center
listeval
}

This command will "evaluate" and list ALL paths matching an input path, including the
parameters of the solids at the bottom of each path.
These parameters will reflect any editing contained in the path listed.
You will be asked to enter the path to match.
The path members are entered on ONE line, seperated by spaces.
The input path need not be a complete path, but must contain at least
one object.
Note that since the solid parameters are printed, this could be a
rather lengthy listing depending on the completeness of the input path.

\section{Evaluate Path and Copy Solid}

{\em \center
copyeval
}

This command allows one to copy an "evaluated solid", that is a complete path
ending in a solid.
You will be asked to enter a complete path.
Again, this path is entered on ONE line with the members seperated by spaces.
If you do not know the complete path, use the "paths" command above to find it.
Next, you will be asked to enter the name of this copied solid.
The input path will be traversed and the accumulated path transformations will
be applied to the parameters of the bottom solid.
These new parameters will then be the copied solid.
Note:  this command is useful for making "dummy solids" to subtract for
overlaps between objects which have been "object" edited.

\section{Edit Objects Region Identifiers}

{\em \center
edcodes obj1 obj2 ... objn
}

This command provides for an easy way to modify the code numbers (item, air, material, and los)
of ALL regions found in paths beginning with an object.
For each object, all paths beginning with that object are traversed until a region
is encountered...at which time the following is listed:

item  air  mat  los      /obj1/.../region.n

The cursor then jumps back to "item" at the beginning of the line.
At this time the cursor can be advanced only by entering a new item code
(only digits are allowed) or by hitting the "space bar" or "tab key".
The space and tab will move the cursor to the next position in the line.
Pressing the "backspace key" will move the cursor to the beginning of
the next location to the left.
Moving the cursor "past" the los location
will return it to the beginning of the line.
When the code numbers are as you desire,
a RETURN will print the next line
for editing.
At any time, pressing the "R" key will restore the idents on the current
line to the way they were originally.
A Control/C or a "q" will abort the process at the current line.
String matching is allowed for the object names.

\section{List Object As Stored}

{\em \center
tab obj1 obj2 ... objn
}

This command will produce a listing of objects as they are stored in
the MGED object data file.
String matching is allowed for the objects.

\section{List Regions With Given Ident}

{\em \center
whichid item1 item2 ... itemn
}

This command will list ALL regions in the data file which have certain
item codes.

\section{Create ARB Given 3 Points}

{\em \center
3ptarb
}

This command will produce a "plate mode" arb8 given 3 points on one face, 2 coordinates
of the 4th point on that face, and a thickness.
The 3rd coordinate of the 4th point will be solved for and the "other" face
will be a normal distance (== the desired thickness) away.
You will be asked for all necessary input.

\section{Create ARB Given Point and Angles}

{\em \center
rfarb
}

This command will produce a "plate mode" arb8 given a point on one face,
rotation and fallback angles for that face, 2 coordinates of the 3
remaining points on that face, and a thickness.
The 3rd coordinates of the three points will be solved for, and the other
face will be a normal distance equal to "thickness" away.
You will be asked for all necessary input.

\section{Push Editing Down Paths}

{\em \center
push obj1 obj2 ... objn
}

This command will "push" an object's path transformations to the solid's parameters
located at the bottom of each path.
If a conflict is encountered, then an error message is printed and NOTHING is done.
A conflict occurs when the same solid "ends" different paths but the transformations
are different.
Conflicts could occur when "instanced" or "copied" groups occur in an object's
paths or when a solid is a member of 2 regions which have been edited separately.
A complication of this command, which will not be listed as a conflict, is when a solid's parameters have been
changed by a push, yet this solid is referenced by another object which was NOT
included in the push.
The user should beware of this situation.
This command can be very useful when "adding" parts from another file.
Once the "object editing" of these new parts is completed, the push command
will put the editing done to the solid level.
Since the added parts should have no cross-references with the existing
objects, there should be no problems.

\section{Check For Duplicate Names}

{\em \center
dup file.g {string}
}

This command will compare the current object data file with another
MGED file "file.g" and list any object names common to both files.
If "string" is present, all object names in "file.g" will be prefixed with
"string" before comparing for duplicate names.
Generally, one uses "string" only when duplicate names are found without it.

\section{Concat Files}

{\em \center
concat file.g {string}
}

This command will concatenate another MGED file "file.g" onto the present
object data file.
If "string" is present, all names in "file.g" will be prefixed with this
string.
No objects from "file.g" will be added if the name already occurs in the
current object file...the object names will be listed and skipped.
However, this command should be used in conjunction with the "dup" command to
eliminate any problems with duplicate names.

\section{Create Pseudo-Track}

{\em \center
track
}

This command adds track components to the data file to fit specified
"wheel" data.
Solids, regions, and a group containing the regions will be created and
displayed on the screen.
You will be prompted for all required data.

\section{Define ARB Face}

{\em \center
facedef \#\#\#\#
}

This command is used to define the face plane of an ARB that is being edited.
The following is the option menu displayed when this command is used:
        a     planar equation
        b     3 points
        c     rot and fb angles + fixed point
        d     same plane thru fixed point
        q     quit
To select any of these methods of defining a plane, enter the appropriate
letter.  You will then be asked for the desired input.

\section{Define Plane Equation of ARB Face}

{\em \center
eqn [A B C]
}

This command is used when one is rotating the face of an edited ARB and defines
the coefficients of the face planar equation (Ax + By + Cz).

\section{Move (Rename) Everywhere}

{\em \center
mvall old new
}

This command is used to rename all occurrences of an object in the data file.  In this
case, the object "old" will be renamed "new" for every occurence.

\section{Miscellaneous Commands}

It is possible to view, specify, and text-edit
information pertaining to the material type and color of various
parts of the model tree.  This is an interim capability
intended to provide enough material properties information for
current rendering and analysis purposes until the design of a
full material properties database can be finalized.

In addition to a variety of usual database manipulation
and status commands, there are commands to
compare the current database for name overlap (conflicts)
with another database, as well as commands to import and export
subtrees to/from the current database.
If name conflicts between the two databases
do exist, there are commands to rename an individual node without
changing any of the references to it (``mv''), or to rename
a node and change all the references to it (``mvall'').
Another command which is useful for preparing to move subtrees between
databases is the ``push'' command,
which adjusts the transformation matrices from the indicated point
down to the leaves of the directed acyclic graph, leaving the higher
level arcs with identity matrices.

\section{UNIX-Plot Output}

The ``plot'' command can store an exact image of the current
(non-faceplate) display on the screen, either using the
System V standard 2-D monochrome UNIX-Plot ({\em plot(4)}) format,
or the BRL 3-D color extended-UNIX-Plot format.
These plots can be sent to a disk file,
or ``piped'' directly to a filter process.
This can be useful for making hard copies of the current MGED view
for showing to others,
using a local pen plotter or laser printer.

\section{Ray-Tracing the Current View}

An important capability even beyond the ability to generate an
evaluated boundary wireframe is the ability of MGED to initiate a
quick ray-trace rendering of the current view on any nearby framebuffer!
This is implemented by using the MGED ``rt'' command to fork off
an instance of the RT program, and sending the RT program
a description of the current view
with any user-specified options.
This allows the designer to
use the power of MGED to select the desired view, and then to
quickly verify the geometry and light source placement.
A 50 by 50 pixel rendering of the current view can usually be done
in less than a minute (on a DEC VAX-780 class processor), and allows
for general verification before the designer uses the ``saveview''
command to submit a batch job for a high resolution ray-trace of
the same view.

\section{Animation}

The MGED editor includes a number of features which are useful
for developing animation tools and scripts.
The full description of the current viewing transformation and eye position
can be saved in a file,
and such a previously saved view can be read back at any time,
immediately changing the editor's view.
In addition, the current viewing transformation and eye position can be
appended to a file containing a collection of keyframes.
Most importantly, a file full of keyframe information, either raw keyframe
positions or smoothly interpolated keyframe sequences, can
by ``played'' in real time using MGED,
providing a powerful and fast animation preview capability.

As a separate animation capability intended
for developing demonstrations and instructional material relating to the
use of the MGED editor,
all user interactions with the editor can be recorded in a file,
along with an indication of the time elapsed between user actions.
This file can be  adjusted using a normal text editor to remove any errors,
or to eliminate dead time where the user stopped to think.
Once created, this session script can be replayed through the editor
at any time, either to provide a smooth ``canned'' demonstration
before a live audience, or to create a film or videotape.
\chapter{TUTORIALS ON VIEWING AND STATES}

Tutorials with illustrations are provided to give the MGED user a
step-by-step walk-through of the basic capabilities of the graphics
editor.
Standard UNIX login and logout procedures appropriate to each site
should be followed prior to
beginning and after ending the tutorials.

Each of the tutorials will use the solids contained the MGED database called
``prim.g''.
These can be obtained by making a copy of ``db/prim.g''
from the BRL-CAD Package distribution tree.  It is important to make
a copy of the database and work with that, rather than using the
supplied one.  Changes made during the editing process are written
to the database when they are {\sl accepted}.

The first tutorial shows a sample invocation dialogue.  All other
tutorials start at the first MGED prompt ({\tt mged> }). If the user
wishes to continue from one tutorial to the next without leaving MGED,
issue the {\em press reject} and {\em press reset} commands
before starting a new tutorial.
User input will be shown in an
{\em emphasized} font, and MGED output will appear in a {\tt typewriter}
font.  If the user input is shown on the same line as a prompt, the
input is literal.  If the user input is shown on a line by itself,
it is a directive, and is entered in an appropriate fashion.

The tutorials are self-contained, and if the user wishes to proceed to
the next tutorial without exiting MGED,
the RESET button should be pressed
to return to the top view, where the model XYZ axes
map to the screen XYZ axes.

The standard recovery procedure when in the middle of an editing operation
is to select REJECT edit.  Control is
returned to the viewing state, and the user can restart with the last edit (e)
command used in the tutorial.

\section{States Within the Edit Process}

In this tutorial, the user will invoke MGED on a file called ``prim.g'';
attach a {\sl display manager\/}; explore the various MGED states;
and finally, exit MGED.  A MGED database has a treelike structure.  The
leaves are the individual solids, and the other nodes are groupings
of those solids.  The solid editing functions are concerned with defining
and modifying the leaves, and the object editing functions operate
on groups, which are Boolean combinations of solids.  One useful mental
model is to envision solid editing as operating directly on a leaf and
object editing as operating on the arc connecting a pair of nodes.  The
object edit will affect everything below the selected arc (this is why
there is an additional state transition when object editing).

\section{Viewing State}

The first task is to invoke MGED.  This tutorial will assume the user
has a copy of the ``prim.g'' database in the current directory.

\noindent
{\tt \$ }{\em mged prim.g}\\
{\tt BRL-CAD Release 3.0 Graphics Editor (MGED) Compilation 82}\\
{\tt Thu Sep 22 08:08:39 EDT 1988}\\
{\tt mike@video.brl.mil:/cad/.mged.4d2}\\

\noindent
{\tt attach (nu|tek|tek4109|ps|plot|sgi)[nu]? }{\em sgi}\\
{\tt ATTACHING sgi (SGI 4d)}\\
{\tt Primitive Objects (units=mm)}\\
{\tt mged> }\\

The first three lines give information about which version of MGED is running,
when it was compiled, and who compiled it.  The next line is the display
manager attach prompt.  This prompt provides a list of available display
managers, then shows what the default will be (selected if the user answers
with a carriage return).  In this case, the Silicon Graphics 4d display
manager was selected, as is noted by the following line.
Next the title of the database and
the unit of measurement used in the database are printed,
and finally, the first prompt is issued.
At this point MGED has loaded ``prim.g''; attached the SGI display;
and is awaiting commands.  Attaching a display also causes what
is known as the MGED {\sl faceplate} to be drawn on the graphics display.

The faceplate has several features of interest.  In the upper left corner
of the display, is a box which always shows the current MGED {\sl state}.
This can be one of six states:  {\bf VIEWING}, {\bf SOL PICK},
{\bf SOL EDIT}, {\bf OBJ PICK}, {\bf OBJ PATH}, or {\bf OBJ EDIT}.

Immediately below, is the menu area.  The only menu item initially shown is
one labeled {\bf BUTTON MENU}.  This menu item toggles the display of the
button menu entries when {\sl selected} (more on selection later).

At the bottom of the display are two status lines.  The first line
contains information about the current view.
The entry labeled {\bf cent=} gives the {\sl model space} coordinates
of the dot in the center of the display.
The entry labeled {\bf sz=} reflects the current size in model units of
the {\sl viewing cube}.  The viewing cube is a mathematical construct
centered on the the dot in the center of the display.  The {\bf ang=}
display shows the current rate of rotation in each of the three axes.
The bottom line is used for several kinds of information.
In the {\bf VIEWING} state, it displays the title of the database.

The MGED viewing features are designed to allow the user to examine
models at different angles.
Preset views can be invoked at
anytime by using either the menu or the button box.
Selecting a preset view does
not change the coordinates of the primitives,
but instead changes the angle from which these primitives are
displayed.  Five standard views (top, right, front, 35/25, and 45/45) can
be obtained by using either the bottom menu on the display screen or the
control box.
Three additional views (button, left, and rear) can be obtained
by using the button box, but not by using the menu.

The normal or default viewing state is the ``top'' orientation,
with model +X pointing towards the right of the screen,
model +Y pointing towards the top of the screen,
and model +Z pointing out of the screen.
In the ``top'' view, the model and screen axes are the same.
The ``reset'' button and ``Reset Viewsize'' menu items also
result in a ``top'' view.

The following table shows the angles of rotation to obtain the other views.
\begin{tabular}{l l}
View    &            Angle of Rotation (from top) \\
 \\
Top      &                     0, 0, 0 \\
Bottom   &                   180, 0, 0 \\
Right    &                   270, 0, 0 \\
Left     &                  270, 0, 180 \\
Front    &                  270, 0, 270 \\
Rear     &                  270, 0, 90 \\
35, 25   &                  295, 0, 235 \\
\end{tabular}

\noindent
{\tt mged>\ }{\em e arb8}\\
{\tt vectorized in 0 sec}\\
{\tt mged>\ }{\em size 12}\\
{\tt mged> }\\

\mfig t1-top-vw, ``arb8'' Top View.
The {\bf e} command causes the named object(s) -- a solid named ``arb8''
in this case
-- to be displayed, and the {\bf size} command sets the size of the
viewing cube. Figure \ref{t1-top-vw} shows what the display currently
looks like.  In this view, the X-axis is to the right, the Y-axis points
up, and the Z-axis is perpendicular to (poking out of) the screen.

\noindent
{\em Twist the {\bf Y ROT} knob clockwise and back.}\\
{\em Twist the {\bf X ROT} knob counterclockwise and back.}\\

These knobs, along with the {\bf Z ROT} knob, rotate the viewing cube.
Use of the rotation
knobs allows the user to view the model from any orientation.
Turning a knob clockwise causes a rotation in the positive direction,
while turning a knob counterclockwise causes a negative rotation
(right-hand rule).  The knobs are rate based, not position based;
once a rotation has been started, it will continue until the
knob is returned to zero (or the {\bf zeroknobs} button is pressed).
Rotations are about the viewing cube (screen) axes, not the model axes.
Systems without knobs can use the {\bf knob} command.

\noindent
{\em Move the mouse (or pen) until the cursor is in the {\bf BUTTON MENU}
block and then press the middle mouse button (depress the pen).}\\

\mfig t1-rot-vw, ``arb8'' Rotated View.
Pressing the middle mouse button (or the pen) {\sl selects} something.
When the cursor is inside the menu area, a selection
causes the event described by the menu item to occur.
Selecting {\bf BUTTON MENU} causes the button menu to appear on the left
side of the screen. The {\bf BUTTON MENU} menu item is
a toggle; subsequent selection of this item will cause the button menu
to disappear.
Figure \ref{t1-rot-vw} shows the new display.

\noindent
{\em Move the cursor from the menu area to a point near the
upper left corner of the solid and select it (press the center mouse
button).}\\

In the {\bf VIEWING} state, making a selection while outside of the menu
area will move the selected point to the center of the display.  Look
carefully at the center of the display; the point just selected is now
located at the center dot. Use the {\bf center} command to reset any
translations made with the mouse.

\noindent
{\tt mged> }{\em center 0 0 0}\\
{\tt mged> }\\

From the {\bf VIEWING} state, the user will normally transition to either the
{\bf SOL PICK} or {\bf OBJ PICK} state.
The {\bf SOL PICK} state is selected by:
\begin{itemize}
\item Selecting the {\bf Solid Illum} button menu entry, or,
\item Pressing the {\bf sill} button (this button may be labeled
using some variation of ``Solid Illum''), or,
\item Typing {\bf press sill}.
\end{itemize}
Similar entries ({\bf Object Illum}) and ({\bf oill}) exist for transitioning
into the {\bf OBJ PICK} state.
In general, the {\bf press} command is the basic mechanism (type
{\bf press help} for a list of available commands).  Most of the press
commands have been mapped onto a button box if it is available,
 and some of the
most common are also mapped into the {\bf BUTTON MENU} so they can
accessed without letting go of the mouse.

\section{Solid Pick State}

\noindent
{\em Place MGED in the {\bf SOL PICK} state using one of the
above mechanisms.}\\

\mfig t1-sol-pk, MGED In Solid Pick State.
Upon entering the {\bf SOL PICK} state, the display will look similar to
Figure \ref{t1-sol-pk}.  The {\bf SOL PICK} state used to select which
of the displayed solids is to be edited.  Note that the color of the
solid has changed from red to white.  The screen is divided into as many
horizontal zones as there are solids displayed, and each zone is
assigned to one solid.  As the mouse is moved vertically through each
zone, the corresponding solid is highlighted (``illuminated'') by
drawing it in white.   In this instance, there is only one solid being
displayed, so this state is relatively uninteresting.
If the system being used has no mouse, there is no reason to enter the
{\bf SOL PICK} state.  The user will instead transition directly to
the {\bf SOL EDIT} state using the {\bf sed} command.

\noindent
{\tt mged> }{\em press reject}\\
{\tt mged> }{\em e ellg}\\
{\tt mged> }\\
{\em Press the {\bf sill} button}\\

\mfig t1-2s-pk, MGED In Solid Pick with Two Solids.
Note that the first action taken was to {\sl reject} the edit.  Any time MGED
is not in the {\bf VIEWING} state, a {\sl reject} command (via
{\bf press}, button, or mouse) discards all editing changes accumulated
since the last transition out of the {\bf VIEWING} state, and places
MGED in the {\bf VIEWING} state.
The display should now look similar to Figure \ref{t1-2s-pk}.
Notice that one solid is white and
the name of that solid is displayed in the upper left corner of the
display, as well as in the bottom status line. The solid to be edited is
selected by moving the mouse up and down until the zone corresponding to
the desired solid is reached. Once the appropriate zone is reached, select it.
This selects a solid, and once a solid is selected,
MGED enters the {\bf SOL EDIT} state.

\section{Solid Edit State}

\noindent
{\tt mged> }{\em d ellg}\\
{\tt mged> }\\
{\em Select the solid called ``arb8''.}\\

\mfig t1-sol-ed, Solid Edit State.
The {\bf d} commmand removes something from the display.  In this
case, the solid ``ellg'' was removed to reduce clutter.
The display should now look like Figure \ref{t1-sol-ed}.
When MGED enters the solid edit state, the following occurs:
\begin{itemize}
\item The solid selected for editing remains illuminated,
\item The solid is labeled,
\item The coordinates (or dimensions) associated with the labels,
and other information is displayed to the right of the menu area,.
\item If the solid is a member of one or more groups, a similar set
of coordinates called the {\sl PATH} is displayed immediately below
the the first set of coordinates,
\item The {\bf *SOLID EDIT*} menu is displayed, and,
\item A solid specific edit menu (in this case the {\bf ARB MENU})
is displayed.
\end{itemize}

The {\bf *SOLID EDIT*} menu provides access to generic operations (translation, rotation
and scaling) common to all solids.
The solid specific edit menu is a list of solid type specific editing operations.
Selecting one of the solid specific edit menus causes a submenu with solid type specific
choices to be displayed.  To remove this submenu, select either the
{\bf RETURN} item in the submenu, or the {\bf edit menu} item in the
{\bf *SOLID EDIT*} menu.

It is in this state that the solid is altered to meet the modeler's
requirements. The shape, positioning, and orientation of the solid is
changed using numeric keyboard input, positioning of the mouse, or by
use of the knobs.  Once the solid has been altered, the edit is
either accepted or rejected.  Accepting the edit causes all changes
made to be written to the database; rejecting the edit ``throws them
away''. Either operation will terminate the edit session and return MGED
to the {\bf VIEWING} state.

\noindent
{\em Reject the edit.}\\

\section{Object Pick State}

\noindent
{\em Place MGED in the {\bf OBJ PICK} state.}\\

\mfig t1-obj-pk, Object Pick State.
Figure \ref{t1-obj-pk} shows what the display looks like when in the
{\bf OBJ PICK} state. As with the {\bf SOL PICK} state, a single solid is
selected.  This solid becomes the reference solid for the object edit.
In the {\bf OBJ PICK} state, the solid will be shown
as a member of one or more objects.  Less obvious is the fact that the
local axes associated with the selected solid are the axes used for the
entire object during the object edit.

\section{Object Path State}

\noindent
{\em Select ``arb8''.}\\

\mfig t1-obj-ph, Object Path Selection State.
MGED transitions into the {\bf OBJ PATH} state once a solid has been
picked from {\bf OBJ PICK}. Figure \ref{t1-obj-ph} is the display in
the {\bf OBJ PATH} state.  When in this state the extent of the editing
operation is set.  Everything below the editing point is affected by the
edit.  The editing point is shown by the {\sl MATRIX} label in the
display.  It is shown as {\bf [MATRIX]} in the upper left part of the
display and as {\bf \_\_MATRIX\_\_} in the second status line.  The editing
point is chosen with the same mechanism used by {\bf SOL PICK} and
{\bf OBJ PICK}.  This time, there is one horizontal zone for each node in
the path between the root and selected leaf.  Moving the mouse up and down
moves the editing point up and down in the tree.  Once again, having a
simple database and only one object in view makes for a relatively
uninteresting situation.

\section{Object Edit State}

\noindent
{\em Select the editing point above ``arb8''.}\\

\mfig t1-obj-ed, Object Edit State.
MGED is now in the {\bf OBJ EDIT} state and the display should look like
Figure \ref{t1-obj-ed}.
When MGED enters the object edit state, the following occurs:
\begin{itemize}
\item The reference solid remains illuminated,
\item The reference solid is labeled,
\item The information associated with the labels is displayed to the right
of the menu area, and
\item The {\bf *OBJ EDIT*} menu is displayed.
\end{itemize}

The {\bf OBJ EDIT} state is used to modify the
Homogeneous Transform Matrix selected during the {\bf OBJ PATH} state.
Permissible operations include uniform and affine scaling of the objects,
as well as translation and rotation.
As with the {\bf SOL EDIT} state, MGED accepts changes entered using
the keyboard, mouse or knobs.

This concludes the first tutorial.  Examples of the appearance of MGED
in each of the six states have been given, along with some idea of what
each of the states is used for.  All that remains is to reject the current
edit, and exit MGED.  Strictly speaking the {\bf q} command could be entered
directly, but doing so, can become a dangerous habit.

\noindent
{\em Select {\bf REJECT Edit} using the mouse.}\\
{\em Press the {\bf reject} button.}\\
{\tt mged> }{\em d arb8}\\
{\tt mged> }{\em q}\\
{\tt \$ }\\

\section{Editing in the Plane of the Screen}
\mfig plane-top1, A Top View of the Coordinate Axes.

When MGED is in a ``translate'' mode within an edit state,
the plane of the mouse or data tablet is mapped to
the plane of the screen, to permit moving objects in a
controlled way in two of the three available dimensions.
The orientation of the plane of the screen is determined by the
currently selected view.
In most circumstances, users will find that repositioning objects
is easiest when the the plane of the screen is oriented in an
axis-aligned view.  This is most easily accomplished by utilizing
one of the preset views.
For this exercise, obtain a copy of the {\em axis.g} database,
and run MGED, eg:

\noindent{\tt
\$ cp cad/db/axis.g . \\
\$ mged axis.g \\
BRL-CAD Release 3.0 Graphics Editor (MGED) Compilation 82 \\
    Thu Sep 22 08:08:39 EDT 1988 \\
    mikel@video.br:/cad/.mged.4d2 \\
\\
attach (nu|tek|tek4109|ps|plot|sgi)[nu]? {\em sgi} \\
ATTACHING sgi (SGI 4d) \\
X,Y,Z Coordinate Axis  (units=none) \\
mged> {\em e axis} \\
vectorized in 0 sec \\
{\em Select ``Top'' in the Button menu} \\
mged> \\
}

\subsection{Top View}
\mfig plane-top2, Translating from the Top View.

The top view is the default view.  The orientation of the axes
is shown in Figure \ref{plane-top1}.
The surface of the viewing screen and the graphics tablet is the XY plane.
Edit changes using the graphics tablet will affect only the X and Y
coordinates of the primitive.

\noindent{\tt
mged> {\em sed x} \\
{\em Select ``Translate'' in the Solid Edit menu} \\
mged>
}

Select different points on the tablet with the mouse, each time
pressing the middle mouse button.
Notice how the X and Y coordinates of the V vector change,
but the Z coordinate does not.
An example of this is shown in Figure \ref{plane-top2};
compare the values of V with those in Figure \ref{plane-top1}.

{\em Select ``REJECT Edit'' in the Button menu}

\subsection{Bottom View}
\mfig plane-bot1, A Bottom View of the Coordinate Axes.
\mfig plane-bot2, Translating from the Bottom View.

\noindent{\tt
mged> {\em press bottom} \\
mged> {\em sed x} \\
{\em Select ``Translate'' in the Solid Edit menu} \\
mged>
}

The {\em press bottom} command selects the bottom view of the
model, and the new configuration of the axes can be seen in
Figure \ref{plane-bot1}.
The surface of the viewing screen and the mouse or tablet
are still in the XY plane.
Edit changes using the graphics tablet will affect only the X and Y
components of the solid.
Select different points on the tablet with the mouse and notice the
changes in the coordinates;
compare the values of V with those in Figure \ref{plane-bot2}.

{\em Select ``REJECT Edit'' in the Button menu}

\subsection{Right View}
\mfig plane-right1, A Right View of the Coordinate Axes.
\mfig plane-right2, Translating from the Right View.

\noindent{\tt
{\em Select ``Right'' in the Button menu} \\
mged> {\em sed x} \\
{\em Select ``Translate'' in the Solid Edit menu} \\
mged>
}

The right hand view has been selected. Model +X still proceeds to the right,
but now model +Z is at the top of the screen, and model +Y is
pointing out of the screen.
This new configuration is depicted in Figure \ref{plane-right1}.
The surface of the viewing screen and the graphics tablet is the XZ plane.
Edit changes using the graphics tablet will affect only the X and Z
coordinates of the solid.
Select different points on the tablet with the mouse and notice the
changes in the V coordinates;  only the X and Z components change,
as in Figure \ref{plane-right2}.

{\em Select ``REJECT Edit'' in the Button menu}

\subsection{Front View}
\mfig plane-front1, A Front View of the Coordinate Axes.
\mfig plane-front2, Translating from the Front View.

\noindent{\tt
{\em Select ``Right'' in the Button menu} \\
mged> {\em sed x} \\
{\em Select ``Translate'' in the Solid Edit menu} \\
mged>
}

The front view has been selected.  Model +X points out of the screen,
model +Y points to the right, and model +Z points towards the top
of the screen, as shown in Figure \ref{plane-front1},
which has been slightly rotated off the preset view to improve
the legibility of the axis labels.
The surface of the viewing screen and the graphics tablet is the YZ
plane.  Edit changes will affect only the Y and Z
coordinates of the primitive, as shown in Figure \ref{plane-front2}.
Select different points on the tablet with the mouse and notice the
changes in the coordinates.

{\em Select ``REJECT Edit'' in the Button menu}

\subsection{35, 25 View}
\mfig plane-35a, An Oblique 35,25 View of the Coordinate Axes.
\mfig plane-35b, Translating in the 35,25 View.

\noindent{\tt
{\em Select ``35,25'' in the Button menu} \\
mged> {\em sed x} \\
{\em Select ``Translate'' in the Solid Edit menu} \\
mged>
}

Figure \ref{plane-35a} is the 35,25 view of the axes model.
The axes are no longer
parallel or perpendicular to the viewing surface or to the graphics tablet.
Edit changes using the graphics tablet will affect all of the coordinates of
the solid, in a manner that is visually intuitive when the solid
is moved around on the screen.
Select different points on the tablet with the mouse and notice the
changes in the coordinates, such as in Figure \ref{plane-35b}.
Note how all three components of the V vector have changed.

{\em Select ``REJECT Edit'' in the Button menu}
\chapter{TUTORIALS ON EDITING SOLIDS}

The Solid Editing state of MGED is used to modify the fundamental
parameters of an individual solid.
Each solid must be modified individually.

\section{Solid Edit: A Six-Sided Polyhedron}
\mfig es8-top, Top View of ARB8.

This section illustrates the use of commands while in
SOL EDIT state to alter the
shape of a polyhedron with six sides and 8 faces (ARB8).

\noindent{\tt
\$ {\em mged es.g} \\
BRL-CAD Release 3.0 Graphics Editor (MGED) Compilation 82 \\
    Thu Sep 22 08:08:39 EDT 1988 \\
    mike@video.brl:/cad/.mged.4d2 \\
 \\
es.g: No such file or directory \\
Create new database (y|n)[n]? {\em y} \\
attach (nu|tek|tek4109|ps|plot|sgi)[nu]? {\em sgi} \\
ATTACHING sgi (SGI 4d) \\
Untitled MGED Database (units=mm) \\
mged> {\em in arb8 rpp -1 1 -1 1 -1 1} \\
mged> {\em size 10} \\
mged>
}

Figure \ref{es8-top}
is a top view of the six-sided polyhedron.
The Z-axis perpendicular to the viewing screen.
Next, the view is rotated so that all sides can be seen.

\noindent{\tt
mged> {\em Twist ROTY knob clockwise and restore} \\
mged> {\em Twist ROTX knob counter-clockwise and restore} \\
mged>
}

\mfig es8-rot, A Rotated View of the ARB8.
Figure \ref{es8-rot} shows a better perspective of the solid.

The next step in this tutorial is to transfer to the solid edit state.
This can be accomplished in two ways:  either by going through
the SOL PICK state (``illuminate mode'') or by direct transfer via
keyboard command.
Using illuminate mode is better when the name of the solid to be
edited may not be known, while the keyboard command is generally
preferred when the name of the solid is known.

\noindent{\tt
mged> {\em Select the ``Solid Illum'' entry in the button menu} \\
mged> {\em Move the mouse out of the menu area} \\
mged> {\em Click the mouse to enter SOL EDIT state} \\
mged>
}

To perform a direct transfer from the viewing state to the solid edit state
using a keyboard command, enter:

\noindent{\tt
mged> {\em sed arb8} \\
mged>
}

\mfig es8-sed, An ARB8 in Solid Edit State.
Figure \ref{es8-sed} corresponds to the view on the display.
The ARB8 MENU is unique to the ARB primitive,
and lists operations that can only be performed on an ARB solid.
The items in the ARB8 MENU are
selected by using the mouse.
Each of the other types of solids have a
similar unique menu.
When one of these items is selected, the top level ARB8 MENU disappears,
to be replaced with the indicated subordinate menu.
The top-level menu reappears when either
the ``edit menu'' item in the SOLID EDIT menu is selected,
or the ``RETURN'' item in the subordinate menu is selected.

The  SOLID EDIT  menu applies to all
solids when in the SOL EDIT state.
The items in the  SOLID EDIT  menu are selected
by either using the mouse or by depressing the appropriate button on the
button box.
When any of the SOLID EDIT menu items are selected
(eg, ``Rotate'', ``Translate'', ``Scale''), the solid-specific menu
disappears.
Th top-level solid-specific menu reappears when
the ``edit menu'' item in the SOLID EDIT menu is selected.

The {\em p [params]} command is used to
make precise changes, where the numeric value of the parameter being
edited is know.
Values for all parameters in the ARB8
and SOLID EDIT menus can be specified by using the {\em p} command,
or by pointing and clicking with the mouse.

\mfig es8-tr0, Translating ARB8 Point 1 to the Origin.
\subsection{Translate Operation}

\noindent{\tt
mged> {\em Select the ``Translate'' entry in the solid edit menu} \\
mged> {\em p 0 0 0} \\
mged>
}

Point 1 of the primitive is moved to point 0 0 0,
as shown in Figure \ref{es8-tr0}.

The translate solid operation is selected
by either picking ``Translate'' on the solid edit menu
with the mouse, by depressing the solid edit button on the button box,
or by entering the {\em press sed} command.
Parameters to the translate solid operation
are of the form {\em p a b c}
where {\em a}, {\em b}, and {\em c} are the new coordinates
of point 1 in the solid.
The other points are transferred to keep the same position
relative to point 1.
The general form of the new coordinates for point is

\begin{center}
\begin{verbatim}
x ' = x + a - x
y ' = y + b - y
Z ' = Z + c - Z
\end{verbatim}
\end{center}

The command

\noindent{\tt
mged> {\em p 1 -1 -1} \\
mged>
}

can be used to restore the primitive to the original position.

\subsection{Rotate Operation}
\mfig es8-xrot, Solid Edit Rotation of 45 Degrees about X.

The rotate operation is initiated by either selecting Rotate on the menu
screen with the mouse,
by depressing the Solid Rotate button on the button box,
or by entering the {\em press srot} command on the keyboard.

\noindent{\tt
mged> {\em Select the ``Rotate'' entry in the solid edit menu} \\
mged> {\em p 45 0 0} \\
mged>
}

The parameter {\em p} command is used to make precise rotation changes.
The
command is entered in the form {\em p a b c} where
{\em a}, {\em b}, and {\em c} are the angles
(in degrees) of rotation about the x, y, and z axes respectively.  Point 1,
the vertex, remains fixed, and the solid is rotated about this point.  A
positive angle of rotation is counter-clockwise when viewed in the positive
direction along an axis.

The order of rotation is not commutative.
Rotation takes place about the
Z axis, Y axis, and X axis in that order.
Figure \ref{es8-xrot} shows the rotation of 45 degrees about the X axis.

%The following is the formula for moving point (x,y,z) through angles
%(a, b, c) to point (x', y', z')
%\begin{verbatim}
%[X'] [cos b cos c                     -cos b sin c        sin b        ][X]
%[Y']=[sin a sin b cos c +cos a sin b   cos a cos c -sin a sin b sin c -sin a cos b ][Y]
%[Z'] [sin a sin c -cos a sin b cos c   cos a sin b sin c + sin a cos c cos a cos b ][Z]
%\end{verbatim}

The values entered after the p are absolute - the rotations are applied to
the primitive as it existed when solid rotation was first selected.  Thus
entering {\em p 0 0 0} will undo any rotations
performed since solid rotation was begun.

\mfig es8-yrot, Solid Edit Rotation of 45 Degrees about Y.
\mfig es8-zrot, Solid Edit Rotation of 45 Degrees about Z.

\noindent{\tt
mged> {\em p 0 45 0} \\
mged>
}

Figure \ref{es8-yrot} displays the solid
after it has been rotated about the Y axis.

\noindent{\tt
mged> {\em p 0 0 45} \\
mged>
}

Figure \ref{es8-zrot} displays the solid
after it has been rotated about the Z axis.

\noindent{\tt
mged> {\em p 0 0 0} \\
mged>
}

This restores the original orientation of the solid.

\subsection{Scale Operation}
\mfig es8-scale, ARB8 Scale Increased by 2X.

\noindent{\tt
mged> {\em Select the ``Scale'' entry in the solid edit menu} \\
mged> {\em p 2} \\
mged>
}

Figure \ref{es8-scale} corresponds to the view that is shown on the display.
The scale operation may be initiated by either selecting
the Scale entry  on the menu with the mouse,
by depressing the Solid Scale button,
or by entering {\em press sscale} on the keyboard.
The parameter command {\em p n} is
used to enter a precise scale factors, where {\em n} is
the scale factor.
The coordinates of point 1 remain the same.  The distances
from point 1 to the other points are multiplied by the scale value {\em n}.
The general equations for the transformation from point p to p' are

\begin{verbatim}
    x'[i] = x[i] + n (x[i] - x[1] )
    y'[i] = y[i] + n (y[i] - y[1] )   i != 1
    z'[i] = z[i] + n (z[i] - z[1] )
\end{verbatim}

The size of the primitive may be changed by depressing the mouse at
different positions.  When the mouse is clicked, the
edited primitive is scaled about point 1 (the key point)
by an amount proportional
to the distance the mouse is from the center of the screen.  If the mouse
is above the center of the screen, the edited primitive will become larger.
If the
mouse is below the center, the primitive will become smaller.

The value of {\em n} entered is applied to the primitive as it existed when the
solid scale state was entered.

Entering {\em p 1} will return the primitive
to the size it had when the solid scale operation first started.

\noindent{\tt
mged> {\em p 1} \\
mged>
}

\mfig es8-edge1, ARB8 Edge 15 Moved Through (9, -2, -2).
\mfig es8-edge2, ARB8 Edge 12 Moved Through (2, 5, -2).
\mfig es8-edge3, ARB8 Edge 14 Moved Through (2, -2, 7).
\subsection{Moving Edges}

The move edge command permits the moving of a line or edge so that the
line passes through the selected point.

\noindent{\tt
mged> {\em Select the ``edit menu'' entry in the solid edit menu} \\
mged> {\em Select the ``move edges'' entry in the ARB menu} \\
mged> {\em Select the ``move edge 15'' entry in the ARB8 edges menu} \\
mged> {\em p 9 -2 -2} \\
mged>
}

The edge 15 is moved so that it passes through the point (9, -2, -2).  The
coordinates of the new points 1 and 5 are the intersection of the new edge
with the planes 234 and 678.  Since both the old edge and new edge 15 are
parallel to the X axis,
the X coordinate of the point given by the {\em p} command
has no meaning.  The X coordinates for points 1 and 5 are not changed.
See Figure \ref{es8-edge1}.

\noindent{\tt
mged> {\em p 9 -1 -1} \\
mged>
}

This restores the original shape.
The choice of ``9'' for the X coordinate was arbitrary.

\noindent{\tt
mged> {\em Select the ``move edge 12'' entry in the ARB8 edges menu} \\
mged> {\em p 2 5 -2} \\
mged>
}

The edge 12 is parallel to the Y axis.  This command moves the points 1
and 2 so that their X and Z coordinates are 2 and -2.
See Figure \ref{es8-edge2}.
The Y coordinates are not changed.

To restore the view, enter:

\noindent{\tt
mged> {\em p 1 5 -1} \\
mged>
}

The choice of ``5'' for the Y coordinate was arbitrary.

\noindent{\tt
mged> {\em Select the ``move edge 14'' entry in the ARB8 edges menu} \\
mged> {\em p 2 -2 7} \\
mged>
}

The edge 14 is parallel to the Z axis.
This command moves the points 1
and 4 so that their X and Y coordinates are 2 and -2.
See Figure \ref{es8-edge3}.
The Z coordinates are not changed.

\noindent{\tt
mged> {\em p 1 -1 7} \\
mged>
}

This restores the original shape.
The choice of ``7'' for the Z coordinate was arbitrary.

\subsection{Extrude Command}
\mfig es8-ex1, ARB8 Rear Face Extruded 5 Units in -Z.
\mfig es8-ex2, ARB8 Rear Face Extruded 3 Units in +Z.

The extrude command is used to move the opposite surface a distance from
the specified surface or plate.
This command can only be used when an ARB solid is in
solid edit state.

\noindent{\tt
mged> {\em extrude 1265 5} \\
mged>
}

In Figure \ref{es8-ex1},
the plane opposite surface whose points are 1, 2, 6, and 5
is moved to a distance of 5 in the positive Z direction from plane 1265.  Note
that the points were selected counter-clockwise when viewed in the positive
direction along the Z axis.

\noindent{\tt
mged> {\em extrude 1562 3} \\
mged>
}

In Figure \ref{es8-ex2},
the plane opposite surface 1562 is moved to a distance of 3
in the negative Z direction from 1562.  Note that the points were selected
clockwise when viewed in the positive direction along the Z axis.

\noindent{\tt
mged> {\em extrude 1265 2} \\
mged>
}

This restores the original shape of this solid.

To return control to the VIEWING state, select the ``REJECT Edit''
item on the button menu, press the ``reject'' button on the button box,
or enter the command {\em press reject} on the keyboard.
Then, enter

\noindent{\tt
mged> {\em d arb8} \\
mged>
}

to drop the ARB8 from view.

\section{Solid Edit:  A Five-Sided Polyhedron}
\mfig es5-top, Top View of an ARB5.
\mfig es5-rot, A Rotated View of the ARB5.
\mfig es5-sed, The ARB5 in Solid Edit State.

This tutorial illustrates the application of the SOL EDIT state to the ARB5
solid.
In this tutorial, the view is modified by using the rotation
knobs so that all sides can be seen.

\noindent{\tt
mged> {\em size 6} \\
mged> {\em in arb5 arb5} \\
Enter X, Y, Z for point 1: {\em 0 0 0} \\
Enter X, Y, Z for point 2: {\em 0 0 1} \\
Enter X, Y, Z for point 3: {\em 0 1 1} \\
Enter X, Y, Z for point 4: {\em 0 1 0} \\
Enter X, Y, Z for point 5: {\em -1 .5 .5} \\
mged>
}

Figure \ref{es5-top} is the display of arb5
in the VIEWING state that is seen when
the solid is first created.
In this view, the Z axis is perpendicular to the viewing screen.

\noindent{\tt
mged> {\em Twist ROTY knob clockwise and restore} \\
mged> {\em Twist ROTX knob counter-clockwise and restore} \\
mged>
}

These actions generate a view shown in Figure \ref{es5-rot}
that shows all sides.

\noindent{\tt
mged> {\em Select the ``Solid Illum'' entry in the button menu} \\
mged> {\em Move the mouse out of the menu area} \\
mged> {\em Click the mouse to enter SOL EDIT state} \\
mged>
}

These actions will place MGED in the SOL EDIT state
as shown in Figure \ref{es5-sed}.

\subsection{Translate Operation}
\mfig es5-tr, Translating an ARB5.

\noindent{\tt
mged> {\em Select the ``Translate'' entry in the solid edit menu} \\
mged> {\em p -1 -1 1} \\
mged>
}

This command cause point 1 to be moved to coordinates (-1, -1, 1) and the
other points are moved so that they keep the same relative position to point
1.  See Figure \ref{es5-tr}.

Enter this command to restore the solid to it's original location:

\noindent{\tt
mged> {\em p 0 0 0} \\
mged>
}

\subsection{Rotate Operation}
\mfig es5-xrot, ARB5 Solid Edit Rotation about X.

\noindent{\tt
mged> {\em Select the ``Rotate'' entry in the solid edit menu} \\
mged> {\em p 45 0 0} \\
mged>
}

Figure \ref{es5-xrot} shows a rotation of 45 degrees
about an axis parallel to the X axis.
The rotate command is entered in the form {\em p a b c}
where {\em a}, {\em b}, and {\em c} are the angles
(in degrees) of rotation about the x, y, and z axes and intersect at point 1.
All rotation takes place about point 1.

\noindent{\tt
mged> {\em p 0 0 0} \\
mged>
}

This restores the original orientation of the solid.

\subsection{Scale Operation}
\mfig es5-scale, ARB5 Scale Increased by 2X.

\noindent{\tt
mged> {\em Select the ``Scale'' entry in the solid edit menu} \\
mged> {\em p 2} \\
mged>
}

Figure \ref{es5-scale} shows the change in the primitive.
Point 1 remains the same
and the distances of the other points from point 1 is multiplied by 2.

Entering {\em p 1} will return the primitive
to the size it had when the solid scale operation first started.

\noindent{\tt
mged> {\em p 1} \\
mged>
}

\subsection{Move Edge Command}
\mfig es5-edge1, ARB5 Edge 14 Moved Through (1, 1, 1).
\mfig es5-edge2, ARB5 Point 5 Moved to (-1.5, 1, 1).
\mfig es5-edge3, ARB5 Edge 45 Moved Through (-1.5 1 1).
\mfig es5-edge4, ARB5 Edge 12 Moved Through (2, 1, 2).

\noindent{\tt
mged> {\em Select the ``edit menu'' entry in the solid edit menu} \\
mged> {\em Select the ``move edges'' entry in the ARB menu} \\
mged> {\em Select the ``move edge 14'' entry in the ARB8 edges menu} \\
mged> {\em p 1 1 1} \\
mged>
}

The edge 14 is moved so that it moves through the point (1, 1, 1).
Note that this point is the mid-point between points 1 and 4.
See Figure \ref{es5-edge1}.

\noindent{\tt
mged> {\em p 0 2 0} \\
mged>
}

This restores the original shape.

\noindent{\tt
mged> {\em Select the ``move point 5'' entry in the ARB5 edges menu} \\
mged> {\em p -1.5 1 1} \\
mged>
}

The point 5 is moved to location -1.5, 1, 1.  See Figure \ref{es5-edge2}.

\noindent{\tt
mged> {\em p -1 .5 .5} \\
mged>
}

will restore the original shape.

\noindent{\tt
mged> {\em Select the ``move edge 45'' entry in the ARB5 edges menu} \\
mged> {\em p -1.5 1 1} \\
mged>
}

In Figure \ref{es5-edge3}, the edge 45 is moved
so that it passes through the point (-1.5, 1, 1).
Note that this point lies between the points 4 and 5.

\noindent{\tt
mged> {\em p -1 .5 .5} \\
mged>
}

This restores the original shape.

\noindent{\tt
mged> {\em Select the ``move edge 12'' entry in the ARB5 edges menu} \\
mged> {\em p 2 1 2} \\
mged>
}

In Figure \ref{es5-edge4},
the edge 12 is moved so that it passes through the point (2, 1, 2).
Note that the coordinates correspond to point 2.

The movement of the edges may yield unpredictable results when the edges
are not parallel to one of the axes.


To return control to the VIEWING state, select the ``REJECT Edit''
item on the button menu, press the ``reject'' button on the button box,
or enter the command {\em press reject} on the keyboard.
Then, enter

\noindent{\tt
mged> {\em d arb5} \\
mged>
}

to drop the ARB5 from view.

\section{Solid Edit: Alter a Cylinder}
\mfig esc-top, Top View of a Cylinder.
\mfig esc-rot, A Rotated View of the Cylinder.
\mfig esc-sed, A Cylinder in Solid Edit State.

This tutorial illustrates the application of the SOL EDIT state to
cylinder solids.

\noindent{\tt
mged> {\em size 12} \\
mged> {\em in cyl rcc} \\
Enter X, Y, Z of vertex: {\em 0 0 0} \\
Enter X, Y, Z of height (H) vector: {\em 2 0 0} \\
Enter radius: {\em 1}
mged>
}

Figure \ref{esc-top} is the display of the cylinder solid
when viewed from the top.
Since the
Z axis is perpendicular to the viewing screen, a view of all sides cannot be
seen.

\noindent{\tt
mged> {\em Twist ROTY knob clockwise and restore} \\
mged> {\em Twist ROTX knob counter-clockwise and restore} \\
mged>
}

These actions generate a view, Figure \ref{esc-rot}, that shows all sides.

\noindent{\tt
mged> {\em Select the ``Solid Illum'' entry in the button menu} \\
mged> {\em Move the mouse out of the menu area} \\
mged> {\em Click the mouse to enter SOL EDIT state} \\
mged>
}

Figure \ref{esc-sed} is the view that displays the menu
for the SOL EDIT state.
The
point V is at the origin (0,0,0) in this example and is in the middle of the
circle that contains points A and B.  H is the point of the center of the
circle that contains points C and D.  The coordinates of H are the coordinates
of the vector from V to H and represent the relative position of H to V.  Mag
is the magnitude of these vectors and is represented by the formula
\begin{center}
\begin{verbatim}
Mag = sqrt( x + y + z )
\end{verbatim}
\end{center}

``H dir cos'' are the direction cosines of the vector H which
is perpendicular to plane of the points A, B, and V.  The coordinates of A
are the coordinates of the vector from V through A.  Mag is the magnitude of
the vectors from V to A.  The coordinates of B are the coordinates of the
vectors from V through B.  Mag is the magnitude of the vector from V to B.
The values for c and d are the magnitudes of the vectors from the tip of
vector H to the points C
and D respectively.
``A x B dir cos'' represents the direction cosines of the
vector ``A x B''.

\subsection{Translate Operation}
\mfig esc-tr, Translating Cylinder Vertex to (1, 1, 1).

\noindent{\tt
mged> {\em Select the ``Translate'' entry in the solid edit menu} \\
mged> {\em p 1 1 1} \\
mged>
}

The location of the vertex point V is moved to (1, 1, 1).
The locations of the other
points relative to V remains the same.  See Figure \ref{esc-tr}.

Move the mouse anywhere on the screen (outside the menu area), and click.
Notice
that the cylinder is moved so that V is placed at this location, and the
coordinates of the other points remain the same relative to V.

\noindent{\tt
mged> {\em p 0 0 0} \\
mged>
}

This restores the solid to the original location.

\subsection{Rotate Operation}
\mfig esc-xrot, Solid Edit Rotation of 45 Degrees about X.
\mfig esc-yrot, Solid Edit Rotation of 45 Degrees about Y.
\mfig esc-zrot, Solid Edit Rotation of 45 Degrees about Z.

\noindent{\tt
mged> {\em Select the ``Rotate'' entry in the solid edit menu} \\
mged> {\em p 45 0 0} \\
mged>
}

When viewing in the positive X direction, the cylinder is rotated counter-
clockwise 45 degrees about an axis through point V parallel to the x axis.
See Figure \ref{esc-xrot}.

\noindent{\tt
mged> {\em p 0 45 0} \\
mged>
}

When viewing in the positive Y direction, the cylinder is rotated counter-
clockwise 45 degrees about an axis through point V parallel to the Y axis.
See Figure \ref{esc-yrot}.

\noindent{\tt
mged> {\em p 0 0 45} \\
mged>
}

When viewing in the positive Z direction, the cylinder is rotated counter-
clockwise 45 degrees about an axis through point V parallel to the Z axis.
See Figure \ref{esc-zrot}.

The command

\noindent{\tt
mged> {\em p 0 0 0} \\
mged>
}

will restore the cylinder to the original orientation.

\subsection{Scale Operation}
\mfig esc-scale, Cylinder Scale Increased by 1.5X.

\noindent{\tt
mged> {\em Select the ``Scale'' entry in the solid edit menu} \\
mged> {\em p 1.5} \\
mged>
}

The point V remains fixed, the distance H between the two end-plate ellipses
is multiplied by 1.5.
See Figure \ref{esc-scale}.

The command

\noindent{\tt
mged> {\em p 1} \\
mged>
}

restores the original scale.

\subsection{Scale H Command}
\mfig esc-sh, Cylinder Scale H Vector.

\noindent{\tt
mged> {\em Select the ``edit menu'' entry in the solid edit menu} \\
mged> {\em Select the ``scale H'' entry in the TGC menu} \\
mged> {\em p 1} \\
mged>
}

The magnitude of the vector H is reduced from 2 to 1.
See Figure \ref{esc-sh}.  The command

\noindent{\tt
mged> {\em p 2} \\
mged>
}

will restore the original shape.

\subsection{Scale A Command}
\mfig esc-sa, Cylinder Scale A Vector.

\noindent{\tt
mged> {\em Select the ``scale A'' entry in the TGC menu} \\
mged> {\em p 2} \\
mged>
}

The magnitude of the vector through point A is increased to 2, ie,
the length of the axis of the ellipse through point A is set equal to p.
See Figure \ref{esc-sa}.  The command

\noindent{\tt
mged> {\em p 1} \\
mged>
}

will restore the original shape.

\subsection{Scale B Command}
\mfig esc-sb, Cylinder Scale B Vector.

\noindent{\tt
mged> {\em Select the ``scale B'' entry in the TGC menu} \\
mged> {\em p 2} \\
mged>
}

The magnitude of the vector through point B is increased to 2, ie,
the length of the axis of the ellipse through point B is set equal to p.
See Figure \ref{esc-sb}.  The command

\noindent{\tt
mged> {\em p 1} \\
mged>
}

will restore the original shape.

\subsection{Scale C Command}
\mfig esc-sc, Cylinder Scale C Vector.

\noindent{\tt
mged> {\em Select the ``scale C'' entry in the TGC menu} \\
mged> {\em p 2} \\
mged>
}

The magnitude of the vector through point c is increased to the value of
p.  The length of the axis of the ellipse through point c is set equal to the
value of p.  See Figure \ref{esc-sc}.  The command

\noindent{\tt
mged> {\em p 1} \\
mged>
}

will restore the original shape.

\subsection{Scale D Command}
\mfig esc-sd, Cylinder Scale D Vector.

\noindent{\tt
mged> {\em Select the ``scale D'' entry in the TGC menu} \\
mged> {\em p 2} \\
mged>
}

The magnitude of the vector through point D is changed to the value of p.
The length of the axis of the ellipse through point D is set equal to the
value of p.  See Figure \ref{esc-sd}.  The command

\noindent{\tt
mged> {\em p 1} \\
mged>
}

will restore the original shape.

The scale H, A, B, C, and D commands provide for setting the magnitude
equal to the value entered by the {\em p} command.
The solid edit {\bf scale} operation provides
for multiplying {\bf all} the vectors by the value
entered by the {\em p} command.

\subsection{Move End H Command}
\mfig esc-mh, Cylinder Move End of H.

\noindent{\tt
mged> {\em Select the ``move end H'' entry in the TGC menu} \\
mged> {\em p 3} \\
mged>
}

The length of the vector H is changed to the value of p.
See Figure \ref{esc-mh}.  The command

\noindent{\tt
mged> {\em p 2} \\
mged>
}

will restore the original shape.

\subsection{Move End H (rt) Command}
\mfig esc-mhrt, Cylinder Move End of H \& Rotate.

\noindent{\tt
mged> {\em Select the ``move end H(rt)'' entry in the TGC menu} \\
mged> {\em p 3} \\
mged>
}

This command is similar to the ``move end H'' command except the vector
through point A is rotated so its direction is in the -Y direction.
See Figure \ref{esc-mhrt}.  The command

\noindent{\tt
mged> {\em p 2} \\
mged>
}

will restore the original shape, but not the original orientation.

To return control to the VIEWING state, select the ``REJECT Edit''
item on the button menu, press the ``reject'' button on the button box,
or enter the command {\em press reject} on the keyboard.
Then, enter

\noindent{\tt
mged> {\em d cyl} \\
mged>
}

to drop the cylinder from view.

\section{Solid Edit: Alter Ellipsoid}
\mfig ese-top, Top View of an Ellipsoid.
\mfig ese-sed, An Ellipsoid in Solid Edit State.

This tutorial illustrates the application of the SOL EDIT state to the
ellipsoid primitive.

\noindent{\tt
mged> press reset
mged> {\em size 6} \\
mged> {\em in ell ellg} \\
Enter X, Y, Z of vertex: {\em 0 0 0} \\
Enter X, Y, Z of vector A: {\em 1 0 0} \\
Enter X, Y, Z of vector B: {\em 0 .3536 -0.3536} \\
Enter X, Y, Z of vector C: {\em 0 .3536  0.3536} \\
mged>
}

Figure \ref{ese-top} is the display of the primitive in the viewing state.
Since the
Z axis is perpendicular to the viewing screen, a view of all sides cannot be
seen.

\noindent{\tt
mged> {\em Twist ROTY knob clockwise and restore} \\
mged> {\em Twist ROTX knob counter-clockwise and restore} \\
mged>
}

These actions generate a view that shows all sides.

\noindent{\tt
mged> {\em Select the ``Solid Illum'' entry in the button menu} \\
mged> {\em Move the mouse out of the menu area} \\
mged> {\em Click the mouse to enter SOL EDIT state} \\
mged>
}

The display will be changed from the VIEWING MODE through the SOL PICK to
the SOL EDIT state.  Figure \ref{ese-sed} is the view that is displayed.

The coordinates of the points A, B, C, are given by the product of the
magnitude of the vector and the cosine of X, Y, and Z direction cosines.  In
the display, the coordinates are:

\begin{center}
\begin{verbatim}
A = (1, 0, 0)
B = (0, 0.3536, -0.3536)
C = (0, 0.3536,  0.3536)
\end{verbatim}
\end{center}
or
\begin{center}
\begin{verbatim}
A = ( 1* cos  0,  1* cos 90,   1* cos 90 )
B = (.5* cos 90, .5* cos 45, -.5* cos 45 )
C = (.5* cos 90, .5* cos 45,  .5* cos 45 )
\end{verbatim}
\end{center}

\subsection{Translate Operation}
\mfig ese-tr, Translating Ellipsoid to (-1, 1, 1).

\noindent{\tt
mged> {\em Select the ``Translate'' entry in the solid edit menu} \\
mged> {\em p -1 1 1} \\
mged>
}

The key point V is moved to (-1, 1, 1) and the ellipsoid maintains its
relative position to V.  See Figure \ref{ese-tr}.

While in the SOL EDIT state, the solid may be translated by
using the mouse.  These changes are not numerically exact, but they can be
useful to visually position a solid with respect to other solids.
Move the mouse to a position outside the menu area on the screen.
Click the mouse.
The center point (V) of the ellipsoid will be translated to that point.
Note that only the value of the coordinates of V are changed.
The command

\noindent{\tt
mged> {\em p 0 0 0} \\
mged>
}

will restore the original position.

\subsection{Rotate Operation}
\mfig ese-xrot, Solid Edit Rotation of 45 Degrees about X.
\mfig ese-yrot, Solid Edit Rotation of 45 Degrees about Y.
\mfig ese-zrot, Solid Edit Rotation of 45 Degrees about Z.

The rotate operation is initiated by either selecting Rotate on the menu
screen with the mouse,
by depressing the Solid Rotate button on the button box,
or by entering the {\em press srot} command on the keyboard.

\noindent{\tt
mged> {\em Select the ``Rotate'' entry in the solid edit menu} \\
mged> {\em p 45 0 0} \\
mged>
}

Figure \ref{ese-xrot} shows the rotation of the ellipsoid about its X axis.
The angle of rotation is counter-clockwise when viewed in the positive X
direction.  The direction cosines of vectors VB and VC are changed by 45 .

\noindent{\tt
mged> {\em Select the ``Rotate'' entry in the solid edit menu} \\
mged> {\em p 0 45 0} \\
mged>
}

Figure \ref{ese-yrot} shows the rotation of the ellipsoid about its Y axis.
The angle
of rotation is counter-clockwise when viewed in the positive Y direction.  The
rotation is made from the original view, and the restoration of the view is
not necessary.

\noindent{\tt
mged> {\em Select the ``Rotate'' entry in the solid edit menu} \\
mged> {\em p 0 0 45} \\
mged>
}

Figure \ref{ese-zrot} shows the rotation of the ellipsoid about its Z axis.
The axis
of rotation is counter-clockwise when viewed in the positive Z direction.
The command

\noindent{\tt
mged> {\em p 0 0 0} \\
mged>
}

restores the original orientation of the solid.

\subsection{Scale Operation}
\mfig ese-scale, Ellipsoid Scale Decreased.

\noindent{\tt
mged> {\em Select the ``Scale'' entry in the solid edit menu} \\
mged> {\em p .5} \\
mged>
}

Point V is not changed,
but the distance from V to the surface of the ellipsoid is multiplied by 0.5,
because the magnitude of the vectors are multiplied by the value of 0.5.
See Figure \ref{ese-scale}.

Move the mouse to a position outside the menu area and above the X axis,
and click the mouse.
Notice that the size of the ellipsoid has grown, ie,
the magnitude of the vectors have increased.
Move the mouse to a position below the X axis, and click the mouse.
Notice that the size of the ellipsoid has increased.

The command

\noindent{\tt
mged> {\em p 1} \\
mged>
}

will restore the original scale.

NOTE:
The use
of the scale operation from the Solid Edit menu
will result in the values of all the vectors being
multiplied by the value of the scale.
Use of the scale operaton from the Ellipsoid menu
with a particular vector A, B, or C changes the
magnitude of that vector to the value of the scale.

\subsection{Scale A Command}
\mfig ese-sa, Ellipsoid Scale A Vector.

\noindent{\tt
mged> {\em Select the ``edit menu'' entry in the solid edit menu} \\
mged> {\em Select the ``scale A'' entry in the ellipsoid menu} \\
mged> {\em p 1.5} \\
mged>
}

The magnitude of the vector to point A is set equal to the value of p
(eg 1.5).
The components of the vector are (1.5, 0, 0) since the vector was
parallel to the X axis.  See Figure \ref{ese-sa}.  The command

\noindent{\tt
mged> {\em Select the ``scale A'' entry in the TGC menu} \\
mged> {\em p 1} \\
mged>
}

will restore the original shape.

\subsection{Scale B Command}
\mfig ese-sb, Ellipsoid Scale B Vector.

\noindent{\tt
mged> {\em Select the ``scale B'' entry in the Ellipsoid menu} \\
mged> {\em p 1.5} \\
mged>
}

The magnitude of the vector to point B is set equal to the value of p
(eg 1.5).
The coordinates of the vector are the product of p and the
direction cosines of B.  See Figure \ref{ese-sb}.  The command

\noindent{\tt
mged> {\em p 0.5} \\
mged>
}

will restore the original shape.

\subsection{Scale C Command}
\mfig ese-sc, Ellipsoid Scale C Vector.

\noindent{\tt
mged> {\em Select the ``scale C'' entry in the Ellipsoid menu} \\
mged> {\em p 1.5} \\
mged>
}

The magnitude of the vector to point C is set equal to the value of p
(ie, 1.5).
The coordinates of the vector are the product of p and the
direction cosines of C.  See Figure \ref{ese-sc}.  The command

\noindent{\tt
mged> {\em p 0.5} \\
mged>
}

will restore the original shape.

To return control to the VIEWING state, select the ``REJECT Edit''
item on the button menu, press the ``reject'' button on the button box,
or enter the command {\em press reject} on the keyboard.
Then, enter

\noindent{\tt
mged> {\em d ell} \\
mged>
}

to drop the ellipsoid from view.

\section{Solid Edit: Alter Torus}
\mfig est-top, Top View of a Torus.
\mfig est-sed, The Torus in Solid Edit State.

This tutorial illustrates the application of the SOL EDIT state to the
torus solid.

\noindent{\tt
mged> {\em size 6} \\
mged> {\em in tor tor} \\
Enter X, Y, Z of vertex: {\em 0 0 0} \\
Enter X, Y, Z of normal vector: {\em 0 1 0} \\
Enter radius 1: {\em 1} \\
Enter radius 2: {\em 0.2} \\
mged>
}

Figure \ref{est-top} is the display of the torus solid in viewing state.
Since the
Z-axis is perpendicular to the viewing screen, a view of all sides cannot be
seen.

\noindent{\tt
mged> {\em Twist ROTY knob clockwise and restore} \\
mged> {\em Twist ROTX knob counter-clockwise and restore} \\
mged>
}

These actions generate a view of the torus that shows all sides,
as shown in Figure \ref{est-sed}.

\noindent{\tt
mged> {\em Select the ``Solid Illum'' entry in the button menu} \\
mged> {\em Move the mouse out of the menu area} \\
mged> {\em Click the mouse to enter SOL EDIT state} \\
mged>
}

The torus is a ring whose cross-section is a circle.  The distance from
the vertex to the center of the cross-section is r1 and r2 is the radius of
the circular cross section.

Let the points I and O be the intersection of the line x=-z and the torus.
Then,
\begin{center}
\begin{verbatim}
I = (-(r2-r1) cos 45, 0, (r2-r1) cos 45 )
O = (-(r2+r1) cos 45, 0, (r2+r1) cos 45 )
\end{verbatim}
\end{center}

\subsection{Translate Operation}
\mfig est-tr, Translating a Torus.

\noindent{\tt
mged> {\em Select the ``Translate'' entry in the solid edit menu} \\
mged> {\em p -.5 -1 .5} \\
mged>
}

The vertex V of the torus is moved to (-.5, -1, .5).
See figure \ref{est-tr}.  The
coordinates of the other points remain the same, relative to the vertex.
The command

\noindent{\tt
mged> {\em p 0 0 0} \\
mged>
}

will restore the original position.

\subsection{Rotate Operation}
\mfig est-xrot, Torus Solid Edit Rotation about X.
\mfig est-yrot, Torus Solid Edit Rotation about Y.
\mfig est-zrot, Torus Solid Edit Rotation about Z.

\noindent{\tt
mged> {\em Select the ``Rotate'' entry in the solid edit menu} \\
mged> {\em p 45 0 0} \\
mged>
}

The torus is rotated 45 degrees counter-clockwise about the positive X axis.
The
coordinates of the points I and H are transformed using the following matrix:
\begin{verbatim}
    [x'] [1   0      0   ] [x]
    [y']=[0  .7071 -.7071] [y]
    [z'] [0  .7071  .7071] [z]
\end{verbatim}
See Figure \ref{est-xrot}.

\noindent{\tt
mged> {\em p 0 45 0} \\
mged>
}

The torus is rotated 45 degrees counter-clockwise about the positive Y axis.
See Figure \ref{est-yrot}.

\noindent{\tt
mged> {\em p 0 0 45} \\
mged>
}

The torus is rotated 45 degrees counter-clockwise about the positive Z axis.
See Figure \ref{est-zrot}.
The original orientation is restored by entering

\noindent{\tt
mged> {\em p 0 0 0} \\
mged>
}

\subsection{Scale Operation}
\mfig est-scale, Torus Scale Increased.

\noindent{\tt
mged> {\em Select the ``Scale'' entry in the solid edit menu} \\
mged> {\em p 1.5} \\
mged>
}

The vertex remains the same and all distances from the vertex are
multiplied by 1.5, the value entered with p.  See Figure \ref{est-scale}.
To return to the original scale, enter

\noindent{\tt
mged> {\em p 1} \\
mged>
}

\subsection{Scale Radius 1 Command}
\mfig est-sr1, Scale Torus Radius 1.

\noindent{\tt
mged> {\em Select the ``edit menu'' entry in the solid edit menu} \\
mged> {\em Select the ``scale radius 1'' entry in the TORUS menu} \\
mged> {\em p 1.5} \\
mged>
}

The distance from the vertex to the center of the cross-section of the
ring is set equal to the values given with {\em p}, eg, 1.5.
See Figure \ref{est-sr1}.
The original scale can be restored with

\noindent{\tt
mged> {\em p 1} \\
mged>
}

\subsection{Scale Radius 2 Command}
\mfig est-sr2, Scale Torus Radius 2.

\noindent{\tt
mged> {\em Select the ``edit menu'' entry in the solid edit menu} \\
mged> {\em Select the ``scale radius 2'' entry in the TORUS menu} \\
mged> {\em p 0.5} \\
mged>
}

The distance from the center of the cross-section of the ring is set equal
to the value given with {\em p}, eg 0.5.
This value must remain less than the value for r1.
See Figure \ref{est-sr2}.
The command

\noindent{\tt
mged> {\em p 0.2} \\
mged>
}

will restore the original shape.

To return control to the VIEWING state, select the ``REJECT Edit''
item on the button menu, press the ``reject'' button on the button box,
or enter the command {\em press reject} on the keyboard.
Then, enter

\noindent{\tt
mged> {\em d tor} \\
mged>
}

to drop the torus from view.
\chapter{TUTORIALS ON OBJECT EDITING}

\section{Object Editing a Six-Sided Polyhedron}

This tutorial illustrates the application of the {\bf OBJ EDIT} state to
the ``arb8'' primitive.  The editing of the arb8 illustrates the
absolute movement of the points.

\mfig eo-start, ``arb8'' Object Edit; Top View.
\noindent
{\tt
mged> {\em e arb8}\\
vectorized in 0 seconds\\
mged> {\em size 8}\\
mged>\\
{\em Select the {\bf BUTTON MENU} if not already displayed.} \\
{\em Select the {\bf Object Illum} menu entry.} \\
{\em Move the mouse away from the menu area and select twice.} \\
}

These operations select an object for editing.
Control is passed through the {\bf OBJ PICK} and {\bf OBJ PATH} states
to the {\bf OBJ EDIT} state.  The display should look similar to
Figure \ref{eo-start}.

\subsection{Scale Operation}

\mfig eo-scale, ``arb8'' Object Edit; Scaled by 0.5.
\noindent
{\tt
mged> {\em scale 0.5}\\
mged>\\
}

As always, the selected operation operates with respect to the key
vertex -- point 1 remains the same, and the distances from point 1 to
the other points are multiplied by the scale factor.  See Figure
\ref{eo-scale}.  What will happen if another {\bf scale 0.5} command
is given?  The {\bf scale} operator is an absolute operator.  It sets
the scale factor associated with a particular transformation matrix.
It does not multiply the current transformation matrix scale factor by
the new scale factor.

\subsection{X, Y, and XY Move Operation}

\mfig eo-xyzmove, ``arb8'' Object Edit; Translated to (0.5, -2, 1.5).
\noindent
{\tt
mged> {\em scale 1}\\
mged> {\em translate .5 -2 1.5}\\
mged>\\
{\em Select the {\bf X move} menu entry.}\\
}

The first two commands undo the effects of the previous scale operation, and
translate the key point to (0.5, -2, 1.5).  The coordinates of the other
points are changed accordingly; preserving their distances relative
to point 1 (see Figure \ref{eo-xyzmove}).
The last operation above placed MGED into a state where the key point
of the object will track the X component of successive selects, but not the Y
component. Note that on some displays point 1 may not be directly
visible.  It is actually behind point 4. Watch the area listing the
information concerning ``arb8'' as selections are made.

\noindent
{\em Do several selects while moving the cursor slowly in a circle.}\\

   Observe that only the X axis information changes. Similarly, the {\bf
Y move} tracks only changes in the Y axis, and {\bf XY move} tracks
changes in both axes.

\mfig eo-xymove, XY Move.
\noindent
{\tt mged> }{\em press 45,45}\\
{\tt mged> }\\
{\em Do several more selects while moving the cursor slowly in a circle.}\\

   This time, moving point 1 modifies the model x and y axes (observe the
changes in the list of vertices).  If {\bf Y move} is selected, all
three sets of axes are modified.  These operators work in screen space,
not model space, so using them with an oblique view moves the model in
more than one axis (Figure \ref{eo-xymove} for example)..


\subsection{Rotate Operation}

\mfig eo-arbrot, ``arb8'' Rotated by (30, 45, 60).
\noindent
{\tt
mged> {\em press reset}\\
mged> {\em translate 0 0 0}\\
mged> {\em rotobj 30 45 60}\\
mged>\\
}

The primitive is rotated 60, 45, and 30 about the z, y, and x axes in
that order as shown in Figure \ref{eo-arbrot}.
Note that the coordinates of the points are changed when scaling,
translation, and rotation are performed.

\noindent
{\tt
mged> {\em press reset}\\
mged> {\em d arb8}\\
mged>\\
}

If an object which is being edited is deleted from view, MGED transitions
back into the {\bf VIEWING} state.

\section{Object Editing an Ellipsoid}

\mfig eo-ellg, Object Edit; Ellipse viewed from 45,45 preset.
\noindent
{\tt
mged> {\em e ellg}\\
mged>\\
{\em Select the {\bf 45,45} button menu entry.}\\
{\em Select the {\bf Object Illum} button menu entry.}\\
{\em Move the cursor outside the menu area and select twice.}\\
}

Control is again passed from the {\bf OBJ PICK} state through the
{\bf OBJ PATH} state to the {\bf OBJ EDIT} state.
When editing a cylinder, ellipsoid, or torus, the coordinates of the
primitives are set relative to the center of the primitive, and are not
changed by any translation of the primitive.  Figure \ref{eo-ellg}
represents the display.

\subsection{Scale Operation}

\mfig eo-ellg2x, Object Edit; Ellipse scaled up by 2.
\noindent
{\tt
mged> {\em scale 2}\\
mged>\\
}

The magnitudes of the vectors from the center V to the points A, B, and C
are multiplied by the scale factor 2.  The location of the center V is
unchanged.  See Figure \ref{eo-ellg2x}.

\subsection{Move Operations}

\mfig eo-ellgxyz, Object Edit; Ellipse Translated.
\noindent
{\tt
mged> {\em scale 1}\\
mged>\\
{\em Select the {\bf XY move} buttom menu entry.}\\
{\em Move the cursor to some location away from the menu area and select.}\\
}

The as with the arb8 in the previous section, the ellipsoid is moved
in the model space plane parallel to the plane of the screen so the
key point (vertex V) is ``placed at'' the point corresponding to the
cursor location.  Note that there is no explicit control over the
location of the point with respect to screen Z (depth in the viewing
cube); the selected point has the same screen Z as the original point.
The screen should look similar to Figure \ref{eo-ellgxyz}.

\noindent
{\tt
mged> {\em d ellg}\\
mged>\\
}

\section{Object Path and Object Edit}

This section illustrates the use of the {\bf OBJ PATH} state to select the
number of objects that are affected by one edit command.  In the previous
sections the user was shown how to manipulate only one primitive.  A group
of primitives may also be edited as a single entity.  This section
shows how an entire group may be edited without addressing each individual
primitive.

MGED generally has several ways to achieve a particular result.  In
this section, keyboard commands are used instead of the control buttons and
the display menu.  The creation and saving of special primitives is
illustrated.

In the database, the original primitives are centered around
the origin.  Copies of these primitives will be made, translated away
from the origin and saved for future editing.

\subsection{Organize the Primitives and Groups}

\mfig eo-stacked, Stacked Primatives.
\noindent
{\tt
mged> {\em cp arb8 arb8.s}\\
mged> {\em cp ellg ellg.s}\\
mged> {\em cp tgc tgc.s}\\
mged> {\em cp tor tor.s}\\
mged>\\
}

Figure \ref{eo-stacked} shows the four primitives; arb8.s, ellg.s, tgc.s,
and tor.s.  One convention frequently used by experienced modelers is to
tack an identifying suffix on the names of the various primitives and
objects.  Often, a ``.s'' suffix denotes a {\em solid}, a ``.r'' denotes
a {\em region} and a ``.g'' denotes a {\em group} (note that this is a
different ``.g'' from the ``.g'' suffix used with filenames.).

\noindent
{\tt
mged> {\em size 16}\\
mged> {\em sed arb8.s}\\
mged> {\em press sxy}\\
mged> {\em p 3 -3 1}\\
mged> {\em press accept}\\
mged>\\
}

Several things happened in the above sequence.  The net result is that
the solid arb8.s was ``unstacked'' using a Solid Edit so it is more
visible.  The same sequence of operations will be performed with the
other objects to move them to other locations.

\mfig eo-spread, Primatives After Translation.
\noindent
{\tt
mged> {\em sed ellg.s}\\
mged> {\em press sxy}\\
mged> {\em p -3 3 1}\\
mged> {\em press accept}\\
mged> {\em sed tgc.s}\\
mged> {\em press sxy}\\
mged> {\em p -3 -3 1}\\
mged> {\em press accept}\\
mged> {\em sed tor.s}\\
mged> {\em press sxy}\\
mged> {\em p 3 3 1}\\
mged> {\em press accept}\\
mged>\\
}

The screen should now look like Figure \ref{eo-spread}.  The next step
is to group the primitives

\noindent
{\tt
mged> {\em g a.g tgc.s arb8.s}\\
mged> {\em g b.g ellg.s tor.s}\\
mged> {\em g c.g a.g b.g}\\
mged> {\em B c.g}\\
vectorized in 0 sec\\
mged> {\em tree c.g}
\begin{verbatim}
| c.g_____________| a.g_____________| tgc.s
                                    | arb8.s
                  | b.g_____________| ellg.s
                                    | tor.s
\end{verbatim}
\noindent
mged>\\
}

The {\em group} operator ({\bf g}) generates an object, named by the
first argument, which is the union of all objects named in succeeding
arguments.  Therefore, the object ``a.g'' is composed of the union of
``tgc.s'' and ``arb8.s''.  Likewise, the object ``c.g'' is the union of
``a.g'' and ``b.g''.  The next command in the above sequence is called
the {\em blast} command.  It is effectively a {\em zap} ({\bf Z})
followed by an {\em edit} ({\bf e}).

The final command is the {\em tree} command.  It is intended to give the
user some idea of the hierarchical structure of an object.  It presents
a tree laid on it's side.  The root is at the left, and the leaves are
at the right. A vertical bar denotes a connection at a given level, with
the proviso that a vertical bar having a line of underscores coming
in from the left represents the start of a particular subtree when read
from top down (``ellg.s'' and ``arb8.s'' do not have a common parent).

\mfig eo-grpath, Object Path With ``tor.s'' as Reference Solid.
\noindent
{\tt
mged> {\em press oill}\\
mged>\\
{\em Move the cursor up and down the screen until the primitive ``tor.s''
is illuminated, then select}\\
}

Selecting the solid ``tor.s'' transitions MGED into the {\bf OBJ PATH}
state, and establishes ``tor.s'' as the reference solid for any future
editing operations.
Note that the name ``tor.s'' is shown in the upper left corner of the
display, and on the second status line at the bottom of the display
(Figure \ref{eo-grpath}).

The {\bf OBJ PATH} state has little meaning unless there is more than one path
or group in the display.  One of the following paths may be selected:

\begin{quote}
c.g/b.g/\_MATRIX\_/tor.s

c.g/\_MATRIX\_/b.g/tor.s

\_MATRIX\_/c.g/b.g/tor.s
\end{quote}

Although the torus primitive has been selected as the reference solid,
the position of {\bf \_MATRIX\_} determines the extent of the effects of
the edit.  The first choice affects only the torus.  The second choice
affects everything under the group ``b.g'' (the torus and ellipsoid).
The third choice affects all of the primitives.  Remember though, that
in all cases, what is being edited is the Homogeneous Transformation
Matrix (thought of as the arc connecting objects), not the underlying
solid.

\subsection{Editing One Primitive}

\mfig eo-gredit, Object Edit With ``tor.s'' as Reference Solid.
\noindent
{\em Move the cursor up and down until {\tt /c.g/b.g/\_MATRIX\_/tor.s} appears
in the bottom status line, then select.}\\

Figure \ref{eo-gredit} is the new display.  Note that the torus is
illuminated. The {\bf OBJ EDIT} state has been reached.

\mfig eo-tor111, Object Edit Affecting Torus Only.
\noindent
{\tt
mged> {\em translate 1 1 1}\\
mged>\\
}

The key point of the torus is moved to 1, 1, 1.  The other primitives are
not moved.  See Figure \ref{eo-tor111}.

\noindent
{\tt
mged> {\em press reject}\\
mged>\\
}

\subsection{Editing a Group of Two Primitives}

\mfig eo-bgrp, Torus and Ellipsoid Selected for Object Edit.
\noindent
{\tt
mged> {\em press oill}\\
mged>\\
{\em Move the cursor up and down the screen until the primitive ``tor.s''
is illuminated, then select.}\\
{\em Move the cursor up and down until {\tt /c.g/\_MATRIX\_/b.g/tor.s} appears
in the bottom status line, then select.}\\
}

Control has again been passed to the {\bf OBJ EDIT} state. Notice that
both the torus and ellipsoid are illuminated (Figure \ref{eo-bgrp}.
Although only the parameters for the torus will be changed in the
display, the ellipsoid in group ``b.g'' will be affected by the edit.

\mfig eo-bgrp311, Torus and Ellipsoid Translated by (3, 1, 1).
\noindent
{\tt
mged> {\em translate 3 1 1}\\
mged>\\
}

The key point of the torus is moved to 3, 1, 1. The ellipsoid is
moved by the same amount.  See Figure \ref{eo-bgrp311}.

\noindent
{\tt
mged> {\em press reject}\\
mged>\\
}

\subsection{Editing Two Groups of Four Primitives}

\mfig eo-cgrp, All Primitives Selected for Object Edit.
\noindent
{\tt
mged> {\em press oill}\\
mged>\\
{\em Move the cursor up and down the screen until the primitive ``tor.s''
is illuminated, then select.}\\
{\em Move the cursor up and down until {\tt /\_MATRIX\_/c.g/b.g/tor.s} appears
in the bottom status line, then select.}\\
}

Control has again been passed to the {\bf OBJ EDIT} state. Notice that
all of the primitives are illuminated (Figure \ref{eo-cgrp}).
Although only the parameters for the torus will be changed in the
display, the ellipsoid in group ``b.g'' will be affected by the edit.

\mfig eo-cgrp321, All Primitives Translated by (3, 2, 1).
\noindent
{\tt
mged> {\em translate 3 2 1}\\
mged>\\
}

The key point of the torus is moved to 3, 2, 1. All other primitives are
moved by the same amount.  See Figure \ref{eo-cgrp321}.

\noindent
{\tt
mged> {\em press reject}\\
mged>\\
}

Control has now returned to the VIEWING state.
\chapter{BUILDING A SET OF COORDINATE AXES}
\mfig axis-3525, The Model Axes Viewed from 35,25.
\mfig rmit-3525, Example Axes Viewed form 35,25.

MGED does not display a set of XYZ co-ordinate axes on the
screen.
When you are in the 35,25 (isometric) viewing state the axes are
positioned as in Figure \ref{axis-3525}.
This database can be found in cad/db/axis.g.

If you would like a set of coordinate axes to assist in model building,
the easiest thing to do is to
construct three axes using "rcc" cylinder primitives via the "in" command;

\noindent{\tt
mged> {\em in x rcc 0 0 0  50 0 0  1} \\
mged> {\em in y rcc 0 0 0  0 100 0  1} \\
mged> {\em in z rcc 0 0 0  0 0 150  1} \\
mged>
}

with the short leg as the "x" axis, next longer leg the "y" axis and longest
leg the "z" axis,
as in Figure \ref{rmit-3525}.
Now, at any stage through construction of the model,
the 'solid' or 'object illuminate' mode can be used
to identify which axis cylinder is going where; they
will have the solid names of "x", "y", and "z".
The name of the solid will also be
displayed in the top left hand corner of the graphics window
and at the bottom of this window.

Before going on to create a model, construct the three axes cylinders
with the "in" commands mentioned above.
Select the "button menu" in the
upper left corner of the graphics window
to enable the button menu, and
select 35, 25 from this menu.
Your axis will be displayed as shown in Figure \ref{rmit-3525}.
\chapter{BUILDING A TIN WOODSMAN}

\mfig wm-prims, WoodsMan Primitives.
The purpose of this tutorial is to demonstrate how to build a model using
a few basic primitives.  The model to be constructed is a tin woodsman.
The four primitives used in the construction of the tin woodsman
are an ARB8, a cylinder, an ellipsoid, and a torus.
These four primitives will be duplicated several times,
and each copy will be modified using solid editing,
to obtain the required shapes.
The finished version of this database can be found in
the BRL-CAD Package file ``db/woodsman.g''.

\section{Create Primitives}

\noindent{\tt
\$ {\em mged woodsman.g} \\
BRL-CAD Release 3.0 Graphics Editor (MGED) Compilation 82 \\
    Thu Sep 22 08:08:39 EDT 1988 \\
    mikel@video.br:/cad/.mged.4d2 \\
 \\
woodsman.g: No such file or directory \\
Create new database (y|n)[n]? {\em y} \\
attach (nu|tek|tek4109|ps|plot|sgi)[nu]? {\em sgi} \\
ATTACHING sgi (SGI 4d) \\
Untitled MGED Database (units=mm) \\
mged> {\em size 20} \\
mged> {\em title A Tin Woodsman} \\
mged> {\em in solid8 rpp} \\
Enter XMIN, XMAX, YMIN, YMAX, ZMIN, ZMAX: {\em -1 1  -1 1  -1 1} \\
mged> {\em in torus tor} \\
Enter X, Y, Z of vertex: {\em 0 0 0} \\
Enter X, Y, Z of normal vector: {\em 0 1 0} \\
Enter radius 1: {\em 1} \\
Enter radius 2: {\em 0.2} \\
mged> {\em in ellipsoid ellg} \\
Enter X, Y, Z of vertex: {\em 0 0 0} \\
Enter X, Y, Z of vector A: {\em 1 0 0} \\
Enter X, Y, Z of vector B: {\em 0 0.5 0} \\
Enter X, Y, Z of vector C: {\em 0 0 0.5} \\
mged> {\em in cylinder rcc} \\
Enter X, Y, Z of vertex: {\em 0 0 0} \\
Enter X, Y, Z of height (H) vector: {\em 2 0 0} \\
Enter radius: {\em 1} \\
mged>
}

At this point, the screen should look like Figure \ref{wm-prims}.

\section{Copy Primitives and Set Up for Edit}

Although eight copies of the ellipsoid and two copies of the cylinder
shall be used in the final solid, fewer copies are made initially since there
is replication in the editing of these primitives.

\noindent{\tt
mged> {\em cp ellipsoid e.2} \\
mged> {\em cp ellipsoid e.6} \\
mged> {\em cp cylinder c.1} \\
mged> {\em cp solid8 s.1} \\
mged> {\em cp torus t.1} \\
mged> {\em Z} \\
mged> {\em e e.* c.1 t.1 s.1} \\
vectorized in 0 sec \\
mged>  \\
}

\mfig wm-hat1, Funnel Bowl Cylinder After Rotation.
\mfig wm-hat2, Funnel Bowl Cylinder After End Scaling.
\mfig wm-hat3, Funnel Bowl Cylinder After Moving.
\mfig wm-tube, Funnel Tube Scaled and Positioned.
\section{Create Funnel Hat}

The solid ``cylinder'' has a height vector (``H'') which is 2mm
long.
This will be used to good advantage, to make the Tin Woodsman's
funnel hat, with the bowl of the funnel being 2mm high,
and the tube of the funnel being 2mm long.
The tube of the funnel will point straight up the +Y axis.

\noindent{\tt
mged> {\em sed c.1} \\
mged> {\em Select the ``Rotate'' entry in the solid edit menu} \\
mged> {\em p 0 45 90} \\
mged>
}

This places the cylinder so that the lines BD and AC are at the outer
ends of the cylinder.  See Figure \ref{wm-hat1}.

Next, the cylinder is shaped to look like the top of a funnel.
The vectors c and d are scaled.

\noindent{\tt
{\em Select the ``edit menu'' entry in the solid edit menu} \\
{\em Select the ``scale c'' entry in the TGC menu} \\
mged> {\em p .1} \\
{\em Select the ``scale d'' entry in the TGC menu} \\
mged> {\em p .1} \\
mged> \\
}

Figure \ref{wm-hat2} is the new shape of the cone.
Note how the on-screen display records the new lengths of the ``c''
and ``d'' vectors.
This cone must be moved to the planned locations for the top of the head.

\noindent{\tt
{\em Select the ``Translate'' entry in the solid edit menu} \\
mged> {\em p 0 2.2 0} \\
{\em Select the ``ACCEPT Edit'' entry in the button menu} \\
mged> \\
}

The bottom of the hat is now properly shaped and positioned.  The
new version of the solid ``c.1'' has been saved in the model database.
The editor returns to the viewing state.  See Figure \ref{wm-hat3}.

A copy of the saved ``c.1'' cone is made.  A byproduct of the {\em cp}
command is to display the new solid, as if the {\em cp c.1 c.2} command
had been immediately followed by an {\em e c.2} command.
This new solid will be edited to make the neck of the funnel,
which is the top of the hat.
The ``c.2'' copy of the cone must be
scaled down to become a tube and the tube must be placed on top of the
cone ``c.1''.

\noindent{\tt
mged> {\em cp c.1 c.2} \\
mged> {\em sed c.2} \\
{\em Select ``scale A,B'' in the TGC menu} \\
mged> {\em p 0.1} \\
{\em Select ``Translate'' in the Solid Edit menu} \\
mged> {\em p 0 4.2 0} \\
{\em Select ``ACCEPT Edit'' in the Button menu} \\
mged>
}

Figure \ref{wm-tube} is the new shape of the funnel tube.
The woodsman's hat is comprised of solids c.1 and c.2.

\mfig wm-head, Head Sphere.
\section{Building the Head}

The head of our Tin Woodsman is perfectly spherical, and will be located
at coordinates (0, 2, 0).
While it would be possible to duplicate the ellipsoid solid created above,
and modifiy it to produce the desired sphere, since all the parameters
of the head sphere are known, it is more economical simply to use
the {\em in} command to construct it directly.
Figure \ref{wm-head} shows the results of this operation.

\noindent{\tt
mged> {\em in e.1} \\
Enter solid type: {\em sph} \\
Enter X, Y, Z of vertex: {\em 0 2 0} \\
Enter radius: {\em 1} \\
mged>
}

\mfig wm-collar, The Woodsman's Collar.
\section{Building the Collar}

The torus (primitive t.1) is used to build a collar between the head and
the body.
The ring of the collar is scaled to 0.1 of its original size,
and repositioned at the base of the head.
The results of this step are shown in Figure \ref{wm-collar}.

\noindent{\tt
mged> {\em sed t.1} \\
{\em Select ``scale radius 2'' in the TORUS menu} \\
mged> {\em p 0.1} \\
{\em Select ``Translate'' in the Solid Edit menu} \\
mged> {\em p 0 1 0} \\
{\em Select ``ACCEPT Edit'' in the Button menu} \\
mged>
}

\mfig wm-body, The Woodsman's Body.
\section{Building the Body}

The ARB8 (primitive s.1) is used to build the body.
The original height of s.1 is only 2mm;  the required
length of the body is 3mm, so the extrusion command is
used to adjust the position of the lower (-Y) face of the solid.
The result is shown in Figure \ref{wm-body}.

\noindent{\tt
mged> {\em sed s.1} \\
mged> {\em extrude 2367 3} \\
{\em Select ``ACCEPT Edit'' in the Button menu} \\
mged>
}

\mfig wm-arm1, An Upper Arm Prototype.
\mfig wm-arm2, The Woodsman's Arms.
\section{Building the Arms}

The ellipsoid primitive (e.2) is used to build the upper and lower parts
of the left and right arms.
The original solid ``e.2'' is oriented with the major axis of the ellipse
oriented along the X axis.
The arms need to have the major axis oriented along the Y axis,
so first the solid is rotated.
A more graceful arm is obtained by decreasing the length of
the B and C vectors, and the resulting upper arm solid can be
seen in Figure \ref{wm-arm1}.

\noindent{\tt
mged> {\em sed e.2} \\
{\em Select ``Rotate'' in the Solid Edit menu} \\
mged> {\em p 0 45 90} \\
{\em Select ``edit menu'' in the Solid Edit menu} \\
{\em Select ``scale B'' in the ELLIPSOID menu} \\
mged> {\em p 0.25} \\
{\em Select ``scale C'' in the ELLIPSOID menu} \\
mged> {\em p 0.25} \\
mged>
}

This e.2 solid will now be moved into final position as the upper left arm.
Then it will be duplicated three times
to make the rest of the arm parts.  Finally, each new arm
part will be translated into the proper position,
as seen in Figure \ref{wm-arm2}.

\noindent{\tt
{\em Select ``Translate'' in the Solid Edit menu} \\
mged> {\em p -1.3 0 0} \\
{\em Select ``ACCEPT Edit'' in the Button menu -- This is the upper left arm} \\
mged> {\em cp e.2 e.3} \\
mged> {\em cp e.2 e.4} \\
mged> {\em cp e.2 e.5} \\
mged> {\em sed e.3} \\
{\em Select ``Translate'' in the Solid Edit menu} \\
mged> {\em p -1.3 -2 0} \\
{\em Select ``ACCEPT Edit'' in the Button menu -- This is the lower left arm} \\
mged> {\em sed e.4} \\
{\em Select ``Translate'' in the Solid Edit menu} \\
mged> {\em p 1.3 0 0} \\
{\em Select ``ACCEPT Edit'' in the Button menu -- This is the upper right arm} \\
mged> {\em sed e.5} \\
{\em Select ``Translate'' in the Solid Edit menu} \\
mged> {\em p 1.3 -2 0} \\
{\em Select ``ACCEPT Edit'' in the Button menu -- This is the lower right arm} \\
mged>
}

\mfig wm-leg1, The First Leg.
\mfig wm-final1, The Tin Woodsman.
\section{Building the Legs}

The ellipsoid primitive (e.6) is used as a prototype
to build the upper and lower parts of both legs from.
The primitive e.6 is scaled, rotated, and translated
into position as the upper left leg, as seen in Figure \ref{wm-leg1}.
Then, copies are made and translated to the remaining positions,
just like the arms were.

\noindent{\tt
mged> {\em sed e.6} \\
{\em Select ``Rotate'' in the Solid Edit menu} \\
mged> {\em p 0 45 90} \\
{\em Select ``Translate'' in the Solid Edit menu} \\
mged> {\em p -0.5 -3 0} \\
{\em Select ``ACCEPT Edit'' in the Button menu -- This is the upper left leg} \\
mged> {\em cp e.6 e.7} \\
mged> {\em cp e.6 e.8} \\
mged> {\em cp e.6 e.9} \\
mged> {\em sed e.7} \\
{\em Select ``Translate'' in the Solid Edit menu} \\
mged> {\em p -0.5 -5 0} \\
{\em Select ``ACCEPT Edit'' in the Button menu -- This is the lower left leg} \\
mged> {\em sed e.8} \\
{\em Select ``Translate'' in the Solid Edit menu} \\
mged> {\em p 0.5 -3 0} \\
{\em Select ``ACCEPT Edit'' in the Button menu -- This is the upper right leg} \\
mged> {\em sed e.9} \\
{\em Select ``Translate'' in the Solid Edit menu} \\
mged> {\em p 0.5 -5 0} \\
{\em Select ``ACCEPT Edit'' in the Button menu -- This is the lower right leg} \\
}

Figure \ref{wm-final1} is the view on the screen, the Tin Woodsman.
Take a moment to use the rotation knobs to view the model from various
angles.

\section{Building Regions}

So far, this example has concentrated on describing the basic shapes
involved in making the Tin Woodsman, without concern for establishing
a proper hierarchical structure.  To illustrate this point,
the various solids will be grouped by purpose and composition.
First, a region will be constructed to contain the torso,
and the color of ``cadet blue'' will be assigned:

\noindent{\tt
mged> {\em r torso.r u s.1} \\
Defaulting item number to 1001 \\
Creating region id=1000, air=0, los=100, GIFTmaterial=1 \\
mged> {\em mater torso.r} \\
Material = \\
Material?  (CR to skip) {\em plastic} \\
Param = \\
Parameter string? (CR to skip) {\em [RETURN]} \\
Color = (No color specified) \\
Color R G B (0..255)? (CR to skip) {\em 95 159 159} \\
Inherit = 0:  lower nodes (towards leaves) override \\
Inheritance (0|1)? (CR to skip) {\em [RETURN]} \\
mged>
}

Second, a region will be constructed to contain the collar,
which will be colored red:

\noindent{\tt
mged> {\em r collar.r u t.1} \\
Defaulting item number to 1003 \\
Creating region id=1002, air=0, los=100, GIFTmaterial=1 \\
mged> {\em mater collar.r} \\
Material = \\
Material?  (CR to skip) {\em plastic} \\
Param = \\
Parameter string? (CR to skip)  {\em [RETURN]} \\
Color = (No color specified) \\
Color R G B (0..255)? (CR to skip) {\em 255 127 0} \\
Inherit = 0:  lower nodes (towards leaves) override \\
Inheritance (0|1)? (CR to skip)  {\em [RETURN]} \\
mged>
}

Third, a region will be constructed to contain all the limbs,
and a flesh color will be assigned.
Even though none of the limbs touch each other, note how they
are combined with the UNION operation, to create a single
object of uniform composition and color.

\noindent{\tt
mged> {\em r limbs.r u e.2 u e.3 u e.4 u e.5 u e.6 u e.7 u e.8 u e.9} \\
Defaulting item number to 1001 \\
Creating region id=1000, air=0, los=100, GIFTmaterial=1 \\
mged> {\em mater limbs.r} \\
Material = \\
Material?  (CR to skip) {\em plastic} \\
Param = \\
Parameter string? (CR to skip)  {\em [RETURN]} \\
Color = 0 0 0 \\
Color R G B (0..255)? (CR to skip) {\em 255 200 160} \\
Inherit = 0:  lower nodes (towards leaves) override \\
Inheritance (0|1)? (CR to skip)  {\em [RETURN]} \\
mged>
}

Next, the funnel needs to be placed in a region.
For the sake of simplicity, the funnel will be solid, rather
than having a hollow center.
Note that the interior of the funnel overlaps with the top
of the Woodsman's head.
The funnel can be made ``form fitting'' by subtracting out
the overlap zone:

\noindent{\tt
mged> {\em r funnel.r u c.1 - e.1 u c.2 - e.1} \\
Defaulting item number to 1004 \\
Creating region id=1003, air=0, los=100, GIFTmaterial=1 \\
mged> {\em mater funnel.r} \\
Material = \\
Material?  (CR to skip) {\em plastic} \\
Param = \\
Parameter string? (CR to skip) {\em sh=100} \\
Color = (No color specified) \\
Color R G B (0..255)? (CR to skip) {\em 35 107 142} \\
Inherit = 0:  lower nodes (towards leaves) override \\
Inheritance (0|1)? (CR to skip)
mged> {\em l funnel.r} \\
funnel.r (len 4) REGION id=1003  (air=0, los=100, GIFTmater=1) -- \\
Material 'plastic' 'sh=100' \\
Color 35 107 142 \\
\ \ u c.1 \\
\ \ - e.1 \\
\ \ u c.2 \\
\ \ - e.1 \\
mged>
}

\mfig wm-hat-E, Evaluation of Funnel Hat Region.
Note how the boolean expression was written.
The concept that we need to express here is
the combination of all the funnel parts, minus the
portion of the head that overlaps with the inside of the funnel.
The natural way to write this is
\begin{center}
(c.1 union c.2) - e.1
\end{center}
but note that there are no grouping operations permitted in the {\em r}
command.
Furthermore, for historic reasons, union operations bind more loosely than
intersection and subtraction, ie, there are implied groups
between union operations.  Thus, the expression above needs to be
rewritten as the the formula:
\begin{center}
(c.1 - e.1) union (c.2 - e.1)
\end{center}
which with the binding precedence can be expressed as:
\begin{center}
c.1 - e.1 union c.2 - e.1
\end{center}
which is what was entered in the sequence above.
To see the effect that this command had on the shape of ``funnel.r'',
run these commands, the effect of which is shown in Figure \ref{wm-hat-E}:

\noindent{\tt
mged> {\em Z } \\
mged> {\em E funnel.r } \\
vectorized in 1 sec \\
mged>
}

These regions should be grouped together into a group,
for convenience in referencing.  This can be done with these commands:

\noindent{\tt
mged> {\em g man.g collar.r funnel.r limbs.r torso.r} \\
mged> {\em Z} \\
mged> {\em e man.g} \\
vectorized in 1 sec \\
mged>
}

The grouping {\em g} command combined the regions, the Zap command {\em Z}
cleared the screen, and the edit {\em e man.g} command drew the whole
object.
As an exercise, run
the database structure printing command {\em tree man.g}
to obtain a simple depiction of the tree structure that has been created.
For the final step of this example, the model will be ray-traced.
Run the command:

\noindent{\tt
mged> {\em rt -s128} \\
rt -s50 -M -s128 woodsman.g man.g  \\
db title:  A Tin Woodsman \\
Buffering single scanlines \\
initial dynamic memory use=35152.\\
Interpreting command stream in old format\\
GETTREE: 0.01 CPU secs in 1 elapsed secs (1\%)\\
                                              \\
...................Frame     0...................\\
PREP: 0.01 CPU secs in 0.01 elapsed secs (100\%)\\
shooting at 13 solids in 4 regions              \\
model X(-2,2), Y(-6,7), Z(-2,2)\\
Beam radius=0.078125 mm, divergance=0 mm/1mm\\
                                             \\
SHOT: 3.73 CPU secs in 6 elapsed secs (62.1667\%)\\
Additional dynamic memory used=29728. bytes\\
3515 solid/ray intersections: 1005 hits + 2510 miss\\
pruned 28.6\%:  13647 model RPP, 8197 dups, 10740 RPP\\
Frame     0:    16384 pixels in       3.73 sec =    4392.49 pixels/sec\\
Frame     0:    16384 rays   in       3.73 sec =    4392.49 rays/sec (RTFM)\\
\\
Press RETURN to reattach\\
{\em [RETURN]} \\
mged>
}
\chapter{BUILDING A ROBOT ARM}

\mfig robot, The RMIT Robot Arm.
The model shown in Figure \ref{robot}
will be described in a step by step instructions
on how to build and display this model.

This is the MGED input file:

\begin{verbatim}
in btm	box	0 0 0    0 -90 0      40 0 0   0 0 6
in btm1	box	0 -90 0  0 -61.549 0  40 0 0   0 0 6
in rad	rcc	20 -150 0   0 0 6   8
in cyl	rcc	20 -45 6    0 0 30  20
in cyl1 rcc	20 -45 0 0 0 36 15.5
in cyl2 rcc	20 -45 0 0 0 36 12.5
in hole rcc	8 -8 0   0 0 6   3
in hole1 rcc	32 -8 0  0 0 6   3
cp hole1 hole2
in gus	raw	21.5 -25.3 6  0 0 30  0 25.3 0  -3 0 0
in cnr	box	0 0 0	6 6 0	6 0 0	0 0 6
in cnr1	box	34 0 0	0 -6 0	6 0 0	0 0 6
cp cnr cnr2
cp cnr1 cnr3
in rad1 rcc	6 -6 0	0 0 6 6
in rad2 rcc	34 -6 0	0 0 6 6
in head rcc	20 -45 36 0 0 30 18
in shaft rcc	20 -45 36 0 0 -50 12.5
in han	rcc	20 -45 51 0 120 0 6
in ball sph	20 75 51  15
in cut box	20 -45 0	0 50 0	25 0 0	0 0 40
in squ box	12 -53 -14  	0 16 0   16 0 0   0 0 -30
r handle u squ u shaft u han - ball
r knob u ball
r cor u cnr2 + rad1
r cor1 u cnr3 + rad2
in hole4 rcc 20 -150 0   0 0 6   3
cp hole2 hole3
r base u btm u btm1 - hole2 - hole3 - hole4 u rad - hole4
g all base handle knob
size 300
e all
\end{verbatim}

This is the MGED dialog:

\begin{verbatim}
mged mark
BRL Graphics Editor (MGED) Version 2.31
  Sat Oct 17 20:33:05 PDT 1987
  mg\@godzilla:/usr/staff/mg/brlcad/mged

mark: No such file or directory
Crete new database (y/n)[n]? y
attach (nu|tek|plot|ir) [nu]? nu
ATTACHING nu (Null Display)
Untitled MGED Database (units=mm)
mged> in btm box 0 0 0 0 -90 0 40 0 0 0 0 6
mged> in btm1 box
Enter X, Y, Z of vertex:  0 -90 0
Enter X, Y, Z of vector H:  0 -61.549 0
Enter X, Y, Z of vector W:  50 0 0   40 0 0
Enter X, Y, Z of vector D: 0 0 6
mged> in rad rcc 20 -150 0 0 0 6 8
mged> in cyl rcc
Enter X, Y, Z of vertex:  20 -45 6
Enter X, Y, Z of height (H) vector: 0 0 30
Enter radius:  20
mged> in cyl1 rcc 20 -45 0 0 0 36 15.5
mged> in cyl2 rcc 20 -45 0 0 0 36 12.5
mged> in hole rcc 8 -8 0 0 0 6 3
mged> in hl ole 1 rcc 2 32 -8 0 0 0 6 2 3
mged> cp hole1 hole2
mged> in gus raw
Enter X, Y, Z of vertex: 21.5 -25.3 6
Enter X, Y, Z of vector H: 0 0 30
Enter X, Y, Z of vector W: 0 25.3 6
Enter X, Y, Z of vector D: -3 0 0
mged> in cnr box 0 0 0 06 6 0 6 0 0 0 0 6
mged> incnr1 box 34 0 0 0 -6 0 6 0 0 0 0 6
incnr1: no such command, type ? for help
mged> in cnr1 box 34 0 0 0 -6 0 6 0 0 0 0 6
mged> cp cnf r cnr2
mged> in cp cnr1 cnr3
mged> in rad1 rcc 6 -6 0 0 0 6 6
mged> in rad2 rcc 34 -6 0 0 0 6 6
mged> in shaft rcc 20 -45 36 0 0 30 18
mged> in shaft rcc 20 -45 36 0 0 -50 12.5
mged> in han rcc 20 -45 51 0 120 0 6
mged> in ball sph
Enter X, Y, Z of vertex: 20 75 51
Enter radius: 15
mged> in cut box 20 -45 0 0 50 0 25 0 0 0 040
Enter Z: 03  NOTE: error again
mged> killall cut
mged> in cut box 20 -45 0 0 50 0 25 0 0 0 0 40
mged> in squ box 12 -53 -14 0 16 0 16 0 0 0 0- -30
mged> r handle + squ shaft u han u ball
Defaulting item number to 1001
Creating region id=1000, air=0, los=100, GIFT material=1
mged> r knob + ball
Defaulting item number to 1002
Creating region id=1001, air=0, los=100, GIFT material=1
mged> r cor + cnr2 + rad1
Defaulting item number to 1003
Creating region id=1002, air=0, los=100, GIFT material=1
mged> r cor1 + cnr3 + rad2
Defaulting item number to 1004
Creating region id=1003, air=0, los=100, GIFT material=1
mged> mater knob plastic
Was
Parameter string? n
Override material color (y/n)[n]? y
R G B (0..255)? 255 0 0  NOTE:  This is color RED
mged> mater handle plastic
mged> Was
Parameter string? n
Override material color (y/n)[n]? y
R G B (0..255)? 219 147 112   NOTE:  This is color TAN
mged> r base + btm u btm1 u gus cyl - cyl1 m1 - hole2 -hole3 -hole4 u rad-
hole4
mged> error in number of args!   NOTE: Typing errors
mged> r base + btm u btm1 - hole2 - hole3 - hole4 u rad - hole4
Defaulting item number to 1005
Creating region id=1004, air=0, los=100, GIFTmaterial=1
dir_lookup:  could not find "hole3"
skipping hole3
dir_lookup:  could not find "hole4"
skipping hole4
dir_lookup:  could not find "hole4"
skipping hole4
mged> t
ball     cnr3     gus     knob/
base/     cor/     han     rad
btm     cor1/     handle     rad1
btm1     cut     head     rad2
cnr     cyl     hole     shaft
cnr1     cyl1     hole1     squ
cnr2     cyl2     hole2
mged> in hole 4 rcc 20 -150 0 0 0 6 3
mged> cp hole2 hole3
mged> killall base NOTE:  Redo "base" region
mged> r base + btm u btm1 - hole2 - hole3 - hole4 u rad - hole4
Defaulting item number to 1006
Creating region id=1005, air=0, los=100, GIFTmaterial=1
mged> g all base handle knob
mged> tree all
| all_________________| base_________| btm
                         | btm1
                         | hole2
                         | hole 3
                         | hole4
                         | rad
                         | hole4
                         | handle______________| squ
                                     | shaft
                                     | han
                                     | ball
                         | knob________________| ball
            | handle_______________| squ
                        | shaft
                        | han
                        | ball
            | knob_________________| ball
mged> l base
base (len 9) REGION id=1005 (air=0, los=100, GIFTmater=1)--
 + btm
 u btm1
 - hole2
 - hole3
 - hole4
 u rad
 - hole4
 u handle
 u knob
mged> l gus
gus:  ARB8 (ARB6)
1 (21.5000, -25.3000, 6.0000)
2 (21.5000, 0.0000, 6.0000)
3 (21.5000, 0.0000, 6.0000)
4 (21.5000, -25.3000, 36.0000)
5 (18.5000, -25.3000, 6.0000)
6 (18.5000, 0.0000, 6.0000)
7 (18.5000, 0.0000, 6.0000)
8 (18.5000, -25.3000, 36.0000)
mged> l ball
ball:  ELL
V (20.0000, 75.0000, 51.0000)
A (15.0000, 0.0000, 0.0000) Mag=15.000000
A dir cos=(0.0, 90.0, 90.0), rot=0.0, fb=0.0
B (0.0000, 15.0000, 0.0000) Mag=15.000000
B dir cos=(90.0, 0.0, 90.0) rot=90.0, fb=0.0
C (0.0000, 0.0000, 15.0000) Mag=15.000000
C dir cos=(90.0, 90.0, 0.0) rot=90.0, fb=90.0
mged> l knob
knob (len 1) REGION id=1001 (air=0, los=100, GIFTmater=1)--
Material "plastic"
Color 255 0 0
 + ball
mged> l handle
handle (len 4) REGION id=1000 (air=0, los=100, GIFT MATER=1)--
Material "plastic" "n
Color 219 147 112
 + squ
 u shaft
 u han
 u ball
mged> canter-0-75 0
mged> size 300
mged> tops
all/     cor1/     cyl2      hole1
cnr      cut     gus
cnr1     cyl     head
cor/     cyl1     hole
mged> analyze cyl
cyl:  TGC
V (20.0000, -45.0000, 6.0000)
H (0.0000, 0.0000, 30.0000) Mag=30.000000
H dir cos=(90.0, 90.0, 0.0), rot=90.0, fb=90.0
A (-17.5032, -9.6767, 0.0000) Mag=20.000000
B (9.6767,-17.5032, 0.0000) Mag=20.000002
c=20.000000, d=20.000002
AxB dir cos=(90.0, 90.0, 0.0), rot=90.0,fb=90.0
Surface Areas:  base(AxB)=1256.6371
  top(CxD)=1256.6371 side=3769.9114
Total Surface Area=6283.1855
   Volume=37699.1132 (0.0100 gal)
mged> q
\end{verbatim}
\chapter{RT MATERIAL TYPE, PROPERTIES, and COLOR}

First the solids must be formed into a "region", eg:

{\em\center
r ball u torus u tube-hole
}

To change material type, properties and color use the "mater" command:

{\tt
mged> {\em mater base} \\
Material = \\
Material?  (CR to skip) {\em plastic} \\
Param = \\
Parameter string? (CR to skip) {\em sh=10 dl=0.2 sp=0.8 re=0.75} \\
Color = (No color specified) \\
Color R G B (0..255)? (CR to skip) {\em 112 219 147} \\
Inherit = 0:  lower nodes (towards leaves) override \\
Inheritance (0|1)? (CR to skip) {\em 0} \\
mged> \\
}

For the values in Parameter String for material ``Plastic'',
you can enter such things as:
"shinyness (sh)",
"specular lighting fraction (sp)",
"diffuse lighting fraction (di)",
"transmission fraction (tr)",
"reflection fraction (re)", and
"refractive index (ri)".
Two formulas must hold to keep the material ``physical'':
sp + di=1.0, and tr + re=1.0.

Suggested values for these properties are listed below:

{\center sh=10, dl=0.2, sp=0.8, re=0.75}

NOTE:  Not all of these fields need to be input, you
can use the system defaults for the rest.

To display objects in different colors on the screen, each object must
be a region with its own material properties and colors.  All regions must be
displayed on screen before a ray tracing can be performed (region objects can
have cutouts to display other parts).

\begin{tabular}{r r r l}
R & G & B &         COLOR \\
112 & 219 & 147 & aquamarine \\
50 & 204 & 153 & med aquamarine \\
0 & 0 & 0 & black  \\
0 & 0 & 255 & blue  \\
95 & 159 & 159 & cadet blue  \\
66 & 66 & 111 & corn flower blue \\
107 & 35 & 142 & dk slate blue  \\
191 & 216 & 216 & light blue  \\
143 & 143 & 188 & light steel blue \\
50 & 50 & 204 & medium blue  \\
127 & 0 & 255 & medium slate blue \\
47 & 47 & 79 & midnight blue  \\
35 & 35 & 142 & navy blue  \\
50 & 153 & 204 & sky blue  \\
0 & 127 & 255 & slate blue  \\
35 & 107 & 142 & steel blue  \\
255 & 127 & 0 & coral  \\
0 & 255 & 255 & cyan  \\
142 & 35 & 35 & firebrick  \\
204 & 127 & 50 & gold  \\
219 & 219 & 112 & golden rod  \\
234 & 234 & 173 & med goldenrod  \\
0 & 255 & 0 & green  \\
47 & 79 & 47 & dark green  \\
79 & 79 & 47 & dk olive green  \\
35 & 142 & 35 & forest green  \\
50 & 204 & 50 & lime green  \\
107 & 142 & 35 & med forest green \\
66 & 111 & 66 & medium sea green \\
127 & 255 & 0 & med spring green \\
143 & 188 & 143 & pale green  \\
35 & 142 & 107 & sea green \\
0 & 255 & 127 & spring green \\
153 & 204 & 50 & yellow green \\
47 & 79 & 79 & dk slate grey \\
84 & 84 & 84 & dim grey \\
168 & 168 & 168 & light grey \\
\end{tabular}

\begin{tabular}{r r r l}
R & G & B &         COLOR \\
159 & 159 & 95 & khaki  \\
255 & 0 & 255 & magenta \\
142 & 35 & 107 & maroon \\
204 & 50 & 50 & orange \\
219 & 112 & 219 & orchid \\
153 & 50 & 204 & dark orchid \\
147 & 112 & 219 & medium orchid \\
188 & 143 & 143 & pink \\
234 & 173 & 234 & plum \\
255 & 0 & 0 & red \\
79 & 47 & 47 & indian red \\
219 & 112 & 147 & medium violet \\
255 & 0 & 127 & orange red \\
204 & 50 & 153 & violet red \\
111 & 66 & 66 & salmon \\
142 & 107 & 35 & sienna \\
219 & 147 & 112 & tan \\
216 & 191 & 216 & thistle \\
173 & 234 & 234 & turquoise \\
112 & 147 & 219 & dk turquoise \\
112 & 219 & 219 & med turquoise \\
79 & 47 & 79 & violet \\
159 & 95 & 159 & blue violet \\
216 & 216 & 191 & wheat \\
252 & 252 & 252 & white \\
255 & 255 & 0 & yellow \\
147 & 219 & 112 & green yellow
\end{tabular}

material types are:  plastic
                               mirror
                               glass
                               texture

Shinyness (ie:  sh=16)

Refractive index for:  crown glass = 1.52
                              Flint glass = 1.65
                              Rock salt = 1.54
                              Water = 1.33
                              Diamond = 2.42

Transmission fraction for a mirror:  re=1.0 (tr=0)

\chapter{RAYTRACING YOUR CREATION}

Once you have finished creating all your solids, positioned them in
their correct relationships to each other, formed all your regions (forming
your finished object), created groups (if required), you can now do a ray-
tracing of the view displayed on the screen.

Note!  If you want to display solids or objects (collection of solids
regioned together) of different colors, each of the solids or objects must be
separate regions so you can give them a specific color.

The raytracing command is
{\em\center
  rt [-s\#]
}

This command produces a color shaded image of the solids or objects on
the display.  This color shaded image  will appear on a frame buffer display.
The resolution of the image (number of rays) is equal to "\#" from the "-s"
option.  If the "-s" option is absent, 50x50 ray solution will be used (very
course raytrace).  The higher the "-s" option the better the raytracing, but
it takes longer to display.
Recommended optimun value of "-s" option for picture
quality and speed of display is 256!.  Some examples follow:

{\em
             rt -s128 \\
             rt \\
             rt -s256 \\
}

When the rt command is given the text and graphic window will appear,
then the frame buffer starts to appear (the picture window).  The first scaned
display will be what was previously stored in it, it will then over write it
with your picture; sometimes two buffer scans are displayed before yours.

The default background color is blue with steel grey colored solids and
objects.  The terminal will beep when the scanned picture is finished; press
return to get back to the "text and graphic" window.

With the blue background it is sometimes hard to visualize the raytraced
picture; two things you can do to improve the situation is:
     (a)  Make separate regions for all solids and objects, so that you can
assign a specific color to each region; this can be a time consuming task if
you have a lot of solids and objects.
     (b)  Construct as a separate region, three thin flat plates to form two
walls with a bottom, as shown below; using "make name arb8",
then solid editing this arb8, using move faces to the required thickness,
then use command "cp"
(copy command) to make two more copies which you can rotate to their
respective relationships, then translate all three into the correct positions
relative to each other and the solids and objects you are displaying.

The advantage of doing this is to give the light source something
to reflect off, giving back lighting; improving contrast considerably.
With the
three plates formed into their own region you can delete them from the screen
with the "d" command, rotate your creation then re-display your plates (walls)
with the "e" command to do another rt, the walls need to be deleted from the
display when you rotate your objects,
otherwise everything will rotate together.

                                   Figure

A bonus of having constructed these three walls is that you can quickly
change the material type to "mirror" so that you can get reflections of the
three hidden faces.

\chapter{CONCLUSIONS}

MGED performs two basic functions:
viewing and editing.
The standard viewing capabilities of zooming, slewing,
slicing, and rotation are available.
Likewise, all the standard editing features are also available.
The user easily traverses the hierarchical data structure, applying
the editing functions of rotation, translation, and scaling to any
position in the hierarchy.
The hierarchical structure can be modified and regrouped and regions
created and modified.
Specific parameter editing can also be applied to the solids to produce
any shape solid desired.

For several decades, the production and modification of geometric models
suitable for sophisticated engineering analysis
has been a slow, labor-intensive procedure.
In an effort to improve the response time of geometric models,
the Ballistic
Research Laboratory (BRL) has developed an interactive model editor
for their combinatorial solid geometry modeling system (The BRL-CAD Package).
The user interface to the geometry of these models
is a program called the Multi-device Graphics Editor (MGED)
that is designed to replace the
traditional manual method
for producing and modifying model databases.
Using MGED, the geometric models
are interactively viewed, modified, and constructed with immediate visual
feedback at each step.
When desired, the MGED editor can be operated without the need for
explicit numerical input
and opens a new dimension in the model building process.
MGED has made great gains in reducing the bottleneck in
the creation of high resolution geometric models.
