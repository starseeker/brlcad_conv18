\chapter{RT MATERIAL TYPE, PROPERTIES, and COLOR}

First the solids must be formed into a "region", eg:

{\em\center
r ball u torus u tube-hole
}

To change material type, properties and color use the "mater" command:

{\tt
mged> {\em mater base} \\
Material = \\
Material?  (CR to skip) {\em plastic} \\
Param = \\
Parameter string? (CR to skip) {\em sh=10 dl=0.2 sp=0.8 re=0.75} \\
Color = (No color specified) \\
Color R G B (0..255)? (CR to skip) {\em 112 219 147} \\
Inherit = 0:  lower nodes (towards leaves) override \\
Inheritance (0|1)? (CR to skip) {\em 0} \\
mged> \\
}

For the values in Parameter String for material ``Plastic'',
you can enter such things as:
"shinyness (sh)",
"specular lighting fraction (sp)",
"diffuse lighting fraction (di)",
"transmission fraction (tr)",
"reflection fraction (re)", and
"refractive index (ri)".
Two formulas must hold to keep the material ``physical'':
sp + di=1.0, and tr + re=1.0.

Suggested values for these properties are listed below:

{\center sh=10, dl=0.2, sp=0.8, re=0.75}

NOTE:  Not all of these fields need to be input, you
can use the system defaults for the rest.

To display objects in different colors on the screen, each object must
be a region with its own material properties and colors.  All regions must be
displayed on screen before a ray tracing can be performed (region objects can
have cutouts to display other parts).

\begin{tabular}{r r r l}
R & G & B &         COLOR \\
112 & 219 & 147 & aquamarine \\
50 & 204 & 153 & med aquamarine \\
0 & 0 & 0 & black  \\
0 & 0 & 255 & blue  \\
95 & 159 & 159 & cadet blue  \\
66 & 66 & 111 & corn flower blue \\
107 & 35 & 142 & dk slate blue  \\
191 & 216 & 216 & light blue  \\
143 & 143 & 188 & light steel blue \\
50 & 50 & 204 & medium blue  \\
127 & 0 & 255 & medium slate blue \\
47 & 47 & 79 & midnight blue  \\
35 & 35 & 142 & navy blue  \\
50 & 153 & 204 & sky blue  \\
0 & 127 & 255 & slate blue  \\
35 & 107 & 142 & steel blue  \\
255 & 127 & 0 & coral  \\
0 & 255 & 255 & cyan  \\
142 & 35 & 35 & firebrick  \\
204 & 127 & 50 & gold  \\
219 & 219 & 112 & golden rod  \\
234 & 234 & 173 & med goldenrod  \\
0 & 255 & 0 & green  \\
47 & 79 & 47 & dark green  \\
79 & 79 & 47 & dk olive green  \\
35 & 142 & 35 & forest green  \\
50 & 204 & 50 & lime green  \\
107 & 142 & 35 & med forest green \\
66 & 111 & 66 & medium sea green \\
127 & 255 & 0 & med spring green \\
143 & 188 & 143 & pale green  \\
35 & 142 & 107 & sea green \\
0 & 255 & 127 & spring green \\
153 & 204 & 50 & yellow green \\
47 & 79 & 79 & dk slate grey \\
84 & 84 & 84 & dim grey \\
168 & 168 & 168 & light grey \\
\end{tabular}

\begin{tabular}{r r r l}
R & G & B &         COLOR \\
159 & 159 & 95 & khaki  \\
255 & 0 & 255 & magenta \\
142 & 35 & 107 & maroon \\
204 & 50 & 50 & orange \\
219 & 112 & 219 & orchid \\
153 & 50 & 204 & dark orchid \\
147 & 112 & 219 & medium orchid \\
188 & 143 & 143 & pink \\
234 & 173 & 234 & plum \\
255 & 0 & 0 & red \\
79 & 47 & 47 & indian red \\
219 & 112 & 147 & medium violet \\
255 & 0 & 127 & orange red \\
204 & 50 & 153 & violet red \\
111 & 66 & 66 & salmon \\
142 & 107 & 35 & sienna \\
219 & 147 & 112 & tan \\
216 & 191 & 216 & thistle \\
173 & 234 & 234 & turquoise \\
112 & 147 & 219 & dk turquoise \\
112 & 219 & 219 & med turquoise \\
79 & 47 & 79 & violet \\
159 & 95 & 159 & blue violet \\
216 & 216 & 191 & wheat \\
252 & 252 & 252 & white \\
255 & 255 & 0 & yellow \\
147 & 219 & 112 & green yellow
\end{tabular}

material types are:  plastic
                               mirror
                               glass
                               texture

Shinyness (ie:  sh=16)

Refractive index for:  crown glass = 1.52
                              Flint glass = 1.65
                              Rock salt = 1.54
                              Water = 1.33
                              Diamond = 2.42

Transmission fraction for a mirror:  re=1.0 (tr=0)

\chapter{RAYTRACING YOUR CREATION}

Once you have finished creating all your solids, positioned them in
their correct relationships to each other, formed all your regions (forming
your finished object), created groups (if required), you can now do a ray-
tracing of the view displayed on the screen.

Note!  If you want to display solids or objects (collection of solids
regioned together) of different colors, each of the solids or objects must be
separate regions so you can give them a specific color.

The raytracing command is
{\em\center
  rt [-s\#]
}

This command produces a color shaded image of the solids or objects on
the display.  This color shaded image  will appear on a frame buffer display.
The resolution of the image (number of rays) is equal to "\#" from the "-s"
option.  If the "-s" option is absent, 50x50 ray solution will be used (very
course raytrace).  The higher the "-s" option the better the raytracing, but
it takes longer to display.
Recommended optimun value of "-s" option for picture
quality and speed of display is 256!.  Some examples follow:

{\em
             rt -s128 \\
             rt \\
             rt -s256 \\
}

When the rt command is given the text and graphic window will appear,
then the frame buffer starts to appear (the picture window).  The first scaned
display will be what was previously stored in it, it will then over write it
with your picture; sometimes two buffer scans are displayed before yours.

The default background color is blue with steel grey colored solids and
objects.  The terminal will beep when the scanned picture is finished; press
return to get back to the "text and graphic" window.

With the blue background it is sometimes hard to visualize the raytraced
picture; two things you can do to improve the situation is:
     (a)  Make separate regions for all solids and objects, so that you can
assign a specific color to each region; this can be a time consuming task if
you have a lot of solids and objects.
     (b)  Construct as a separate region, three thin flat plates to form two
walls with a bottom, as shown below; using "make name arb8",
then solid editing this arb8, using move faces to the required thickness,
then use command "cp"
(copy command) to make two more copies which you can rotate to their
respective relationships, then translate all three into the correct positions
relative to each other and the solids and objects you are displaying.

The advantage of doing this is to give the light source something
to reflect off, giving back lighting; improving contrast considerably.
With the
three plates formed into their own region you can delete them from the screen
with the "d" command, rotate your creation then re-display your plates (walls)
with the "e" command to do another rt, the walls need to be deleted from the
display when you rotate your objects,
otherwise everything will rotate together.

                                   Figure

A bonus of having constructed these three walls is that you can quickly
change the material type to "mirror" so that you can get reflections of the
three hidden faces.

\chapter{CONCLUSIONS}

MGED performs two basic functions:
viewing and editing.
The standard viewing capabilities of zooming, slewing,
slicing, and rotation are available.
Likewise, all the standard editing features are also available.
The user easily traverses the hierarchical data structure, applying
the editing functions of rotation, translation, and scaling to any
position in the hierarchy.
The hierarchical structure can be modified and regrouped and regions
created and modified.
Specific parameter editing can also be applied to the solids to produce
any shape solid desired.

For several decades, the production and modification of geometric models
suitable for sophisticated engineering analysis
has been a slow, labor-intensive procedure.
In an effort to improve the response time of geometric models,
the Ballistic
Research Laboratory (BRL) has developed an interactive model editor
for their combinatorial solid geometry modeling system (The BRL-CAD Package).
The user interface to the geometry of these models
is a program called the Multi-device Graphics Editor (MGED)
that is designed to replace the
traditional manual method
for producing and modifying model databases.
Using MGED, the geometric models
are interactively viewed, modified, and constructed with immediate visual
feedback at each step.
When desired, the MGED editor can be operated without the need for
explicit numerical input
and opens a new dimension in the model building process.
MGED has made great gains in reducing the bottleneck in
the creation of high resolution geometric models.
