\chapter{VIEWING FUNCTIONS}

The MGED viewing features are designed to allow one to examine an object
in close detail.  Any of the viewing features can be invoked at any time.
It should be noted, that these functions do not change the actual data,
only the way the data is displayed.

\section{Preset Views}

Six standard views (front, rear, top, bottom, left, and right) and one
oblique view (azimuth 35, elevation 25 isometric) are each assigned to the
function buttons, views 35 25 (isometric), top, right, front,
45 45 are available from the screen editor menu.  Hence, any of these views
is immediately available at the press of the appropriate function button or
mouse selection.  The views available are not limited to these standard views
however, as the display can be rotated to any view by using the dial box.  By
pressing the function button labeled ``save view'' or entering
the keyboard {\em saveview} command, the present view as displayed display is
saved (used for raytracing,
producing colored pictures, which will be discussed later).
At any time, the saved view can be immediately returned to the
screen by pressing the ``restore view'' function button.  The
``restore view'' button will be lit whenever a view has been saved.
The function button labeled ``reset'',
restores the display to the default view (front) when pressed.

\section{View Translation}

The displays can be panned or slewed on the screen in two ways -- using the
mouse pointer or by using the dial box knobs.  When one is editing, the mouse
functions are not available for slewing, hence one must use the dial box knobs
to slew the display.

To slew the display using the control knobs, one uses the knobs labeled
``slew x'' or ``slew y''.  The null positions on these knobs is in the
center or straight up.  If the ``slew x'' knob is turned clockwise of center,
the display will move to the right.  If it turned counterclockwise, the
display will move to the left.  For the ``slew y'' central knob, clockwise
of the center moves the display up and counterclockwise moves the display
down.  The further these knobs are turned from center, the faster the display
moves.

\section{View Zooming}

One can zoom the display by using the dial box knob labeled ``zoom''.
Again the null position of this knob is center or straight
up.  Turning this knob clockwise of center causes the display to increase in
size producing a zoom-in effect.  Turning this knob counterclockwise of center
causes the display to decrease in size or zoom-out.  Again, the further the
``zoom'' knob is turned from center, the faster the zooming will occur.

\begin{figure}
\centering \includegraphics{adc.ps}
\caption{The Angle Distance Cursor.}
\label{adc}
\end{figure}

\section{The Angle Distance Cursor (ADC)}

The angle distance cursor is a construction aid used to measure
angles and distances. It should be noted that all measurements are
made in the projected space of the screen, so one should measure only
in a view normal to the surface where the measurement is to take place.
The ADC is placed on (or removed from) the display by pushing the ``ADC''
button.
The ADC consists of three cursors which cover the entire screen.
Figure \ref{adc} depicts the ADC as it appears on the screen.
All the cursors are centered at the same point and can be moved to any
location on the screen.  Two of these cursors rotate for angle measuring
purposes. Angle cursor 1 is solid while angle cursor 2 is dashed.  Angle
cursor 1 has movable tic marks for measuring distances on the screen.
The two angle cursors move with the horizontal and vertical
lines of the main cursor.
The resulting effect is the moving of the center point
horizontally or vertically.
The ADC is controlled by the bottom row
of the (Megatek) knobs:

\begin{tabular}{rl}
Knob  & Function \\
 \\
6     & moves the center in the horizontal direction \\
7     & moves the center in the vertical direction \\
8     & rotates angle cursor 1  (alpha) \\
9     & rotates angle cursor 2  (beta) \\
10    & moves the tic marks
\end{tabular}

Whenever the ADC is on the screen, there is a readout at the bottom of
the screen listing pertinent information about the ADC.
This information includes the angles that angle cursors 1 and 2
have been rotated (alpha and beta), the distance the tic marks are
from the center of the ADC, and the location of the center of the ADC.
This information is continually updated on the screen.

\chapter{MGED EDITING FEATURES}

The heart of the MGED system is its editing features.
The editing features are divided into two classes: object editing
and solid editing.
Object editing is designed to allow one to change the location,
size, and orientation of an object.
Recall that an object is defined as the basic data unit of the
MGED system and includes both combinations and solids.
In the case of a solid, one needs to change not only its location,
size, and orientation, but also its ``shape''.
Changing the shape of a solid means changing any of its individual parameters.
Hence, solid editing is handled separately.

\section{Combination Editing (OBJECT EDIT)}

Before being able to enter the OBJECT EDIT state
(i.e. edit non-terminal),
it is necessary to pass through two intermediate states
in which the full path of an object to be edited is specified,
and the location of one arc along that path is designated for editing.
It is possible to create a transformation matrix to be applied
above the root of the tree, affecting everything in the path,
or to apply the matrix between any pair of nodes.
For example, if the full path /car/chassis/door is specified,
the matrix could be applied above the node ``car'', between
``car/chassis'', or between ``chassis/door''.

The transformation matrix to be applied at the
designated location can be created by the concatenation of
operations, each specified through several types of user direction.
Trees can be rotated around the center of the viewing cube;
this rotation can be specified in degrees via keyboard command, or can
be controlled by the rotation of a set of control dials or motions on
a three-axis joystick.
Translation of trees can be specified in terms of a precise new location
via keyboard command, or by adjusting a set of control dials.
Tree translation can also be accomplished by pointing and
clicking with the mouse or tablet.
Uniform and single-axis (affine) scaling of a tree can be controlled
by a numeric scale factor via keyboard command, or by way of repeated
analog scaling by pointing and clicking with the mouse or tablet.
Before we discuss the editing features of MGED, we will discuss
how one selects objects for editing.

\section{Selecting Objects For Editing}

To select a displayed object for editing, press the object illuminate button
or select ``Object illum'' from the ***BUTTON MENU***.
The object selection is a two step process.

Whenever an object is displayed (using the {\em e} command), all paths in the
object's hierarchy are traversed recursively, accumulating the transformation
matrices.  When the bottom of the path (a solid) is encountered, the
accumulated
transformations are applied to the solid's parameters and the solid is drawn.
Thus every solid displayed is really a path ending with that solid.
If the object has been displayed using the {\em E} command, the same procedure
is followed, but only until a region is encountered.
Then all members of the region have the accumulated transformations
applied and the region is then ``evaluated'' and drawn.

In the first step of the object selection process, the path is selected.
Again, the data tablet is divided in as many horizontal sections as there are
paths drawn.
The path (solid or evaluated region) corresponding to the horizontal
section the pen/mouse is located in will be illuminated (brighter on B/W
displays and white on color displays).
This complete path is also listed on the display.
When the pen/mouse is pressed the illuminated path is selected.

In the second step, a member of the selected path is chosen.
All editing will then be applied to this member.
The tablet is divided into as many horizontal sections as there are
path members.
The word ``[MATRIX]'' is used to illuminate path members and will appear
above the member corresponding to the location of the pen/mouse.
Pressing the pen/mouse when the desired path member is ``illuminated''
will put MGED in the object edit state.
The editing will be performed on the path member selected.

If a solid is located at the bottom of this path, it becomes the key
solid and its vertex becomes the key point.
If an evaluated region is at the bottom of the path, the center of
this region becomes the key point.
All object editing is done with respect to this key point.

The object editing features can be invoked in any order and at any time
once an object has been selected for editing.  During object editing, any of
the viewing features, such as changing views, zooming, and slewing, can be
used, and in fact, are usually quite useful.  Again, the only way to exit the
object editing mode is to ``accept'' or ``reject'' the editing.
If the ``reject'' button is pressed (or selected from the edit menu), the
object will return to its pre-edit state.  If the ``accept'' button is pressed
(or selected from the edit menu), the data base will be changed to reflect the
object editing performed.

\section{Object Edit State}

When MGED enters the object edit state, the following occurs:

\begin{enumerate}
\item all the solids/evaluated regions of the edited object
become illuminated
\item the key solid's parameters are labeled  OR \\
the center of the key evaluated region is marked
\item the key solid's parameters are listed and
continually updated  OR \\
the key evaluated region's center is listed and
continually updated
\item the ***OBJ EDIT*** menu is displayed
\end{enumerate}

\section{Translate An Object}

There are three ways to translate an object:  translate in the screen
X direction only (X move), translate in the screen Y direction only
(Y move) or just straight translation (XY move).
In all cases, the complete object is translated so that the ``key point''
is positioned at the desired location.
The {\em translate} command is used to enter a precise location (x,y,z) for
the key point.
Entering {\em translate x y z} will move the complete object so that the key
point will be at coordinates (x,y,z).

\section{Rotate An Object}

Rotation of the object may be accomplished by selecting the ``Rotate''
menu item, or pressing the ``Rotate'' button.
Turning the knobs results in the object being rotated.
The {\em rotobj x y z} command can be used here, to specify
a precise rotation in degrees.
While in this edit state, the only way to rotate view is to use
the {\em vrot x y z} command.

\section{Scale An Object}
\subsection{Global Scale}

To select global object scale, press the object scale button or select
``Scale'' from the ***OBJ EDIT*** menu.
When the pen/mouse is pressed, the edited object is scaled about the key
point by an amount
proportional to the distance the pen/mouse is from the center of the screen.
If the pen/mouse is above the center, the edited object will become larger.
If it is below the center, the object will become smaller.
The {\em scale} command can be used to enter precise scale factors.
The value entered is applied to the object as it existed when object
scale was entered.
Hence entering {\em scale 1} will return the object to its size when the
object scale session first started.

\subsection{Local Scale}

Local object scaling is allowed about any of the coordinate axes.
To select local scaling, press one of the buttons (OBJ Scale X, OBJ Scale Y,
or OBJ Scale Z) or select ``Scale X'', ``Scale Y'', or ``Scale Z'' from
the ***OBJ EDIT*** menu.
When the pen/mouse is pressed, the edited object is scaled in the selected
coordinate axis only, about the key point.
The amount of scaling is proportional to the distance the pen/mouse is
from the center of the screen.
If the pen/mouse is above the center, the edited object will become larger
in the selected axis direction.
If it is below the center, the object will become smaller in the selected
axis direction.
The {\em scale} command can be used to enter precise scale factors.
The value entered is applied to the object as it existed when local object
scale was entered.
Hence entering {\em scale 1} will return the object (in the selected axis
direction) to its size when the object scale session first started.

\section{Solid Editing}

There are two classes of editing operations that can be performed on
leaf nodes, the primitive solids.
The first class of operations are generic operations which can be applied to
any type of solid, and the second class of operations are those operations
which are specific to a particular type of solid.
Generic operations which can be applied to all primitive solids are
rotation, translation and scaling.
Recall that primitives can be treated as any other object and ``object
edited'' as detailed above.
Each primitive solid also has a variety of editing operations available that
are specific to the definition of that solid.  These operations are
detailed below.

The solid editing mode is necessary to
perform to basic shapes of solids.
Precise modifications
of the shape are possible (using the {\em p} keyboard command) in the solid
editing mode.

The solid editing feature allows the user to interactively translate,
rotate, scale, and modify individual parameters of a solid.  Whenever one is
in the solid edit mode, the parameters of the solid being edited are listed
and continually updated at the top of the screen.  Certain parameters are
also labeled on the solid being edited.  Solid editing  is generally used to
``build'' objects by producing solids of the desired shape and size in the
correct orientation and position.  Once the object is built, object editing
is used to scale, orient, and position the object in the description.  The
general philosophy of solid editing is to first create a solid with the
desired name and then to edit this solid.  As an example, suppose one were
to build an object called ``BRACKET''; to produce the base of the object the
primitive solid type ARB8 (see Figure 1) would be used along with either the
{\em in} command or {\em make} command, so one would type:
\begin{verse}
     in btm box 0 0 0  0 -90 0  40 0 0  0 0 6 \\
     make block arb8
\end{verse}
A new solid called ``btm'' or ``block'' would be created and displayed on the
screen.  These solids would then be edited using solid editing to produce the
solid parameters for the shape desired.

\section{Selecting Solids For Editing}

The procedure for solid editing is quite similar to that for object editing.
First, solid edit state must be entered, by pressing the
``solid illuminate'' button, or selecting the ``solid illum'' menu item.
Second,
A solid is selected for
editing using the illuminate mode, just as in object editing,
by moving the cursor up and down, and choosing the desired solid.
The solid data is listed at the top of the screen and a
header depending on the solid type is written above the solid editing data.
Third, select the appropriate function button or edit menu operations,
and perform the editing desired.  Finally, the solid
editing mode is exited
by either accepting or rejecting the editing performed.

A solid must be displayed before it can be picked for editing.
To pick a displayed solid for editing, press the ``solid illum'' button or
select ``Solid Illum'' from the ***BUTTON MENU***.
The data tablet and pen/mouse are then used to pick the solid.
The surface of the data tablet is divided into as many horizontal sections
as there are solids displayed.
The displayed solid corresponding to the horizontal section the pen/mouse
is located in will be ``illuminated'' (it will become brighter on black and
white devices and white on color devices).
The complete hierarchical path to reach the solid is also listed on the
display.  When the pen/mouse is pressed, MGED enters the solid edit state
with the illuminated solid as the solid to be edited.
If the solid is not multiply referenced, entering {\em sed solidname} on
the keyboard will immediately put MGED in the solid edit state with
{\em solidname} as the edited solid.

\section{Solid Edit State}

When MGED enters the solid edit state, the following occurs:

\begin{enumerate}
\item the edited solid remains illuminated
\item the edited solid's parameters are labeled
\item the edited solid's parameters are listed
\item  (and continually updated)
\item the ***SOLID EDIT*** menu is displayed
\item the parameter edit menu is initially displayed (default)
\end{enumerate}

\section{Rotate A Solid}

Solid rotation allows the user to rotate the solid being edited to any
desired orientation.  The rotation is performed about the vertex of the
solid.  To select this option, one presses the function button labeled
``solid rotate'' or selects from the edit menu on screen.
The rotation can be done using the dial box or one can input exact angles
of rotation of the solid by using the {\em p} keyboard command.
For example, typing:
{\em \center
p alpha beta gamma
}
will rotate the solid {\em alpha} degrees about the x-axis, {\em beta} degrees
about the y-axis and {\em gamma} degrees about the z-axis.  Alpha, beta, and
gamma are measured from the original ``zero'' orientation of the solid,
defined when the ``solid edit'' function button was
pressed.  Hence, typing
{\em \center
p 0 0 0
}
will always return the solid to its original position (its position when the
current solid editing session began) before accepting edit.

To select solid rotation, press the solid rotate button or select ``Rotate''
from the ***SOLID EDIT*** menu.
The joy stick or appropriate rotation knobs then will rotate the edited solid
about the coordinate axes.
The solid is rotated about its vertex.
The parameter (p) command can be used to make precise rotation changes.
The values entered after the p are absolute -- the rotations are applied
to the solid as it existed when solid rotation was first selected.
Thus entering {\em p 0 0 0} will ``undo'' any rotations performed since
solid rotation was selected.
The rotation about the z-axis is done first, then the y, then the x.

\section{Translate A Solid}

Solid translation allows the user to place the solid being edited anywhere
in the description.  To invoke this option, one presses the function
button labeled ``solid trans'' or selects from the screen edit
menu.  To move the solid, use the mouse pointer to position the solid and
click the center mouse button.  Whenever the mouse button is pressed, the
VERTEX of the solid moves to that location on the screen.

One can read the actual coordinates of the vertex on the top of the
screen, along with other data.  If the actual desired coordinates of the
vertex are known, one can place the solid exactly using the {\em p} keyboard
command.  For example, to place a solid's vertex at the coordinates (x,y,z)
one would type:
{\em \center
p40 20 10
}
The solid would then jump to this location.

To select solid translation, press the solid translate button or
select ``Translate'' from the ***SOLID EDIT*** menu.
When the pen/mouse is pressed, the vertex of the edited solid will
move to that location.
The parameter (p) command can be used to translate the solid to
a precise location.
Entering {\em p x y z} will place the vertex of the edited solid at (x, y, z).

\section{Scale A Solid}

The solid SCALE feature allows the user to scale the solid being to any
desirable size.  The scaling is done about the vertex of the solid, hence NO
translation of the solid occurs.  The scaling is performed using the mouse
pointer and clicking the center mouse button, just as in object scaling.  One
can input an exact scale factor using the {\em p} keyboard command, in the form
of.  For example, typing
{\em \center
p factor
}
will scale the solid by an amount equal to {\em factor}.   The value of
{\em factor} is absolute -- the original solid is scaled.  By setting {\em factor}
equal to one (1), the original size solid will be displayed on the screen
before accepting your edit.

To select solid scale, press the solid scale button or select ``Scale''
from the ***SOLID EDIT*** menu.
When the pen/mouse is pressed, the edited solid is scaled by an amount
proportional to the distance the pen/mouse is from the center of the screen.
If the pen/mouse is above the center, the edited solid will become larger.
If it is below the center, the solid will become smaller.
The parameter (p) command can be used to enter precise scale factors.
The value entered is applied to the solid as it existed when solid
scale was entered.
Hence entering {\em p 1} will return the solid to its size when solid scale
session first started.

\section{Solid Parameter Editing}

To modify individual solid parameters, press the menu button or select
``edit menu'' from the ***SOLID EDIT*** menu.
A menu listing what parameter editing is available for that particular
solid type will be displayed.
Using the pen/mouse select the desired item(s) from this menu.
For most of the parameter editing, the {\em p} command can be used to
make precise changes.
Parameter editing is the default edit mode entered when MGED first
enters the solid edit state.
In the following paragraphs, we will discuss parameter editing
for each of the MGED general types of solids.

\begin{figure}
\centering \includegraphics{menu-arb-ctl.ps}
\caption{ARB Control Menu.}
\label{menu-arb-ctl}
\end{figure}

\subsection{ARB Parameter Editing}

The GENERAL ARB class of solids represents all the convex polyhedrons
(RPP, BOX, RAW, and ARBs).
The ARBs comprise five classes of polyhedrons each with a characteristic
number of vertices.
These are the ARB8, ARB7, ARB6, ARB5, and ARB4, where the ARB8 has
eight vertices, etc.
During editing, all the vertices are labeled on the screen.

An ARB is defined by a fixed number of vertices where all faces must
be planar.  This fact means that during parameter editing, movement
of individual vertices in faces containing four vertices is not allowed.
There are three classes of parameter editing that can be done to ARBs:
move edges,
move faces, and rotate faces.  There is an ``ARB control menu''
(see Figure \ref{menu-arb-ctl}) from
which one selects the type of parameter editing to be done.
A specific ARB edit menu will appear dependent on which parameter editing
option was selected.  The ``return'' entry on each of these specific menus
will return the ``ARB control'' menu to the screen.

Note that there are several keyboard commands that apply only to ARB solids
which are being edited in SOLID EDIT state.
Once such command is {\em mirface}, which replaces a designated
face of the ARB with a copy of an original face mirrored about
the indicated axis.
Another such command is {\em extrude}, which projects a designated face
a given amount in the indicated direction.

\begin{figure}
\centering \includegraphics{menu-arb8-edge.ps}
\caption{Move Edge Menu for ARB8.}
\label{menu-arb8-edge}
\end{figure}

\begin{figure}
\centering \includegraphics{menu-arb4-edge.ps}
\caption{Move Edge Menu for ARB4.}
\label{menu-arb4-edge}
\end{figure}

\begin{figure}
\centering \includegraphics{menu-arb8-face.ps}
\caption{Move Face Menu for ARB8.}
\label{menu-arb8-face}
\end{figure}

\begin{figure}
\centering \includegraphics{menu-arb4-face.ps}
\caption{Move Face Menu for ARB4.}
\label{menu-arb4-face}
\end{figure}

\subsection{Move ARB Edges}

To move an ARB edge, select the desired edge from the ``move edge'' menu.
For example, Figure \ref{menu-arb8-edge} shows the menu for
moving an edge of an ARB8, and Figure \ref{menu-arb4-edge}
shows the menu for moving an edge of an ARB4.
A point is then ``input'' either through a pen press or through the {\em p}
command.
The line containing the selected edge is moved so that it goes through
coordinate of the input point.
Any affected faces are automatically adjusted to remain planar.

\subsection{Move ARB Faces}

To move an ARB face, select the desired face from the ``move face'' menu.
A point is then ``input'' either through a pen press or through the {\em p}
command.  The plane containing the edited face is then moved so that it
contains the input point.  The new face is then calculated and the ARB
is displayed.
The move face menus for an ARB8 are shown
in Figure \ref{menu-arb8-face}, and the move face menus for an ARB4
are shown in Figure \ref{menu-arb4-face}.

\begin{figure}
\centering \includegraphics{menu-arb8-rot.ps}
\caption{Rotate Face Menu for ARB8.}
\label{menu-arb8-rot}
\end{figure}

\begin{figure}
\centering \includegraphics{menu-arb4-rot.ps}
\caption{Rotate Face Menu for ARB4.}
\label{menu-arb4-rot}
\end{figure}

\subsection{Rotate ARB Faces}

ARB faces may be rotated around any of the vertices comprising that face.
First, select the desired face from the ``rotate face'' menu.  You will then
be asked to select the vertex number around which to rotate the face.
The face can be rotated about the three coordinate axes.  The knobs (Rotate X,
Rotate Y, and Rotate Z) are used for this purpose.  For precise rotations,
use the {\em p} command.  If three values are entered after the {\em p}, then
they are interpreted as angles (absolute) of rotation about the X, Y, Z axes
respectively.  If only two values are entered, then they are considered as
rotation and fallback angles for the normal to that face.  The {\em eqn}
command can also be used here to define the plane equation coefficients of
the face being rotated.
The rotate face menus for an ARB8 are shown
in Figure \ref{menu-arb8-rot}, and the rotate face menus for an ARB4
are shown in Figure \ref{menu-arb4-rot}.

\begin{figure}
\centering \includegraphics{ped-tgc.ps}
\caption{Typical TGC During Parameter Editing.}
\label{ped-tgc}
\end{figure}

\subsection{Truncated General Cone (TGC) Parameter Editing}

The TGC general class of solids includes all the cylindrical COMGEOM solids.
The defining parameters of the TGC are two base vectors (A and B), a height
vector (H), two top vectors (C and D), and the vertex (V).
The top vectors C and D are directed the same as the base vectors A and
B respectively,
hence the top vectors are defined only by their lengths (c and d).
During solid editing, only vectors A and B are
labeled on the display.
Figure \ref{ped-tgc} depicts a typical TGC during parameter editing.

It is possible to change the length of the H, A, B, C, or D
vectors, resulting in a change in height or eccentricity of the
end plates.  The overall size of the A,B or C,D end plates can
be adjusted, or the size of both can be changed together, leaving
only the H vector constant.
The H vector or the base plate (AXB) can be rotated.
Recall that vectors A \& C and vectors B \& D have like directions, hence
rotating the base (AXC) will automatically rotate the top (BXD).
Finally, one can move the end of the height vector H
with the TGC becoming or remaining
a right cylinder (move end H (rt)),
or with the orientation of the base (and top)
unchanged (move end H).
Either the mouse/tablet or the {\em p} command can be used.
These functions are selected from the menu which can be seen
in Figure \ref{ped-tgc}.

\begin{figure}
\centering \includegraphics{ped-ell}
\caption{Ellipsoid Parameter Editing Menu.}
\label{ped-ell}
\end{figure}

\subsection{Ellipsoid Parameter Editing}

The ELLG general class represents all the ellipsoidal solids, including
spheres and ellipsoids of revolution.
The defining parameters of the ELLG are three mutually perpendicular
vectors (A, B, and C) and the vertex (V).
When an ELLG is being edited, only vectors A and B are labeled on the display.
Figure \ref{ped-ell} depicts a typical ELLG during parameter editing.

The parameter editing of the ELLG consists of scaling the lengths of the
individual vectors A, B, C.
One may also scale all theses vectors together of equal length.

The scaling of these vectors is done using the data tablet/mouse in
exactly the same manner as in object scaling.
The {\em p} keyboard command again can be used to produce a vector of
desired length.

\begin{figure}
\centering \includegraphics{ped-tor}
\caption{Torus Parameter Editing Menu.}
\label{ped-tor}
\end{figure}

\subsection{Torus Parameter Editing}

The TOR general class of solids contains only one type of torus, one with
circular cross-sections.
The defining parameters of the TOR are two radii (r1 and r2), a normal
vector (N), and the vertex (V).
The scalar r1 is the distance from the vertex to the midpoint of the
circular cross section.
The scalar r2 is the radius of the circular cross-section.
The vector N is used to orient the torus.
During solid editing, none of these parameters are labeled on the screen.
Figure \ref{ped-tor} depicts a typical torus during parameter editing.

The parameter editing of the TOR consists of scaling the radii, hence the
menu contains only two members.
